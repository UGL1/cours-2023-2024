\documentclass[11pt,a4paper,article]{nsi}
\pagestyle{empty}
\begin{document}
\titre{Capture The Flag - Cybersécurité}
\classe{}
\maketitle

\textit{Résumé des documents qui ont été transmis à M. Leleu:}\\

Un groupe d’étudiants ingénieurs en cybersécurité à l’École Nationale Supérieure d’Ingénieurs de Bretagne Sud (ENSIBS) de Vannes organise un CTF (Capture The Flag), compétition de cybersécurité durant laquelle les collégiens et lycéens pourront s’initier par équipes aux multiples domaines de la cybersécurité (osint, sécurité web, stéganographie, cryptographie, programmation, etc.) à travers des séries de challenges ludiques. Résoudre ces challenges leur permettra non seulement de gagner des points et progresser vers le haut du classement, mais aussi de monter en compétences sur une thématique de la cybersécurité et tenter de gagner des lots.

De par sa date, le 11 novembre 2023, le thème de cette édition du NoBracketsCTF sera la Première Guerre mondiale. Ce sera l’occasion d’allier histoire, cybersécurité et amusement.

Ce CTF se déroulera en ligne du 11 au 12 novembre 2023. Il sera supervisé en ligne  par les élèves ingénieurs afin de venir en aide aux élèves et de permettre à chacun d’en tirer satisfaction, quels que soient son lieu de résidence ou ses connaissances. Les élèves n’auront besoin d’aucun prérequis pour participer.

Au terme de ce CTF, un événement aura lieu à l’ECW (European Cyber Week, un salon de cybersécurité à Rennes) entre les meilleures équipes, en collaboration avec le Pôle d’Excellence Cyber pour l’organisation de ce CTF. C’est une entité composée de plus de 100 entreprises, universités et laboratoires qui ont compris que la cybersécurité est un enjeu de demain, mais aussi d’aujourd’hui.

Dans le but de leur présenter plus précisément le monde de la cybersécurité et de leur proposer de participer à cet événement, les étudiants désirent venir au lycée exposer leur projet aux élèves.\\

\textbf{Public visé :}\\ SNT / NSI première et terminale\\

\textbf{Finalité :}\\ Monter en compétence sur le thème de la cybersécurité\\

\textbf{Modalités : }
\begin{itemize}
    \item une intervention au lycée, menée par les organisateurs de l'événement, au mois de septembre si possible 
    \item un week-end de travail à la maison (11 et 12 novembre)\\
\end{itemize} 

\textbf{Besoins matériels : }\\
Pour la réunion de présentation au lycée, un ou deux créneaux d'une heure et demi, avec un ordinateur, un micro, un accès internet, pour exposer ce qu'est un CTF aux élèves.
\end{document}