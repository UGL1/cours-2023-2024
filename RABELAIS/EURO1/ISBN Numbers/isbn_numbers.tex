\documentclass[12pt,a4paper]{nsi}
\begin{document}
\pagestyle{empty}
\titre{ISBN Numbers}
\classe{Euro 1\ere}
\maketitle

Many codes have been designed for use with new technology. These include bar codes, ISBN numbers, ASCII codes,
post codes, bank account codes. Many of these modern codes rely on a checking system, often referred to as a check
digit.\\

An example of this is the ISBN (International Standard Book Number) numbers, used universally on all books
published since 1972.\\

When ordering a book you usually give the author and the title. Sometimes it's more convenient to give its ISBN number, to make sure you get the exact edition you are looking for.\\

\picright{0.3}{img/isbn}{
    An ISBN is preceded by letters ISBN and has ten digits made up from components, as illutrated opposite. The digits are arranged according to the number of digits in each
    component part, with a space or a hyphen between each part. The check digit is designed so that any one error in the previous nine digits is spotted.

}\medskip\par
It is calculated so that any \textit{single} error in the previous nine digits is spotted, in the following way :\\

Multiply the first nine numbers by 10, 9, 8, ... , 2 respectively and find the
sum of the resulting numbers. The check digit is the smallest number that
needs to be added to this total so that it is exactly divisible by 11. \\

For the example above, we have
$0\times 10 + 8 \times 9 + 5 \times 8 +0\times 7 + 2 \times 6 + 0 \times 5 + 0 \times 4 + 1 \times 3 + 4 \times 2 = 135$
so the check digit must be 8, since 143 is divisible by 11.
Note that if the number 10 is needed for the check digit, the symbol X is used.\\

\question Determine the check digit $a$ for the following ISBN number 186993100 $a$. Explain your answer.\\

\question There is exactly one error in the following ISBN number: 1869932238. Can you correct this error ? Explain your answer.\\

\question Apply the following \textsc{Python} program to the ISBN number: 1834721164 .
\newpage
\begin{pyc}
    \begin{minted}{python}
        # create a list with nine 0 in it
        d = [0] * 9

        # repeat for i in 0, ..., 8
        for i in range(9):
        
            # ask the user for a digit and put it in the list
            d[i] = int(input("Enter a digit : "))
        
        # calculate the total
        total = 10 * d[0] + 9 * d[1] + 8 * d[2] + 7 * d[3] + 6 * d[4] 
                + 5 * d[5] + 4 * d[6] + 3 * d[7] + 2 * d[8]
       
        check_digit = 0
        
        # while total is not divisible by 11
        while total % 11 != 0:
            check_digit = check_digit + 1
        
        print(check_digit)
    \end{minted}
\end{pyc}

\question Change this program so that it will detect a mistake in a given ISBN number: it should display a message which says whether there is an error in the check digit (and therefore in the ISBN number) or not.\\

\question Why can this method detect only single errors ? Give an example of two valid ISBN numbers, the latter derived from the former by changing two digits.\\

\begin{center}
    

\tabstyle[UGLiBlue]
\begin{tabular}{|c|c|c|c|c|c|c|c|}
    110 & 121 & 132 & 143 & 154 & 165 & 176 & 187 \\
    198 & 209 & 220 & 231 & 242 & 253 & 264 & 275 \\
    286 & 297 & 308 & 319 & 330 & 341 & 352 & 363 \\
    374 & 385 & 396 & 407 & 418 & 429 & 440 & 451 \\
    462 & 473 & 484 & 495 & 506 & 517 & 528 & 539 \\
    550 & 561 & 572 & 583 & 594 & 605 & 616 & 627 \\
    638 & 649 & 660 & 671 & 682 & 693 & 704 & 715 
\end{tabular}\bigskip\par \textit{A fragment of the 11 times table.} 
\end{center}
\end{document}