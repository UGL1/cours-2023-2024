\documentclass[12pt,a4paper,article,english,firamath]{nsi}
\pagestyle{empty}
\usepackage{fontawesome5}
\begin{document}
{\Large\titlefont FRÉDÉRIC LELEU}\\
{\color{UGLiBlue}\textbf{Mathematics and Computer Science Teacher}}\\

{\color{UGLiBlue}\faEnvelope} 29, avenue d'Armorique, 22000 Saint Brieuc\\
{\color{UGLiBlue}\faAt} frederic.leleu@ac-rennes.fr\\
{\color{UGLiBlue}\faPhone*} 06 99 13 08 41\\

%{\color{UGLiBlue}\faCalendar*}


\begin{multicols}{2}
{\color{UGLiBlue}\large\titlefont TEACHING EXPERIENCE\\[-1em]\hrule}

\bigskip Teaching Assistant \\
{\color{UGLiBlue}\textbf{EPITA }}\\
{\color{lightgray}\faCalendar*  1998 - 2000\\ \faMapMarker* Paris-Villejuif (94)}

\bigskip Math teacher \\
{\color{UGLiBlue}\textbf{Éducation Nationale}}\\
{\color{lightgray}\faCalendar*  2000 - Present\\ \faMapMarker*\small Argenteuil (95), Ermont (95), Saint Brieuc (22)}

\bigskip CS teacher \\
{\color{UGLiBlue}\textbf{Éducation Nationale}}\\
{\color{lightgray}\faCalendar*  2013 - Present\\ \faMapMarker* Ermont (95), Saint Brieuc (22)}

\bigskip Python Trainer \\
{\color{UGLiBlue}\textbf{Académie de Rennes}}\\
{\color{lightgray}\faCalendar*  2017 - 2018\\ \faMapMarker* Rennes (35)}

\bigskip CS Teacher / Educator \\
{\color{UGLiBlue}\textbf{INSPÉ de Bretagne}}\\
{\color{lightgray}\faCalendar*  2022 - 2023\\ \faMapMarker* Rennes (35)}



\columnbreak
{\color{UGLiBlue}\large\titlefont EDUCATION\\[-1em]\hrule}


\bigskip Maîtrise de Mathématiques Pures \\
{\color{UGLiBlue}\textbf{Université Pierre et Marie Curie Paris 6}}\\
{\color{lightgray}\faCalendar*  1996\\ \faMapMarker* Paris (75)}

\bigskip CAPES de Mathématiques \\
{\color{UGLiBlue}\textbf{Université Pierre et Marie Curie Paris 6}}\\
{\color{lightgray}\faCalendar*  2000\\ \faMapMarker* Paris (75)}

\bigskip Agrégation de Mathématiques \\
{\color{UGLiBlue}\textbf{Université Paris-Saclay}}\\
{\color{lightgray}\faCalendar*  2009\\ \faMapMarker* Orsay (91)}

\bigskip DIU Formation à la Science Informatique - ISN \\
{\color{UGLiBlue}\textbf{Université Versailles Saint-Quentin}}\\
{\color{lightgray}\faCalendar*  1996\\ \faMapMarker* Versailles (78)}

\bigskip Enseigner l'informatique au lycée \\
{\color{UGLiBlue}\textbf{Université Rennes 2}}\\
{\color{lightgray}\faCalendar*  2020\\ \faMapMarker* Rennes (35)}

\end{multicols}
\ \\
\begin{multicols}{2}
{\color{UGLiBlue}\large\titlefont LANGUAGES\\[-1em]\hrule}
\medskip 
French: Native Speaker
English : Level C1\\
Spanish : Level B2\\
Slovak : Level A2\\

\columnbreak

{\color{UGLiBlue}\large\titlefont INTERESTS\\[-1em]\hrule}
\medskip 
Programming, Music, Reading (mainly in English)

\end{multicols}
\newpage
I started teaching mathematics almost 25 years ago, but I've been programming on an almost daily basis since I was 13.\\
I remember 15 years ago or so when I decided to explain to my math students how a sorting algorithm works (maybe I was getting a bit bored of teaching the same math again and again, and never getting any question from my pupils). I was amazed by the way they reacted to the class : they seemed much more focused than usual and teemed with numerous questions, each one more interesting than the other.\\
So when ISN was created I embraced the opportunity with a lot of enthusiasm and now that NSI is here to stay  I spend most of my time teaching computer science. Officially, I'm still teaching "Mathématiques pour l'Informatique" in BTS SIO, but I try to colorate this math teaching as much as I can with computer science and I like to think that most of my students are grateful for this.\\

This year I started teaching mathematics in English (mainly acting as a substitute for a colleague who has health issues) and until now it has been a pleasant, rewarding experience. You will find below an example of what I'm actually doing, which I'd be glad to explain.\\
Right now, my main concern is that I'm not yet able to gauge my pupils' level in English. I obviously try to speak as much English as possible but sometimes, I guess It would be better to explain some aspects of the problem we're trying to solve in French.\\

However, the main reason that pushes me to get this certification is Computer Science. Indeed, it abounds with technical terms, most of which are in English, so when teaching I constantly find myself switching from French to English, then back to French and so on. With this certification I could offer my students the possibility to have a whole lesson in English (which is in my opinion a really good option considering the way technique and English are intertwined). I consider myself lucky : when my pupils are working with computers, their group is split in two. As I'm sure at least half of them would like to try a whole class in English maybe we could dedicate one of the  subgroups to this.

\newpage 
\titre{A way to look for words online}
\classe{Euro 1\ere}
\maketitle

Let's assume we have to look for the translation of a mathematical term.\\ Where do we begin ?\\ How can we be (fairly) sure we have the right translation ?

For instance : how to we say "le milieu (d'un segment)" in English ?

\section*{Don't be afraid of \textsc{ChatGpt}}

Go to \texttt{https://chat.openai.com/} and sign up : use one of your email accounts and chose a password.\\
Now ask your question in english. Don't worry, even if you are a bit clumsy, \textsc{ChatGpt} will do its best to understand. For instance you could ask:\\ "How do you call the point which is right in the middle of a segment ?".\\
Here's what \textsc{ChatGpt} answers :\\

\texttt{\noindent
    The point that is right in the middle of a segment is called the\\
    "midpoint." The midpoint is equidistant from both endpoints of \\
    the segment and divides the segment into two equal halves. It is
    a \\
    fundamental concept in geometry and is often denoted by the
    letter \\
    "M" or by other notations such as "P" or "Q" depending on
    the\\
    context.}\\

So we're pretty sure the right word is \textit{midpoint}.


\section*{Use \textsc{Wikipedia} to double check}

Now, on your favorite search engine, type "midpoint wiki" and read the corresponding \textsc{Wikipedia} page.\\
It \textit{midpoint} the correct word ?\\


\begin{remarque}[ : What's the tough job ?]
    The hardest thing is to formulate the right question to \textsc{ChatGpt}.\\
    That is where your english skills come into play !
\end{remarque}

\section*{Your turn !}
Find the translations of the following words or expressions using (if need be) \textsc{ChatGpt} and \textsc{Wikipedia}.

\begin{center}
    \tabstyle[UGLiRed]

    \begin{tabular}{c|c|}
        \ccell French                & \ccell English \\
        sommet                       & \hspace{10cm}\ \\
        angle aigu                   &                \\
        angle saillant               &                \\
        angle rentrant               &                \\
        médiatrice                   &                \\
        médiane (d'un triangle)      &                \\
        hauteur (d'un triangle)      &                \\
        bissectrice                  &                \\
        concourantes                 &                \\
        centre de gravité            &                \\
        orthocentre                  &                \\
        centre du cercle circonscrit &                \\
        centre du cercle inscrit     &                \\
        losange                      &                \\
        triangle équilatéral         &                \\
        triangle isocèle             &                \\
        triangle rectangle           &                \\
        triangle quelconque          &                \\
    \end{tabular}
\end{center}

You can add more words if you're curious... try "équation du second degré", "racines", "discriminant", "sommet de la parabole"...

\end{document}