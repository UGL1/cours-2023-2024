\documentclass[12pt,a4paper,english,firamath]{nsi}
\pagestyle{empty}
\begin{document}
\titre{Oral Presentation}
\classe{Lisea}
\maketitle
\textit{(For the)} \textbf{In}  exerci\textbf{s}e four, we must find the h\textbf{e}ight of the table. For this, we have two diagrams (two drawings) with cats, turtles and tables.\\

Firstly, I name the unknowns: the cat (B) because it is blue, the turtle (V) because it is green\footnote{Then why don't you name it G ?} and the table (T).\\

Next, I \textit{(do)} \textbf{formulate} the equation: on the one hand, B plus T minus V equals one hundred ans seventy centimeters. On the other side, V plus T minus B equals one hundred and thirty centimeters.\\ 
\textit{(Which)} \textbf{This}  is equivalent to \textit{(saying)} : B plus T minus V plus V plus T minus B.\\ 
Equivalent to saying: T plus T because we cross out all the common factors\footnote{All the opposite terms.}. \textbf{It} Is equal to one hundred and seventy plus one hundred and thirty equals three hundred.\\

Finally, as we need the size of \textit{(a)} one table, we divide by two. So three hundred divided by two equals one hundred ans fifty.\\

The size of the table is one hundred and fifty centimeters.\\

I chose this exercise because it's interesting and funny.
\end{document}