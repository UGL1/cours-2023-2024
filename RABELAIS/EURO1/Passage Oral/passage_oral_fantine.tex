\documentclass[12pt,a4paper,english,firamath]{nsi}
\pagestyle{empty}
\begin{document}
\titre{Oral Presentation}
\classe{Fantine}
\maketitle
In the statement, \textit{(it is said that)} \textbf{we can read} : "As defined by the International Astronomical Union (IAU), the light-year is the product of the Julian year and the speed of light."\\

Here we know that a light-year is equal to a Julian year, which is 365.25 days multiplied by the speed of light which is 299,792,458 meters per second.\\

First we have to put everything in the same unity\footnote{We have to make units match, so whe must calculate how many seconds is a year.}. So 365.25 days is equivalent to 365.25 multiplied by 24 (because there is 24 hours in a day ) then multiplied by 3600 ( there \textit{(is)} \textbf{are} 3600 seconds in an hour). It gives us 31,557,600 seconds.\\

Now to \textit{(respond)} \textbf{answer}  \textit{(to)} the question, which is : "How many kilometers is a light year" we have to multiply 31,557,600 by 2599,792,458. It gives us 9,460,730,472,580,800 meters per year. So now we know how many meters there is in a light-year. But we need the result in kilometers so we divid\textbf{e}  it by 1000 and we are left with 9,460,730,472,580. We can also round up this result to approximately 10 to the power of 12 or one trillion.
\end{document}