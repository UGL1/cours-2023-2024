\documentclass[12pt,a4paper,english,firamath]{nsi}
\pagestyle{empty}
\begin{document}
\titre{Oral Presentation}
\classe{Camille}
\maketitle
I chose the fifth exercise in which we have two problems, but I am just going to explain the first one.\\
\textit{(The first one)} \textbf{It} is is about an orange considered as a sphere with a diameter of 10 cm and a rope just long enough to make a full circle around \textit{(the orange)} it . We want to know how much we have to add to the rope's length to make it “float\textit{(s)}” 1cm \textit{(around)} \textbf{above} the orange.\\

First, to solve this problem, we have to know the circumference of the "old" rope or the orange.

For that, we know that the calculation for a circumference is 2 times $\pi$ multiplied by the radius. We know the diameter of the orange, w\textbf{b}ich is equal to 10 so its radius is equal to 5.
Then, we calculate the circumference of the orange by doing 2 times $\pi$ multiplied by five which gives us 10$\pi$.\\

Secondly, we have to find the circumference of the rope when it's 1 cm around the orange. \textit{(If the rope is 1 cm around the orange)} \textbf{If the distance between the rope and the orange is 1 cm}, that means that we \textbf{have to} add 1 cm to the radius of the rope. So, the radius of the rope around the orange is now equal to 6 cm.
Then we have to calculate the circumference of the rope by doing 2 times $\pi$ multiplied by 6 which gives us $12\pi$.\\

But the question wasn't what is the circumference of the new rope but how much we have to add to the old rope to make it float 1 cm above the orange.
To finally know that, we just have to subtract the circumference of the new rope to the circumference of the old rope\footnote{Actually, it's the opposite.}. So we do 12$\pi$ minus 10$\pi$ and it gives us $2\pi$.\\

\textit{(To conclude)} \textbf{In conclusion}, the answer of the problem is that we have to add 2 times $\pi$ to the length of the rope to make it float 1 cm above the orange.
\end{document}