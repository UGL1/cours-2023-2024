\documentclass[12pt,a4paper,article,english,firamath]{nsi}
\pagestyle{empty}
\begin{document}
\titre{Written test 01}
\classe{Euro 1\ere}
\maketitle

\begin{exercice}[]
    Three cousins, Zoé, Luc, and Serge, have a combined age of 60.\\
    
    What is the age of each, knowing that Luc is three times the age of Zoé, and Serge is ten years younger than Luc?
\end{exercice}

\begin{encadrecolore}{Solution}{UGLiGreen}
    Let $x$ be the age of Zoé. We know than Luc is $3x$ and Serge is $3x-10$.\\
    Adding their ages, whe get 60, which leads to the following equation
    $$x + 3x + 3x-10 = 60$$
    which simplifies to
    $$7x - 10 = 60$$
    So, adding ten on both sides, we get $$7x = 70$$
    and we conclude that $x = 10$ by dividing both sides by 10.\\

    Thus, Zoé's 10, Luc is 30 and Serge is 20.
\end{encadrecolore}

\begin{exercice}[]
    The wheels of my bike have the same diameter. When they make a complete turn, I move forward by approximately 230 cm.\\

    What's their diameter (you can provide an approximate value) ?
\end{exercice}

\begin{encadrecolore}{Solution}{UGLiGreen}
    Let $d$ be the wheels' diameter. We know their perimeter is 230 cm, so we have
    $$\pi \times d = 230$$
    which leads to $d = \dfrac{230}{\pi}\simeq 73.2$ rounded to the nearest millimeter, and that's the diameter of my wheels. 
\end{encadrecolore}

\begin{remarque}[]
    This is the diameter of a 29'' (29 inches) wheel on a mountain bike. 
\end{remarque}


\begin{exercice}[]
    Pierre is half the age his father had 12 years ago.\\
    He was born when his father was celebrating his 25$^{\text{th}}$ birthday.\\
    
    What is Pierre's age?
\end{exercice}
\begin{encadrecolore}{Solution}{UGLiGreen}
    Let $x$ be Pierre's age and $y$ his father's.\\
    From the first sentence we derive a first equation :
    $$x = \dfrac{y-12}{2}$$
    And the second sentences implies that Pierre's father's 25 years older than him, hence
    $$y = x + 25$$
    So we have to solve the following system
    $$\begin{cases}
         x & = \dfrac{y-12}{2} \\
         y &= x + 25
    \end{cases}$$
    which is really easy since we can substitute the value of $y$ into the first equation, getting
    $$x = \dfrac{x+25-12}{2}$$
    which is equivalent, by multiplying both sides by 2, to
    $$2x = x + 13$$
    which gives us $x=13$ by subtracting $x$ to both sides.\\

    Thus Pierre's 13 and his father's $25+13=37$.
\end{encadrecolore}
\end{document}