\documentclass[12pt,a4paper,article]{nsi}
\pagestyle{empty}
\begin{document}
\titre{Benford's Law}
\classe{Euro 1\ere}
\maketitle
\picright{0.3}{img/Frank_Benford}{
    Benford's law, also called the first-digit law, is a phenomenological law about the
    frequency distribution of leading digits in many (but not all) real-life sets of numerical data.

    The law states that in many naturally occurring collections of numbers the small digits occur
    disproportionately often as leading significant digits. For example, in sets which obey the law the number 1 would appear as the most significant digit about 30\% of the time, while larger digits would occur in that position less frequently : 9 would appear less than 5\% of the time.

    It has been shown that this result applies to a wide variety of data sets, including electricity bills, street addresses, population numbers\ldots}


\begin{enumerate}
    \item If all digits were distributed uniformly in a collection of numbers, what percentage would
          correspond to each number ?
    \item Give other examples where Benford’s law could be observed.
    \item The population of European Union countries in 2015 is given in the table below.
          \tabstyle[UGLiBlue]
          \begin{center}
              \begin{tabular}{l|l||||||||l|l}
                  Germany  & 81 829 901 & Italy          & 60 808 668 \\
                  Austria  & 8 507 786  & Latvia         & 2 028 200  \\
                  Belgium  & 11 209 044 & Lithuania      & 2 941 953  \\
                  Bulgaria & 7 351 234  & Luxembourg     & 562 958    \\
                  Cyprus   & 1 172 45   & Malta          & 433 504    \\
                  Croatia  & 4 242 067  & Netherlands    & 16 791 405 \\
                  Danmark  & 5 566 856  & Poland         & 38 509 789 \\
                  Spain    & 46 439 864 & Portugal       & 10 427 301 \\
                  Estonia  & 1 315 819  & Czech Republic & 10 537 800 \\
                  Finland  & 5 450 614  & Romania        & 19 779 963 \\
                  France   & 66 627 602 & United Kingdom & 64 596 752 \\
                  Greece   & 10 775 557 & Slovakia       & 5 404 322  \\
                  Hungary  & 9 908 798  & Slovenia       & 2 059 313  \\
                  Ireland  & 4 609 600  & Sweden         & 9 639 741
              \end{tabular}
          \end{center}
          Complete the table below with the number of countries whose leading digit is on line 1.

          Work out
          the frequencies of these numbers.
          \begin{center}
              \begin{tabular}{l|c|c|c|c|c|c|c|c|c}
                  Leading digit       & 1 & 2 & 3 & 4 & 5 & 6 & 7 & 8 & 9 \\
                  Number of countries &   &   &   &   &   &   &   &   &   \\
                  Relative frequency  &   &   &   &   &   &   &   &   &   \\
              \end{tabular}
          \end{center}
    \item The following graph represents the distribution of first digits, according to Benford's law. Each
          bar represents a digit, and the height of the bar is the percentage of numbers that starts with that
          digit.
          \begin{center}
              \begin{tikzpicture}[yscale=.2]
                  \draw[ystep=5,dotted,UGLiBlue] (0,0)grid(10,35);
                  \foreach \x in {1,...,9}
                      {\node at (\x ,-5em) {$\x$};
                          \draw[UGLiBlue,fill=UGLiBlue!10] (\x-.5,0) rectangle (\x+.5,{(log10(\x+1)-log10(\x))*100});}

                  \foreach \x in {0,5,10,...,35}
                      {\node at (-1em,\x) {$\x$};}
                  \node at (5,-15em){digit};
                  \node[rotate=90] at (-3em,17.5){probability in \%};
              \end{tikzpicture}
          \end{center}
    \item Complete the following graph with the data you gathered in the table above. Does the population of European Union countries in 2015 follow Benford's Law ?
          \begin{center}
              \begin{tikzpicture}[yscale=.2]
                  \draw[ystep=5,dotted,UGLiBlue] (0,0)grid(10,35);
                  \foreach \x in {1,...,9}
                      {\node at (\x ,-5em) {$\x$};
                      }
                  \foreach \x in {0,5,10,...,35}
                      {\node at (-1em,\x) {$\x$};}
                  \node at (5,-15em){digit};
                  \node[rotate=90] at (-3em,17.5){probability in \%};
              \end{tikzpicture}
          \end{center}
\end{enumerate}
\end{document}