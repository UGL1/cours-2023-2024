\documentclass[12pt,a4paper,article,english,firamath]{nsi}
\pagestyle{empty}
\begin{document}
\titre{Secret Santa}
\classe{Euro 1\ere}
\maketitle

From 00:00 to 00:42 What does Secret Santa consists in ?
\begin{encadrecolore}{Answer}{UGLiRed}
Secret Santa is a method which makes each person in a group choose another one to give a present to, so that everybody buys a present for someone and also gets one.
\end{encadrecolore}
From 01:22 to 01:36 What are the two fundamental things for a perfect Secret Santa ?
\begin{encadrecolore}{Answer}{UGLiRed}
    \begin{enumerate}
        \item It should be anonymous : nobody should know who bought his or her present.
        \item It should be perfectly random : you should have the same probability of getting a present from each member of the group.
    \end{enumerate}
\end{encadrecolore}
From from 01:44 to 02:03 How does Hannah describe the "hat method" ?
\begin{encadrecolore}{Answer}{UGLiRed}
    Everybody writes his or her name on a piece of paper and puts it in a hat.
    Then each member of the group successively picks a paper and reads a name. If someone picks his or her own name, he or she puts the paper back in the hat, otherwise, he or she will have to offer a present to the person whose name's written on the paper. 
\end{encadrecolore}
What is the obvious problem with this method and what does the group have to do in this case ?
\begin{encadrecolore}{Answer}{UGLiRed}
    If the last person's left with his or her own paper, there's nothing to do but start over. 
\end{encadrecolore}
When the method works, is it completely random ?
\end{document}