\documentclass[12pt,a4paper,eval,english,firamath]{nsi}
\pagestyle{empty}
\begin{document}
\titre{About 7}
\classe{Euro 1\ere}
\maketitle

\ \\[-5em]
\section*{Method}
Watch the video "Method Without Subtitles" once.\\
You can watch it a second time and if you feel lost, ask the teacher for the subtitled version.\\

Did you need the subtitled version ?$\qquad\Box$ YES$\quad \Box$ NO\\

What's the video about ?\\

\carreauxseyes{16.8}{2.4}\\

Give another example which shows how the method described in the video works.\\

\carreauxseyes{16.8}{6.4}\\
\section*{Proof}
Watch the video "Proof Without Subtitles" once.\\
You can also watch this one a second time and/or ask the teacher for the subtitled version.\\

Did you need the subtitled version ?$\qquad\Box$ YES$\quad \Box$ NO\\

Explain why this method is correct.\\

\begin{encadrecolore}{Beware}{UGLiRed}
    In the video, the man forgot an important detail : he actually proved that 5 times the original number is divisible by 7 if and only if $x+5y$ is divisible by 7\ldots but that's it !
    
    We will admit that if  $5n$ is a multiple of 7,  then $n$ is also a multiple of 7.
\end{encadrecolore}

\carreauxseyes{16.8}{20}

\section*{One step further}
There is another method we can use to work out wether an integer is divisible by 7, here it is showed on an example : is 12345 divisible by 7 ?\\

Let's take 12345, split it into 1234 and 5 and calculate $1234 - 2\times 5$.\\
This gives us 1224, let's repeat the process : \\
$122 - 2\times 4 = 114$\\
$11 - 2\times 8 = -5$ so, as -5 is not divisible by 7, 12345 isn't either.\\

Prove that this method is valid.\\

\carreauxseyes{16.8}{17.6}

\newpage
{\LARGE\color{UGLiBlue}\titlefont APPENDIX : theorems used in the video (or not)}\\



\begin{encadrecolore}{Theorem 1}{UGLiRed}
    Let $a$, $b$ and $c$ be three integers, with $b$ and $c$ not equal to zero.\\

    If $a$ is divisible by $b$ then $ca$ is divisible by $b$.
\end{encadrecolore}

% \begin{demonstration}
%     \begin{tabbing}
%         $b|a$ 	\= $\Longrightarrow\exists k\in\N^*$, $a=kb$ \\
%         \> $\Longrightarrow\exists k\in\N^*$, $ca=ckb$\\
%         \> $\Longrightarrow\exists k'\in\N^*$, $ca=k'b$,  $k'$ being $ck$\\
%         \> $\Longrightarrow b|ca$
%     \end{tabbing}
% \end{demonstration}

\begin{exemple}[]
    4 is divisible by 2 so $4\times 123$ is also divisible by 2.
\end{exemple}

\begin{encadrecolore}{Theorem 2}{UGLiRed}
    Let $a$, $b$ and $c$ be three integers such that $a = b + c$\\
    Suppose that $c$ is divisible by $k$, then either $a$ and $b$ are also divisible by $k$, or none of them are.
\end{encadrecolore}

\begin{exemple}[]
    $x$ is an integer.
    \begin{enumerate}
        \item Suppose $30 = x + 5$, then $x$ is also divisible by 5.
        \item Suppose $30 = x + 11$, then $x$ is not divisible by 11.
    \end{enumerate}
\end{exemple}


\begin{encadrecolore}{Theorem 3 (Gauss, to be used with $a=7$ and $b=5$)}{UGLiRed}
    Let $a$, $b$ and $c$ be three integers such that $a$ divides $bc$ and $a$ and $b$ have no common factor (except 1).\\

    Then $a$ divides $c$.
\end{encadrecolore}

\begin{exemple}[]
    If 7 divides $5n$, since 7 and 5 have no common factors other than 1, we can conclude that 7 divides $n$.
\end{exemple}

\end{document}

