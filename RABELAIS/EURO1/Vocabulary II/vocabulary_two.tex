\documentclass[12pt,a4paper,article,english,firamath]{nsi}
\pagestyle{empty}
\begin{document}
\titre{Vocabulary II \small (right one)}
\classe{Euro 1\ere}
\maketitle
\begin{flushright}
    \color{UGLiBlue}\textit{"Still Boring, still useful"}
\end{flushright}

\section*{Rounding a decimal}

\picright{0.3}{img/decimal_places}{
    Let's choose $A=3.165$ as an example.
    \begin{itemize}
        \item When we round $A$ to the nearest integer, we get 3.
        \item $A$ rounded to the nearest tenth is 3.2.
        \item We can round $A$ up to 3.17.
        \item $A$ can be also rounded down to 3.1.
    \end{itemize}}

\section*{Order of magnitude}

When I'm looking on the internet  right now, I can read that the estimated current world population is 8,059,292,083 people (it's increasing every second).\\
The order of magnitude of this number is the nearest power of ten, so it's ten to the power of ten, which I can write $10^{10}$, or ten billions.

\section*{Big numbers}

\tabstyle[UGLiRed]
\begin{center}
    \begin{tabular}{l|l|l|l}
        \ccell number & \ccell short scale naming & \ccell prefix & \ccell symbol \\
        $10^6$        & one million               & mega-         & M             \\
        $10^{9}$      & one billion               & giga-         & G             \\
        $10^{12}$     & one trillion              & tera-         & T             \\
        $10^{15}$     & one quadrillion           & peta-         & P             \\
        $10^{18}$     & one quintillion           & exa-          & E             \\
        $10^{21}$     & one sextillion            & zetta-        & Z             \\
        $10^{24}$     & one septillion            & yotta-        & Y             \\
    \end{tabular}
\end{center}

\end{document}