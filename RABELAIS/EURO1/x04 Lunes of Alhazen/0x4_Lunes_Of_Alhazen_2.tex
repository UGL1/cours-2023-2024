\documentclass[12pt,a4paper,article,english,firamath]{nsi}
\geometry{bottom=1cm}
\pagestyle{empty}
\begin{document}
\titre{The Lunes of Alhazen}
\classe{Euro 1\ere}
\maketitle


\begin{center}
    \begin{tikzpicture}[rotate=-143.13]
        % Draw triangle
        \coordinate (A) at (0,0);
        \coordinate (B) at (4,0);
        \coordinate (C) at (0,3);
        % Draw half circles on each side
        \draw[fill=UGLiBlue!25] (A)  arc (180:360:4/2) -- (B) -- cycle;
        \draw[fill=UGLiBlue!25] (C)  arc (90:270:3/2) -- (A) -- cycle;
        \draw[fill=white] (C) arc[start angle = 180-36.87, delta angle=180,radius=2.5] -- (B)--cycle;
        \draw[fill=UGLiRed!25](A)--(B)--(C)--cycle;
        
    \end{tikzpicture}
\end{center}
The red triangle is rectangle.\\
Can you show its area is the same as the sum of the areas of the two blue lunes ?\\

To help you, here are the figures corresponding to the main steps of the demonstration. Can you complete it ?\\


\subsection*{Step 1}
\begin{center}
    \begin{tikzpicture}[rotate=-143.13]
        % Draw triangle
        \coordinate (A) at (0,0);
        \coordinate (B) at (4,0);
        \coordinate (C) at (0,3);
        % Draw half circles on each side
        \draw[fill=UGLiBlue!25] (A)  arc (180:360:4/2) -- (B) -- cycle;
        \draw[fill=UGLiBlue!25] (C)  arc (90:270:3/2) -- (A) -- cycle;
        \draw[fill=UGLiGreen!25] (B) arc[start angle = -36.87, delta angle=180,radius=2.5] -- (C)--cycle;
        \draw[fill=UGLiRed!25](A)--(B)--(C)--cycle;
    \end{tikzpicture}
\end{center}
\carreauxseyes{16.8}{8}
\subsection*{Step 2}
\begin{center}
    \begin{tikzpicture}[rotate=-143.13]
        % Draw triangle
        \coordinate (A) at (0,0);
        \coordinate (B) at (4,0);
        \coordinate (C) at (0,3);
        % Draw half circles on each side
        \draw[fill=UGLiBlue!25] (A)  arc (180:360:4/2) -- (B) -- cycle;
        \draw[fill=UGLiBlue!25] (C)  arc (90:270:3/2) -- (A) -- cycle;
        \draw[fill=UGLiGreen!25] (B) arc[start angle = -36.87, delta angle=180,radius=2.5] -- (C)--cycle;
        \draw[fill=UGLiRed!25](A)--(B)--(C)--cycle;
        \draw[semitransparent,fill=UGLiGreen!25,dashed] (C) arc[start angle = 180-36.87, delta angle=180,radius=2.5] -- (B)--cycle;           
    \end{tikzpicture}
\end{center}
\carreauxseyes{16.8}{4.8}
    
\subsection*{Step 3}
\begin{center}
    \begin{tikzpicture}[rotate=-143.13]
        % Draw triangle
        \coordinate (A) at (0,0);
        \coordinate (B) at (4,0);
        \coordinate (C) at (0,3);
        % Draw half circles on each side
        \draw[fill=UGLiBlue!25] (A)  arc (180:360:4/2) -- (B) -- cycle;
        \draw[fill=UGLiBlue!25] (C)  arc (90:270:3/2) -- (A) -- cycle;
        \draw[fill=UGLiRed!25](A)--(B)--(C)--cycle;
        \draw[semitransparent,fill=UGLiGreen!25] (C) arc[start angle = 180-36.87, delta angle=180,radius=2.5] -- (B)--cycle;
        \node at (4/3,1) {a};
        \node at (2,-.5) {b};
        \node at (2,-1.45) {c};
        \node at (-.25,1.5) {d};
        \node at (-.95,1.45) {e};
    \end{tikzpicture}
\end{center}
\carreauxseyes{16.8}{8}
\end{document}