\documentclass[12pt,a4paper,article,english,firamath]{nsi}
\pagestyle{empty}
\begin{document}
\titre{The Lunes of Alhazen}
\classe{Euro 1\ere}
\maketitle


\begin{center}
    \begin{tikzpicture}[scale=2,rotate=-143.13]
        % Draw triangle
        \coordinate (A) at (0,0);
        \coordinate (B) at (4,0);
        \coordinate (C) at (0,3);
        % Draw half circles on each side
        \draw[fill=UGLiBlue!25] (A)  arc (180:360:4/2) -- (B) -- cycle;
        \draw[fill=UGLiBlue!25] (C)  arc (90:270:3/2) -- (A) -- cycle;
        \draw[fill=white] (C) arc[start angle = 180-36.87, delta angle=180,radius=2.5] -- (B)--cycle;
        \draw[fill=UGLiRed!25](A)--(B)--(C)--cycle;
        
    \end{tikzpicture}
\end{center}
The red triangle is rectangle.\\
Can you show its area is the same as the sum of the areas of the two blue lunes ?
\end{document}