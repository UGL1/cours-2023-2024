\documentclass[12pt,a4paper,article,english,firamath]{nsi}
\pagestyle{empty}
\begin{document}
\titre{The Lunes of Alhaze (solution)}
\classe{Euro 1\ere}
\maketitle

\dleft{6.5cm}
{
    \begin{tikzpicture}[rotate=-143.13]
        % Draw triangle
        \coordinate (A) at (0,0);
        \coordinate (B) at (4,0);
        \coordinate (C) at (0,3);
        % Draw half circles on each side
        \draw[fill=UGLiBlue!25] (A)  arc (180:360:4/2) -- (B) -- cycle;
        \draw[fill=UGLiBlue!25] (C)  arc (90:270:3/2) -- (A) -- cycle;
        \draw[fill=UGLiGreen!25] (B) arc[start angle = -36.87, delta angle=180,radius=2.5] -- (C)--cycle;
        \draw[fill=UGLiRed!25](A)--(B)--(C)--cycle;
    \end{tikzpicture}
}
{
    Since the triangle is rectangle, the Pythagorean theorem applies : the square of the length the hypotenuse is equal to the sum of the squares of the lengths of the two other sides.\\

    But as a half circle's area is proportional to its diameter, this means that the blue area and the green one are equal.
}
\dright{6.5 cm}
{
    Now consider the symmetrical of the green half disk with respect to the hypotenuse. Its area is the same so the equation still holds.
}
{
    \begin{tikzpicture}[rotate=-143.13]
        % Draw triangle
        \coordinate (A) at (0,0);
        \coordinate (B) at (4,0);
        \coordinate (C) at (0,3);
        % Draw half circles on each side
        \draw[fill=UGLiBlue!25] (A)  arc (180:360:4/2) -- (B) -- cycle;
        \draw[fill=UGLiBlue!25] (C)  arc (90:270:3/2) -- (A) -- cycle;
        \draw[fill=UGLiRed!25](A)--(B)--(C)--cycle;
        \draw[semitransparent,fill=UGLiGreen!25] (C) arc[start angle = 180-36.87, delta angle=180,radius=2.5] -- (B)--cycle;
    \end{tikzpicture}
}
\\

\dleft{6cm}
{
    \begin{tikzpicture}[rotate=-143.13]
        % Draw triangle
        \coordinate (A) at (0,0);
        \coordinate (B) at (4,0);
        \coordinate (C) at (0,3);
        % Draw half circles on each side
        \draw[fill=UGLiBlue!25] (A)  arc (180:360:4/2) -- (B) -- cycle;
        \draw[fill=UGLiBlue!25] (C)  arc (90:270:3/2) -- (A) -- cycle;
        \draw[fill=UGLiRed!25](A)--(B)--(C)--cycle;
        \draw[semitransparent,fill=UGLiGreen!25] (C) arc[start angle = 180-36.87, delta angle=180,radius=2.5] -- (B)--cycle;
        \node at (4/3,1) {a};
        \node at (2,-.5) {b};
        \node at (2,-1.45) {c};
        \node at (-.25,1.5) {d};
        \node at (-.95,1.45) {e};
    \end{tikzpicture}
}
{
    Naming all the regions, we can derive the following equation :
    $$b+c+e+d = a+b+d$$
    Hence, subtracting $b+d$ on both sides, we get 
    $$c + e = a$$
}\\

which proves that the blue lunes have the same area as the red triangle.
\end{document}