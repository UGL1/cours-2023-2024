\documentclass[12pt,a4paper,article,english,firamath]{nsi}
\pagestyle{empty}
\begin{document}
\titre{\LARGE Problems involving quadratic equations}
\classe{Euro 1\ere}
\maketitle


\section*{The mystery number}
There is a two-digit number whose digits are the same, and has got the following property: When squared, it produces a four-digit number, whose first two digits are the same and equal to the original's minus one, and whose last two digits are the same and equal to the half of the original's.\\
Find that number.

\begin{remarque}[]
    Of course, you can easily find this number by testing all the possibilities, but it is more interesting to find it out of calculations.
\end{remarque}


\section*{A very complicated way of writing an integer}

Assume $$\sqrt{6+\sqrt{6+\sqrt{6+\ldots}}}$$ exists and is an integer.\\
Find its value.



\section*{How many benches ?}
In a concert hall, 800 people are seated on benches of equal length (the same number or persons are seated on each bench). If there had been 20 fewer benches, it would have been necessary to add two more people per bench.\\
How many benches are there ?


\section*{A family affair}
A father and his son work for the same employer. After a certain number of hours of work, the father receives 500 €. The son, who worked 5 hours less and has an hourly wage that is 8 € lower than that of his father, only receives 240 €. \\

What's the hourly wage of each and how many hours have they worked (the hours are integers) ?

\section*{Tangency of an hyperplan and an hypersphere \small(just as a challenge)}

Let $n\in\N^*$ and $x_1$, \ldots, $x_n$ be $n$ numbers such that $x_1+\ldots +x_n = n$ and $x_1^2+\ldots +x_n^2 = n$.\\

Prove that $x_1=\ldots=x_n=1$.


\end{document}