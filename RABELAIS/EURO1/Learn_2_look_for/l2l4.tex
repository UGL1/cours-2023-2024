\documentclass[12pt,a4paper,article,english,firamath]{nsi}
\pagestyle{empty}
\begin{document}
\titre{A way to look for words online}
\classe{Euro 1\ere}
\maketitle

Let's assume we have to look for the translation of a mathematical term.\\ Where do we begin ?\\ How can we be (fairly) sure we have the right translation ?

For instance : how to we say "le milieu (d'un segment)" in English ?

\section*{Don't be afraid of \textsc{ChatGpt}}

Go to \texttt{https://chat.openai.com/} and sign up : use one of your email accounts and chose a password.\\
Now ask your question in english. Don't worry, even if you are a bit clumsy, \textsc{ChatGpt} will do its best to understand. For instance you could ask:\\ "How do you call the point which is right in the middle of a segment ?".\\
Here's what \textsc{ChatGpt} answers :\\

\texttt{\noindent
    The point that is right in the middle of a segment is called the\\
    "midpoint." The midpoint is equidistant from both endpoints of \\
    the segment and divides the segment into two equal halves. It is
    a \\
    fundamental concept in geometry and is often denoted by the
    letter \\
    "M" or by other notations such as "P" or "Q" depending on
    the\\
    context.}\\

So we're pretty sure the right word is \textit{midpoint}.


\section*{Use \textsc{Wikipedia} to double check}

Now, on your favorite search engine, type "midpoint wiki" and read the corresponding \textsc{Wikipedia} page.\\
It \textit{midpoint} the correct word ?\\


\begin{remarque}[ : What's the tough job ?]
    The hardest thing is to formulate the right question to \textsc{ChatGpt}.\\
    That is where your english skills come into play !
\end{remarque}

\section*{Your turn}
Find the translations of the following words or expressions using (if need be \textsc{ChatGpt} and \textsc{Wikipedia}).

\begin{center}
    \tabstyle[UGLiRed]

    \begin{tabular}{c|c|}
        \ccell French                & \ccell English \\
        sommet                       & \hspace{10cm}\ \\
        angle aigu                   &                \\
        angle saillant               &                \\
        angle rentrant               &                \\
        médiatrice                   &                \\
        médiane (d'un triangle)      &                \\
        hauteur (d'un triangle)      &                \\
        bissectrice                  &                \\
        concourantes                 &                \\
        centre de gravité            &                \\
        orthocentre                  &                \\
        centre du cercle circonscrit &                \\
        centre du cercle inscrit     &                \\
        losange                      &                \\
        triangle équilatéral         &                \\
        triangle isocèle             &                \\
        triangle rectangle           &                \\
        triangle quelconque          &                \\
    \end{tabular}
\end{center}

You can add more words if you're curious... try "équation du second degré", "racines", "discriminant", "sommet de la parabole"...
\end{document}