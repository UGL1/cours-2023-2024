\documentclass[a4paper,12pt,article,firamath]{nsi}
\begin{document}
\resetquestion
\textit{Cet exercice porte sur les arbres binaires de recherche, la programmation orientée
    objet et la récursivité.}\\

Dans cet exercice, la taille d'un arbre est le nombre de nœuds qu'il contient. Sa hauteur
est le nombre de nœuds du plus long chemin qui joint le nœud racine à l'une des
feuilles (nœuds sans sous-arbres). On convient que la hauteur d'un arbre ne contenant
qu'un nœud vaut 1 et la hauteur de l'arbre vide vaut 0.\\

On considère l'arbre binaire représenté ci-dessous:
\begin{center}
    \includegraphics[width=7cm]{img/fig1.png}\\
    Figure 1
\end{center}
Donner la taille et la hauteur de cet arbre.\\
\begin{encadrecolore}{Réponse}{gray}
    La taille est 8 et la hauteur 4.
    \end{encadrecolore}
\dleft{7cm}
{
    \includegraphics[width=3cm]{img/fig3.png}
}{\question Représenter ci-contre le sous-arbre droit du n\oe ud de valeur 15.\\

    \question Justifier que l'arbre de la figure 1 est un arbre binaire de recherche.
}

\begin{encadrecolore}{Réponse}{gray}
    C'est un arbre binaire de recherche car pour chaque n\oe ud $n$, le sous-arbre gauche (respectivement droit) est composé de n\oe uds de valeurs inférieures (respectivement supérieures) ou égales à celle de $n$.
\end{encadrecolore}


\dleft{7cm}
{
    \begin{center}
        \includegraphics[width=7cm]{img/fig4.png}\\
    \end{center}
}{
    On insère la valeur 17 dans l'arbre de la figure 1 de telle sorte que 17 soit une
    nouvelle feuille de l'arbre et que le nouvel arbre obtenu soit encore un arbre
    binaire de recherche.\\

    \question Représenter ci-contre ce nouvel arbre.
}

On considère la classe \mintinline{python}{Noeud} définie de la façon suivante en Python :


\begin{minted}{python}
        class Noeud:
            def __init__(self, g, v, d):
                self.gauche = g
                self.valeur = v
                self.droit = d
\end{minted}

\picleft{0.17}{img/fig2}{\question Parmi les trois instructions suivantes, entourer celle qui construit et stocke dans la variable abr l'arbre représenté ci-contre.}


\mintinline{python}{abr = Noeud(Noeud(Noeud(None, 13, None), 15, None), 21, None)}\\
\mintinline{python}{abr = Noeud(None, 13, Noeud(Noeud(None, 15, None), 21, None))}\\
\fbox{\mintinline{python}{abr = Noeud(Noeud(None, 13, None), 15, Noeud(None, 21, None))}}\\


\question Compléter le code de la fonction \mintinline{python}{ins} ci-dessous qui
prend en paramètres une valeur \mintinline{python}{v} et un arbre binaire de recherche \mintinline{python}{abr} et qui
renvoie l'arbre obtenu suite à l'insertion de la valeur \mintinline{python}{v} dans l'arbre \mintinline{python}{abr}.

Les lignes 8 et 9 permettent de ne pas insérer la valeur \mintinline{python}{v} si celle-ci est déjà présente dans \mintinline{python}{abr}.


\begin{minted}[linenos=true]{python}
  def ins(v, abr):
    if abr is None:
        return Noeud(None, v, None)
    if v > abr.valeur:
        return Noeud(abr.gauche, abr.valeur, ins(v, abr.droit))
    elif v < abr.valeur:
        return Noeud(ins(v,abr.gauche), abr.valeur, abr.droit)
    else:
        return abr      
\end{minted}

La fonction \mintinline{python}{nb_sup} prend en paramètres une valeur \mintinline{python}{v} et un arbre binaire de recherche \mintinline{python}{abr} et renvoie le nombre de valeurs supérieures ou égales à la valeur \mintinline{python}{v} dans l'arbre \mintinline{python}{abr}.\\

Le code de cette fonction \mintinline{python}{nb_sup} est donné ci-dessous :

\begin{minted}{python}
def nb_sup(v, abr):
    if abr is None:
        return 0
    else:
        if abr.valeur >= v:
            return 1 + nb_sup(v, abr.gauche) + nb_sup(v, abr.droit)
        else:
            return nb_sup(v, abr.gauche) + nb_sup(v, abr.droit)
\end{minted}

On exécute l'instruction \mintinline{python}{nb_sup(16, abr)} dans laquelle \mintinline{python}{abr} est l'arbre initial de la figure 1.\\

\question Déterminer le nombre d'appels à la fonction \mintinline{python}{nb_sup} lors de cette exécution.\\

\carreauxseyes{16.8}{6.4}\\

L'arbre passé en paramètre étant un arbre binaire de recherche, on peut
améliorer la fonction \mintinline{python}{nb_sup} précédente afin de réduire ce nombre d'appels.\\

\question Écrire sur la copie le code modifié de cette fonction. 

\carreauxseyes{16.8}{8}
\end{document}