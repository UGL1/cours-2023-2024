\documentclass[a4paper,12pt,article,firamath]{nsi}
\begin{document}
\resetquestion
\textit{Cet exercice porte sur les arbres binaires de recherche, la programmation orientée
    objet et la récursivité.}\\

Dans cet exercice, la taille d'un arbre est le nombre de nœuds qu'il contient. Sa hauteur
est le nombre de nœuds du plus long chemin qui joint le nœud racine à l'une des
feuilles (nœuds sans sous-arbres). On convient que la hauteur d'un arbre ne contenant
qu'un nœud vaut 1 et la hauteur de l'arbre vide vaut 0.\\

On considère l'arbre binaire représenté ci-dessous:
\begin{center}
    \includegraphics[width=7cm]{img/fig1.png}\\
    Figure 1
\end{center}
Donner la taille et la hauteur de cet arbre.\\

\carreauxseyes{16.8}{2.4}\\

\question Représenter ci-dessous le sous-arbre droit du n\oe ud de valeur 15.\\

\begin{tikzpicture}
    \draw (0,0) rectangle (\linewidth,5);
\end{tikzpicture}

\question Justifier que l'arbre de la figure 1 est un arbre binaire de recherche.\\

\carreauxseyes{16.8}{3.2}\\


On insère la valeur 17 dans l'arbre de la figure 1 de telle sorte que 17 soit une
nouvelle feuille de l'arbre et que le nouvel arbre obtenu soit encore un arbre
binaire de recherche.\\

\question Représenter ci-dessous ce nouvel arbre.\\

\begin{tikzpicture}
    \draw (0,0) rectangle (\linewidth,7);
\end{tikzpicture}
On considère la classe \mintinline{python}{Noeud} définie de la façon suivante en Python :


\begin{minted}{python}
        class Noeud:
            def __init__(self, g, v, d):
                self.gauche = g
                self.valeur = v
                self.droit = d
\end{minted}

\question Parmi les trois instructions suivantes, entourer celle qui construit et stocke dans la variable abr l'arbre représenté ci-dessous.
\begin{center}
    \includegraphics[width=6cm]{img/fig1}
\end{center}


\begin{minted}{python}
abr = Noeud(Noeud(Noeud(None, 13, None), 15, None), 21, None)
abr = Noeud(None, 13, Noeud(Noeud(None, 15, None), 21, None))
abr = Noeud(Noeud(None, 13, None), 15, Noeud(None, 21, None)) 
\end{minted}

La fonction \mintinline{python}{ins} ci-dessous qui
prend en paramètres une valeur \mintinline{python}{v} et un arbre binaire de recherche \mintinline{python}{abr} et qui
renvoie l'arbre obtenu suite à l'insertion de la valeur \mintinline{python}{v} dans l'arbre \mintinline{python}{abr}.

Les lignes 8 et 9 permettent de ne pas insérer la valeur \mintinline{python}{v} si celle-ci est déjà présente dans \mintinline{python}{abr}.


    \begin{minted}[linenos=true]{python}
        def ins(v, abr):
          if abr is None:
              return Noeud(None, v, None)
          if v > abr.valeur:
              return Noeud(abr.gauche, abr.valeur, ins(v, abr.droit))
          elif v < abr.valeur:
              return ............................................
          else:
          return abr      
      \end{minted}          

\question Compléter le code de la fonction \mintinline{python}{ins}.\\



La fonction \mintinline{python}{nb_sup} ci dessous prend en paramètres une valeur \mintinline{python}{v} et un arbre binaire de recherche \mintinline{python}{abr} et renvoie le nombre de valeurs supérieures ou égales à la valeur \mintinline{python}{v} dans l'arbre \mintinline{python}{abr}.\\
Le code de cette fonction \mintinline{python}{nb_sup} est donné ci-dessous :

\begin{minted}{python}
def nb_sup(v, abr):
    if abr is None:
        return 0
    else:
        if abr.valeur >= v:
            return 1 + nb_sup(v, abr.gauche) + nb_sup(v, abr.droit)
        else:
            return nb_sup(v, abr.gauche) + nb_sup(v, abr.droit)
\end{minted}

On exécute l'instruction \mintinline{python}{nb_sup(16, abr)} dans laquelle \mintinline{python}{abr} est l'arbre initial de la figure 1.\\

\question Déterminer le nombre d'appels à la fonction \mintinline{python}{nb_sup} lors de cette exécution.\\

\carreauxseyes{16.8}{6.4}\\

L'arbre passé en paramètre étant un arbre binaire de recherche, on peut
améliorer la fonction \mintinline{python}{nb_sup} précédente afin de réduire ce nombre d'appels.\\

\question Écrire sur la copie le code modifié de cette fonction.\\

\carreauxseyes{16.8}{8}
\end{document}