\documentclass[10pt,firamath,cours]{nsi}



\begin{document}
\setcounter{chapter}{4}
\chapter{BDD partie 4}
\section{Requêtes SQL}
\subsection{Requête et résultat}
Une requête est une commande SQL et renvoie une table.\\
On se replace dans le contexte du chapitre précédent.

\subsection{Sélection d'attributs}
\begin{minted}{sql}
SELECT nom, prenom
FROM Auteur;
    \end{minted}

\begin{center}
    \tabstyle[UGLiOrange]
    \begin{tabular}{c|c}
        \ccell nom  & \ccell prenom \\
        Ammaniti    & Niccolo       \\
        Avallone    & Silvia        \\
        Camus       & Albert        \\
        Hamilton    & Peter         \\
        Hugo        & Victor        \\
        Murgia      & Michela       \\
        Rhode james & Montague      \\
        Tolkien     & John
    \end{tabular}
\end{center}


\subsection{Sélection de tous les attributs}
\begin{minted}{sql}
SELECT *
FROM Auteur;
    \end{minted}

\begin{center}
    \tabstyle[UGLiOrange]
    \begin{tabular}{c|c|c|c|c}
        \ccell id\_auteur & \ccell nom\_pays & \ccell nom  & \ccell prenom & \ccell date\ naissance \\
        1                 & France           & Hugo        & Victor        & 1802-02-26             \\
        2                 & France           & Camus       & Albert        & 1913-11-07             \\
        4                 & Italie           & Avallone    & Silvia        & 1948-04-13             \\
        5                 & Italie           & Ammaniti    & Niccolo       & 1966-09-25             \\
        6                 & Italie           & Murgia      & Michela       & 1972-06-03             \\
        7                 & Royaume-Uni      & Hamilton    & Peter         & 1960-03-02             \\
        8                 & Royaume-Uni      & Tolkien     & John          & 1892-01-03             \\
        9                 & Royaume-Uni      & Rhode James & Montague      & 1862-08-01
    \end{tabular}
\end{center}


\subsection{Sélection avec condition}
\begin{minted}{sql}
SELECT nom, date_naissance
FROM Auteur
WHERE date(date_naissance) < '1900';
    \end{minted}

\begin{center}
    \tabstyle[UGLiOrange]
    \begin{tabular}{c|c}
        \ccell nom  & \ccell date\_naissance \\
        Hugo        & 1802-02-16             \\
        Tolkien     & 1892-01-03             \\
        Rhode James & 1862-08-01             \\
    \end{tabular}
\end{center}


\subsection{Sélection avec conditions multiples}
\begin{minted}{sql}
SELECT nom, date_naissance
FROM Auteur
WHERE date(date_naissance) < '1900'
  AND nom_pays = 'France';
    \end{minted}

\begin{center}
    \tabstyle[UGLiOrange]
    \begin{tabular}{c|c}
        \ccell nom & \ccell date\_naissance \\
        Hugo       & 1802-02-16             \\
    \end{tabular}
\end{center}


\subsection{Renommer les colonnes}
\begin{minted}{sql}
SELECT titre AS Titre_ouvrage, num_isbn AS Reference_ISBN
FROM Livre
WHERE date(annee) > '2015';
\end{minted}

\begin{center}
    \tabstyle[UGLiOrange]
    \begin{tabular}{c|c}
        \ccell Titre\_ouvrage & \ccell Reference\_ISBN \\
        les misérables        & 9782072730672          \\
        et je t'emmène        & 9782221133651          \\
        d'acier               & 9782867465987          \\
        salvation             & 9791093835334          \\
    \end{tabular}
\end{center}

\subsection{Fonction COUNT}
\begin{minted}{sql}
SELECT COUNT(titre) AS Nb_Livres_avant_2015
FROM Livre
WHERE annee < 2015;    
\end{minted}

\begin{center}
    \tabstyle[UGLiOrange]
    \begin{tabular}{c}
        \ccell Nb\_Livres\_avant\_2015 \\
        4
    \end{tabular}
\end{center}

\subsection{Autres fonctions similaires}
Fonctions MIN, MAX, SUM et AVG (moyenne).

\subsection{\'Eliminer les doublons}
Sans élimination :
\begin{minted}{sql}
SELECT id_auteur
FROM Ecrire;
    \end{minted}

\begin{center}
    \tabstyle[UGLiOrange]
    \begin{tabular}{c}
        \ccell id\_auteur \\
        1                 \\
        1                 \\
        2                 \\
        4                 \\
        5                 \\
        6                 \\
        7                 \\
        8                 \\
        9
    \end{tabular}
\end{center}

Avec élimination :
\begin{minted}{sql}
SELECT DISTINCT id_auteur
FROM Ecrire;
    \end{minted}

\begin{center}
    \tabstyle[UGLiOrange]
    \begin{tabular}{c}
        \ccell id\_auteur \\
        1                 \\
        2                 \\
        4                 \\
        5                 \\
        6                 \\
        7                 \\
        8                 \\
        9
    \end{tabular}
\end{center}
\subsection{Ordonner les tuples}
Ordonner les noms dans l'ordre croissant :
\begin{minted}{sql}
SELECT nom,prenom FROM Auteur
ORDER BY nom ASC;
    \end{minted}

\begin{center}
    \tabstyle[UGLiOrange]
    \begin{tabular}{c|c}
        \ccell nom  & \ccell prenom \\
        Ammaniti    & Niccolo       \\
        Avallone    & Silvia        \\
        Camus       & Albert        \\
        Hamilton    & Peter         \\
        Hugo        & Victor        \\
        Murgia      & Michela       \\
        Rhode james & Montague      \\
        Tolkien     & John
    \end{tabular}
\end{center}
Pour l'ordre décroissant on utilise \mintinline{sql}{DESC}.


\section{Jointures}
\subsection{Principe}
Considérons 2 tables T1 et T2 et supposons que c est une clé étrangère qui fait référence à b.

\begin{center}
    \tabstyle[UGLiOrange]
    \begin{tabular}{c|c}
        \ccell a & \ccell b \\
        0        & 0        \\
        0        & 1        \\
        1        & 2        \\
        2        & 3        \\
        3        & 4        \\
        4        & 5 
    \end{tabular}\hspace{3em}
    \begin{tabular}{c|c}
        \ccell c & \ccell d \\
        0        & 10       \\
        0        & 30       \\
        1        & 12       \\
        2        & 100      \\
        2        & 200 
    \end{tabular}
\end{center}

Voici table qui est la \textit{jointure} T1 et T2 selon la condition b=c :


\begin{center}
    \tabstyle[UGLiOrange]
    \begin{tabular}{c|c|c|c}
        \ccell a & \ccell b & \ccell c & \ccell d \\
        0        & 0        & 0        & 10       \\
        0        & 0        & 0        & 30       \\
        0        & 1        & 1        & 12       \\
        1        & 2        & 2        & 100      \\
        1        & 2        & 2        & 200      \\
    \end{tabular}
\end{center}
C'est la table obtenue en faisant correspondre chaque tuple de T1 avec chaque autre tuple de T2 tel que b et c soient égaux.

\subsection{Applications}
Produire la table des noms des auteurs venant de pays de plus de 61 millions d'habitants :
\begin{minted}{sql}
SELECT nom
from Auteur
         JOIN Pays ON Auteur.nom_pays = Pays.nom_pays
WHERE population > 62000000;
    \end{minted}

\begin{center}
    \tabstyle[UGLiOrange]
    \begin{tabular}{c}
        \ccell nom \\
        Hugo       \\
        Camus      \\
        Avallone   \\
        Ammaniti   \\
        Murgia
    \end{tabular}
\end{center}



Produire la table des noms et prénoms des auteurs ayant écrit un livre dont le titre comporte « la»  :
\begin{minted}{sql}
SELECT DISTINCT nom, prenom
FROM Auteur
         JOIN Ecrire ON Ecrire.id_auteur = Auteur.id_auteur
         JOIN Livre ON Livre.num_isbn = Ecrire.num_isbn
WHERE Livre.titre LIKE '%la%';
\end{minted}

\begin{center}
    \tabstyle[UGLiOrange]
    \begin{tabular}{c|c}
        \ccell nom  & \ccell prenom \\
        Camus       & Albert        \\
        Rodhe James & Montague
    \end{tabular}
\end{center}

\section{Mises à jour}


\subsection{INSERT INTO}
Insérer un nouveau tuple dans la table \textbf{Auteur} :
\begin{minted}{sql}
INSERT INTO Auteur VALUES
    (128,'France','Leleu','Frédéric','1974-05-16');
\end{minted}

Les colonnes doivent être dans le même ordre qu'à la création, sinon utiliser
\begin{minted}{sql}
INSERT INTO Auteur VALUES (nom,id_auteur)
    ('Leleu',128);
\end{minted}

Les colonnes non renseignées prendront par défaut la valeur \mintinline{sql}{NULL} ce qui peut poser problème.


\subsection{DELETE}

Supprimer les tuples de \textbf{Ecrire} dont l'auteur a l'id\_auteur 1:
\begin{minted}{sql}
DELETE FROM Ecrire WHERE id_auteur = 1;
\end{minted}

Penser aux contraintes de références (clé étrangères) : si on supprime un tuple et qu'un tuple d'une autre table fait référence à celui qu'on supprime, cela provoquera une erreur.


\subsection{UPDATE}

Mettre à jour l'id du tuple de \textbf{Auteur} dont le nom est Hugo
\begin{minted}{sql}
UPDATE Auteur
SET id_auteur = 1024
WHERE nom = 'Hugo';
	\end{minted}

Penser aux contraintes de références (clé étrangères) lors de la mise à jour.
\section{Exercices}
Ils sont à faire sut \textsc{Capytale}.
\end{document}