\documentclass[10pt]{nsibeamer}

\title{Programmation défensive}
\subtitle{Chapitre 11}
\author{NSI2}

\begin{document}
\maketitle

\begin{frame}{Ce n'est pas ça\ldots}
    \begin{center}
    \includegraphics[height=8cm]{img/kf.jpg}
    \end{center}
\end{frame}
\section{Gestion des exceptions}


\begin{frame}[fragile]{L'utilisateur et ses facéties}
    \begin{minted}{python}
        def inverse(x : float) -> float :
            return 1 / x
        \end{minted}
\end{frame}



\begin{frame}[fragile]{L'utilisateur et ses facéties}
    L'utilisateur évalue \mintinline{python}{inverse(0)} et obtient :\\\pause
    
    {\color{red}\texttt{Traceback (most recent call last):\\
            \hspace{2em}File "fonction1.py", line 4, in <module>\\
            \hspace{4em}inverse(0)\\
            \hspace{2em}File "fonction1.py", line 2, in inverse\\
            \hspace{4em}return 1 / x\\
            ZeroDivisionError: division by zero}}
\end{frame}

\begin{frame}[fragile]{Encore plus fou}
    De la même manière lorsque l'utilisateur évalue \mintinline{python}{inverse('chaussette')} il obtient :\\\pause
    
    {\color{red}\texttt{Traceback (most recent call last):\\
            \hspace{2em}File "fonction1.py", line 4, in <module>\\
            \hspace{4em}inverse(0)\\
            \hspace{2em}File "fonction1.py", line 2, in inverse\\
            \hspace{4em}return 1 / x\\
            TypeError: unsupported operand type(s) for /: 'int' and 'str'}}
\end{frame}

\begin{frame}[fragile]{Notion d'exception}
    \mintinline{python}{ZeroDivisionError} et \mintinline{python}{TypeError} sont deux représentants de ce qu'on appelle des \textit{exceptions}.\\\pause
    \textit{Une exception est levée} (\textit{an exception is raised} en Anglais) lorsque l'interpréteur \textsc{Python} rencontre un problème qu'il ne peut résoudre ou bien que le programme lui-même indique que ce doit être le cas.    
\end{frame}

\begin{frame}[fragile]{Exceptions}
    Les exceptions les plus courantes sont ces deux premières ainsi que 
    \begin{itemize}
        \item   \mintinline{python}{NameError} pour une variable non définie;
        \item   \mintinline{python}{IndexError} pour un indice de liste trop grand;
        \item   \mintinline{python}{KeyError} pour une clé de dictionnaire inexistante;
    \end{itemize}
    
    On pourra trouver la liste de toutes les exceptions ici :\\
    \texttt{https://docs.python.org/fr/3.8/library/exceptions.html}.       
\end{frame}

\begin{frame}[fragile]{Syntaxe de la gestion des exceptions}
    \pause
    \begin{itemize}
        \item   la partie du code susceptible de lever une exception est mise dans un bloc \mintinline{python}{try} ;\pause
        \item   si une exception est levée on la gère avec un bloc \mintinline{python}{except}.
    \end{itemize}
\end{frame}

\begin{frame}[fragile]{Exemple}
    \begin{minted}[fontsize=\small]{python}
            def inverse(x: float) -> float:
                try:
                    return 1 / x
                except ZeroDivisionError:
                    print('Erreur - 1 /', x, ': division par zéro.')
                except TypeError:
                    print('Erreur - 1 /', x, ': type incorrect.')
                return 0

        \end{minted}
\end{frame}

\begin{frame}[fragile]{Remarque}
    Les \mintinline{python}{try .. except} n'éliminent pas les erreurs mais permettent de les « intercepter ». 
\end{frame}

\begin{frame}[fragile]{Conseil}
    \pause
    Toujours préciser le type de l'exception qui a été levée sinon ce n'est pas clair.\pause
    
    \begin{minted}{python}
        def inverse(x: float) -> float:
            try:
                return 1 / x
            except:
                print('Erreur avec 1 /', x)
                return 0    
    \end{minted}
    \pause
    Dans le bloc \mintinline{python}{except} on ne sait pas quelle erreur est survenue. 
\end{frame}


\section{Tests unitaires}
\begin{frame}[fragile]{Définition}
    Un (ou des) test(s) unitaire(s) sert à vérifier qu'une partie d'un programme (une \textit{unité}) fonctionne comme on l'a prévu.
\end{frame}

\begin{frame}[fragile]{Conseils}\pause
    Écrire quelques tests \textit{avant même d'écrire le programme}.\\\CurrentFilePathUsed
    
    Les tests doivent être pensés pour aborder le cas général ainsi que les cas particuliers.
\end{frame}

\begin{frame}[fragile]{Utilisation de \mintinline{python}{assert}}\pause
    \mintinline{python}{assert} sert à vérifier qu'une \textit{assertion} (un test de condition) est vraie.\\\pause
    
    Si c'est le cas le programme continue, sinon une exception du type \mintinline{python}{AssertionError} est levée.
    
\end{frame}

\begin{frame}[fragile]{Exemple}\pause
    Coder une fonction \mintinline{python}{sort} (qui signifie \textit{trier} en Anglais) qui     
    \begin{itemize}
        \item   en entrée prend une liste d'\mintinline{python}{int};
        \item   en sortie renvoie une liste qui contient les mêmes valeurs que la liste d'entrée, mais triées dans l'ordre croissant.
    \end{itemize}
\end{frame}

\begin{frame}[fragile]{On écrit d'abord les tests}\pause
    
    \begin{minted}{python}
    def sort(l: list) -> list:
        ...


    assert sort([]) == []
    assert sort([0, 2, 3, 1]) == [0, 1, 2, 3]
    assert sort([1, 0, 1, 2]) == [0, 1, 1, 2]
    \end{minted}
\end{frame}

\begin{frame}[fragile]{Attention}
    Une batterie de tests ne constitue pas une preuve.
\end{frame}

\begin{frame}[fragile]
    Différence \mintinline{python}{try ... except}\\
    et\\
    \mintinline{python}{assert}  
\end{frame}
\begin{frame}[fragile]{Différence}\pause
    \mintinline{python}{try ... except} pour détecter et contrôler les erreurs.\\\pause
    
    \mintinline{python}{assert} pour vérifier conditions, résultats.\\\pause

    Les premières ont vocation à rester dans le programme, les secondes non.
\end{frame}

\section{Utiliser des modules déjà écrits}

\begin{frame}[standout]
    Parce que c'est crétin\\
    de chercher à \\
    réinventer la roue.
\end{frame}

\begin{frame}[standout]
    sauf...    
\end{frame}

\begin{frame}[fragile]{Raisons}\pause
    \begin{itemize}
        \item   pas satisfait des performances du module (trop lent, trop gourmand en mémoire) ;\pause
        \item   une situation technique particulière fait que vous ne pouvez pas l'utiliser dans votre projet ;\pause
        \item   le prof le demande, pour se former !
    \end{itemize}
        
\end{frame}



\end{document}