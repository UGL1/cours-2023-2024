\documentclass[12pt,firamath,article]{nsi}
\begin{document}
\titre{Passage à la caisse}
\classe{NSI2}
\maketitle
Une grande surface possède n caisses. On choisit une unité de temps arbitraire (le tour) et on décide que lorsqu'un client passe à la caisse, cela prend un temps aléatoire compris entre 1 et n unités de temps.\\
À chaque unité de temps, un client arrive aux caisses.\\
On commence la simulation avec n clients qui arrivent.\\
On aimerait simuler le temps d'attente aux caisses et évaluer le temps d'attente moyen par client.\\
\section*{Avec une seule file}

On décide qu'une seule file existe : les clients attendent dans la file et dès qu'une caisse se libère, le premier client dans la file est reçu.

\subsection*{Simulation}
 Voici un début de simulation avec n=3 caisses
\subsubsection*{Initialisation}
\picleft{0.2}{img/single_queue00}{
L'heure est \texttt{00}.\\
Puisqu'il y a 3 caisses, on place 3 clients dans la file (les carrés verts).
Pour l'instant personne n'a attendu donc le total des temps d'attente est \texttt{00}.\\
Les trois caisses sont libres : elles sont bleues, avec un temps d'attente de zéro.}

\subsubsection*{Première itération}
\picleft{0.2}{img/single_queue01}{
L'horloge a fait un tour.\\
Les 3 clients sont passés en caisse : les deux dernières prendront 3 tours pour traiter les achats de son client, la première 2 tours.\\
Un nouveau client se présente et attend.}
\subsubsection*{Deuxième itération}
\picleft{0.2}{img/single_queue02}{
L'horloge a fait un tour.\\
Un nouveau client arrive dans la file.\\
Il est obligé d'attendre lui aussi.
Pour l'instant le temps d'attente total n'est pas actualisé : on attend qu'un client soit servi avant de comptabiliser son temps d'attente.}
\subsubsection*{Troisième itération}
\picleft{0.2}{img/single_queue03}{
L'horloge a fait un tour.\\
Le client qui attendait depuis 2 tours est reçu, on actualise le temps d'attente total.
Il passe dans la première caisse.
Au total, 4 clients ont été reçus, avec un temps d'attente total de 2 tours, donc un temps d'attente moyen de 0.5 tour par client.\\
Un nouveau client se présente dans la file.}
\subsubsection*{Quatrième itération}
\picleft{0.2}{img/single_queue04}{
L'horloge a fait un tour.\\
Les 2 clients précédents sont servis, on ajoute leurs temps d'attente, un nouveau client arrive, \textit{et c\ae tera}.}

\subsection*{Conseils pour démarrer}

\subsubsection*{File}
Pour commencer on peut créer une file \mintinline{python}{file_attente} et une variable \mintinline{python}{tour} valant \mintinline{python}{0}.\\
Ensuite, pour se rappeler de l'heure d'arrivée d'un client il suffit d'enfiler son heure d'arrivée, c'est-à-dire la valeur de la variable \mintinline{python}{tour}.\\
On peut définir une constante \mintinline{python}{NB_CAISSES} et mettre en file \mintinline{python}{NB_CAISSE} clients.\\

\subsubsection*{Caisses}
On peut créer une classe \texttt{Caisse} qui va fonctionner un peu comme la classe \texttt{Ball} déjà rencontrée, avec
\begin{itemize}
\item Une variable \textbf{de classe} \texttt{nb\_caisses} valant 0 au départ;
\item Une variable \textbf{de classe} \texttt{caisses} de type \mintinline{python}{list}, valant \mintinline{python}{[]} au départ, pour stocker les différentes caisses;
\item Une variable \textbf{de classe} \texttt{nb\_clients\_servis} valant 0;
\item Une méthode \mintinline{python}{__init__} qui crée des instances de classes avec (au minimum) les attributs suivants :
\begin{itemize}
    \item \mintinline{python}{self.file}, la file d'attente, passée en paramètre dans les constructeur (ainsi dans les prochaines partie, chaque caisse pourra avoir sa propre file);
    \item \mintinline{python}{self.temps_attente}, un \mintinline{python}{int} mesurant le nombre de tours avant que la caisse soit libre;
\end{itemize}
Quand le constructeur est appelé, \mintinline{python}{Caisse.nb_caisses} augmente de 1 et l'instance (\mintinline{python}{self}) est ajoutée à la liste \mintinline{python}{Caisse.caisses}.
\item Une méthode \mintinline{python}{sert_client} qui
\begin{itemize}
    \item commence par enlever un client de sa file : alors on récupère son heure d'arrivée (notons la \texttt{heure}) dans la file et \mintinline{python}{tour-heure} nous donne son temps d'attente, qu'on peut ajouter à la variable \mintinline{python}{Caisse.temps_attente_total};
    \item  comme on sert un nouveau client, on peut incrémenter \mintinline{python}{Caisse.nb_clients_servis}
    \item  le temps passé à s'occuper du client est \mintinline{python}{randint(1,NB_CAISSES)} et devient la nouvelle valeur de l'attribut \mintinline{python}{temps_attente} de la caisse.
\end{itemize}
\item Une méthode \mintinline{python}{actualise} qui sera plus tard appelée une fois par tour et :
    \begin{itemize}
        \item enlèvera 1 au temps d'attente de la caisse;
        \item si elle est libre, servira un client (s'il y en a dans la file);
    \end{itemize}
\subsubsection*{Boucle principale}
On pourra créer \mintinline{python}{NB_CAISSES} caisses, créer une constante \mintinline{python}{NB_TOURS} et boucler sur la variable \mintinline{python}{tour}: tant qu'on a pas atteint  \mintinline{python}{NB_TOURS}, on
\begin{itemize}
\item fait arriver un client dans la file;
\item parcourt la liste \mintinline{python}{Caisse.caisses};
\item actualise chaque caisse;
\item termine en ajoutant 1 à \mintinline{python}{tour}
\end{itemize}
\subsection*{Fin du programme}
C'est à vous de jouer, vous avez tout pour calculer le temps moyen d'attente par client servi.
\end{itemize}

\section*{Plusieurs files, au hasard}
Adapter le programme précédent avec une file par caisse, avec 1 client par file au départ, et les suivants arrivent en choisissant une file au hasard sans en changer.

\section*{Plusieurs files, au hasard}
Adapter le programme précédent avec une file par caisse, avec 1 client par file au départ, et les suivants arrivent en choisissant une caisse libre ou une avec le moins de monde.

\section*{Bilan}
Dresser le bilan des 3 méthodes.


\end{document}