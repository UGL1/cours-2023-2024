\documentclass[a4paper,12pt,french]{article}
\usepackage[margin=2cm]{geometry}
\usepackage[thinfonts]{uglix2}
%
% UGLiPurple		UGLiRed			UGLiOrange		UGLiYellow		UGLiGreen		UGLiDarkGreen	UGLiBlue	
% UGLiDarkBlue		UGLiBlack		UGLiWhite
%
% TolDarkPurple		TolDarkBlue		TolLightBlue	TolLightGreen	TolDarkGreen	TolDarkBrown	TolLightBrown	
% TolDarkRed		TolLightRed		TolLightPink	TolDarkPink		TolLightPurple
%
\begin{document}
\titre{Adressage réseau}{NSI2}{2022} 

\begin{encadrecolore}{Paire réseaux / masque : notation CIDR}{UGLiDarkGreen}

On rappelle qu'une adresse IP version 4 est de la forme xxx.xxx.xxx.xxx, soit 4 octets, généralement écrits en base 10. On ne parlera pas d'IP version 6.\\

Un réseau possède une adresse IP, chaque machine du réseau également, et une dernière adresse IP, la « plus grande » dite de \textit{broadcast}, est réservée à la diffusion sur tout le réseau.\\

La notation CIDR permet de donner l'adresse et la taille d'un réseau. Elle est de la forme 
\begin{center}
xxx.xxx.xxx.xxx / n
\end{center}
Où n est un entier compris entre 1 et 32 et le reste une adresse IP (celle du réseau).\\
Le nombre n correspond au nombre de bits fixes en partant de la gauche, les autres bits sont libres.\\
Ainsi plus n est petit, plus il y a de bits libres, plus il y a d'adresses disponibles sur le réseau et plus celui-ci est susceptible d'être grand.
\end{encadrecolore}

\begin{exemple}
En notation CIDR, mon réseau domestique est 192.168.1.0 / 24.
\begin{enumerate}[--]
	\item 192.168.1.0 est l'adresse du réseau ;
	\item les 24 bits en partant de la gauche sont fixes, donc les 3 premiers octets sont fixes, et ainsi seul le dernier octet de l'IP peut varier ;
	\item les adresses IP du réseau sont de la forme 192.168.1.xxx ;
	\item l'adresse de \textit{broadcast} est 192.168.1.255.
\end{enumerate}
\end{exemple}

\begin{definition}[ : masque de sous-réseau]
Le masque de sous réseau d'un réseau de la forme xxx.xxx.xxx.xxx / n consiste en une adresse IP dont les n premiers bits sont à 1 et les autres à 0.
\end{definition}

\begin{exemple}
Le masque de sous-réseau de 192.168.1.0 / 24 est
\begin{center}
\texttt{11111111 . 11111111 . 11111111 . 00000000}
\end{center}
\end{exemple}

\begin{methode}[ : trouver l'adresse d'un réseau]
L'IP de ma machine dans un réseau xxx.xxx.xxx.xxx / 20 est 192.168.181.3.\\

Quelle est l'adresse du réseau ?
\begin{enumerate}[--]
	\item On écrit l'IP de la machine en binaire, on obtient 
	\begin{center}\texttt{11000000 . 10101000 . 10110101 . 00000011}\end{center}
	\item Le masque de sous réseau est 
	\begin{center}\texttt{11111111 . 11111111 . 11110000 . 00000000}\end{center}
	\item On fait un \textsc{et} bit à bit entre ces deux adresses, on obtient 
	\begin{center}\texttt{11000000 . 10101000 . 10110000 . 00000000}\end{center}
	\item On écrit cela en décimal : l'adresse IP du réseau est 192.168.176.0.\\
\end{enumerate}
\end{methode}
\begin{methode}[ : trouver l'adresse de diffusion]
Quelle est l'adresse de diffusion du réseau précédent ?
\begin{enumerate}[--]
	\item On reprend l'adresse du réseau en binaire 
	\begin{center}\texttt{11000000 . 10101000 . 10110000 . 00000000}\end{center}
	\item Puisque le réseau est en / 20, on met les 12 derniers bits à 1 :
	\begin{center}\texttt{11000000 . 10101000 . 10111111 . 11111111}\end{center}
	\item En écrivant en décimal, l'adresse IP de diffusion sur le réseau est 192.168.191.255.
\end{enumerate}

\end{methode}

\begin{exercice}[ : réseaux et masques]

Dans chaque cas, l'IP d'une machine sur un réseau est donnée, ainsi que le masque du réseau.
\begin{enumerate}[--]
	\item Dire combien d'IP comporte le réseau et combien de machines peuvent être adressées.
	\item Retrouver l'IP du réseau.
	\item Donner l'IP de \textit{broadcast}.
\end{enumerate}

\begin{enumerate}
\item 202.2.18.149 sur un réseau en / 8.
\item 97.124.36.142 sur un réseau en / 24.
\item 192.168.180.57 sur un réseau en / 18.
\end{enumerate}
\end{exercice}

\begin{exercice}
Dire si l'IP appartient au réseau.
\begin{enumerate}
	\item 172.26.21.46 et 172.26.21.0 / 25
	\item 192.168.186.240 et 192.168.186.224 / 29
	\item 172.18.47.54 et 172.18.46.0 / 23
\end{enumerate}


\end{exercice}

\begin{exercice}[ : projet qu'on fera sans doute après les écrits]
Les méthodes vues précédemment peuvent être automatisées avec un programme.\\

Tu peux écrire une classe \texttt{IP} pour représenter une IP, implémenter la méthode \mintinline{python}{__str__} et  également la méthode \mintinline{python}{__and__} pour effectuer un \texttt{et} bit à bit entre 2 IPs, ce qui te permettra d'écrire l'opération en \textsc{Python} \mintinline{python}{ip3 = ip1 & ip2}.\\

Ensuite tu pourras écrire une classe \mintinline{python}{Network}, initialisée avec une IP d'une machine sur le réseau (ou du réseau) et le nombre de bits du masque. Il sera alors possible d'obtenir le masque de sous-réseau, l'adresse de \textit{broadcast}, tu peux même implémenter \mintinline{python}{__contains__} pour vérifier si une IP est une IP du réseau ou non... \\

Voici un  exemple d'utilisation

\begin{minted}[fontsize=\small]{python}
ip1 = IP(192, 168, 181, 3)      # IP d'une machine sur un réseau
print(ip1.to_bin())             # affiche l'IP en binaire

n = Network(ip1, 20)            # réseau avec la machine, 20 bits fixes

print(n.mask_ip.to_bin())       # affiche le masque en binaire
print(n.mask_ip)                # puis en décimal
print(n.network_ip.to_bin())    # IP du réseau en binaire
print(n.network_ip)             # puis en décimal
print(n.broadcast_ip.to_bin())  # IP de broadcast en binaire
print(n.broadcast_ip)           # puis en décimal

ip2 = IP(192,168,192,0)         # une deuxième IP
ip3 = IP(192,168,180,255)       # une troisième
print(ip2 in n)                 # ip2 est-elle sur le réseau ?
print(ip3 in n)                 # ip3 est-elle sur le réseau ?
\end{minted}

Et voici le résultat
\begin{minted}{python}
11111111.11111111.11110000.00000000
255.255.240.0
11000000.10101000.10110000.00000000
192.168.176.0
11000000.10101000.10111111.11111111
192.168.191.255
False
True
\end{minted}
\end{exercice}


\end{document}