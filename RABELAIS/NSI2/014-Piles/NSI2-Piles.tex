\documentclass[10pt,cours,firamath,a4paper]{nsi}

\begin{document}
\chapter{Piles}
\section{Principe}
\picleft{0.5}{img/cours_stack}{On part d'une structure vide sur laquelle on peut empiler au fur et à mesure.\\

Seul l'élément du sommet est immédiatement accessible.\\

On peut dépiler, les dernières valeurs empilées sont les premières dépilées : on parle de LIFO (\textit{Last In First Out}).}

\begin{encadrecolore}{Applications}{UGLiGreen}
	\begin{itemize}
		\item 	Lors de la navigation sur le web, on peut considérer que les liens sont sauvegardés sur une pile par le navigateur.
		\item 	De même pour les logiciels qui utilisent la fonction \texttt{annuler} (le fameux \texttt{CTRL+Z}).
		\item	Lors d'appels de fonctions, les états-mémoires sont sauvegardés sur une pile (notamment lors d'appels récursifs, on l'a déjà vu).
		\item 	Des piles sont utilisées dans divers algorithmes, notamment pour parcourir un arbre « en profondeur ».
		\item 	D'autres applications sont données en exercice.
	\end{itemize}
\end{encadrecolore}


\subsection{Interface de la structure de données pile}

Elle est très simple !
\begin{encadrecolore}{Interface de la pile}{UGLiBlue}
	\begin{itemize}
		\item \textit{pile\_vide()} créée une pile vide ;
		\item \textit{empiler(pile,valeur)} empile la valeur sur la pile ;
		\item  \textit{depiler(pile)} renvoie la valeur sur la pile et l'enlève de la pile ;
		\item \textit{est\_vide()} indique si la pile est vide on non ;
	\end{itemize}
\end{encadrecolore}
\subsection{Implémentations de l'interface}
\begin{itemize}
	\item Simple liste python..
	\item Liste encapsulée dans un objet.
	\item Listes imbriquées (pour se compliquer la vie) :\\
	      \begin{minted}{python}
[]
[1] # on a empilé 1
[2, [1]] # puis 2
[3, [2, [1]]] # puis 3
...
\end{minted}
\end{itemize}

\begin{encadrecolore}{Implémentation objet en Python}{UGLiPurple}
\begin{minted}{python}
class Stack:
    def __init__(self):
        """ Creates an empty stack """
        self.content = []
        
    def is_empty(self) -> bool:
        """ Indicates whether the stack's empty or not """
        return self.content == []
\end{minted}


\begin{minted}{python}
    def push(self, value):
        """ Pushes the value on the top of the stack """
        self.content.append(value)

    def pop(self):
        """ Retrieves the value from the top of the stack """
        if self.is_empty():
            raise IndexError('Stack Empty')
        return self.content.pop()
	\end{minted}
\end{encadrecolore}

\section{Exercices}

\begin{exercice}[ : Implémentation objet]
	\'Ecrire une classe \mintinline{python}{Stack} qui implémente la structure de pile vue en cours. 

	Sauvegarder dans un module \texttt{Stack.py}. Celui-ci servira pour les exercices suivants.
\end{exercice}

\begin{exercice}[ : Les piles c'est renversant]
\'Ecrire une fonction \mintinline{python}{reverse_with_stack} qui utilise une pile et
\begin{itemize}
	\item 	en entrée prend un \mintinline{python}{str};
	\item 	en sortie renvoie un \mintinline{python}{str} qui est la chaîne passée en paramètre, mais dans l'autre sens.	
\end{itemize}
\end{exercice}

\begin{exercice}[ : Expressions bien parenthésées]
L'expression $(a+b\times(c+(a+b)^2))^3$ est bien parenthésée. $(a-b)\times c)$ ne l'est pas.
\begin{enumerate}
	\item 	\'Ecrire une fonction \mintinline{python}{is_valid} qui utilise une pile et
	\begin{itemize}
		\item 	en entrée prend in \mintinline{python}{str} qui est l'expression à tester.
		\item 	en sortie renvoie un \mintinline{python}{bool} : \mintinline{python}{True} si elle est bien parenthésée, \mintinline{python}{False} sinon.
	\end{itemize}
	\item 	Proposer un ensemble de tests unitaires.	
\end{enumerate}
\end{exercice}

\begin{exercice}[ : \'Egalité de deux piles]
	\begin{enumerate}
		\item 	\'Ecrire une fonction \mintinline{python}{are_equal} qui
		\begin{itemize}
			\item 	en entrée prend 2 piles.
			\item 	en sortie renvoie un \mintinline{python}{bool} : \mintinline{python}{True} si les deux piles ont les mêmes éléments (même nombre et même ordre), \mintinline{python}{False} sinon.
		\end{itemize}
		La fonction doit remettre les piles telles quelles.
	\item 	Proposer un ensemble de tests unitaires.	
	\end{enumerate}

\end{exercice}

\begin{exercice}[ : Notation polonaise inversée]
La Notation Polonaise Inverse (NPI), ou notation post-fixée, est une marnière d'écrire les expressions mathématiques en se passant des parenthèses. Elle a été introduite par le mathématicien polonais Jan Lucasievicz dans les années 1920.\\
Le principe de cette méthode est de placer chaque opérateur juste après ses deux opérandes.\\
Par souci de simplicité nous ne considèrerons que des expressions mettant en jeu des entiers naturels.\\

L'expression $2 + 3$ devient en NPI  2 3 +.
\begin{itemize}
	\item $2 + 6 − 1 $ s'écrit  2\ 6 + 1 −
	\item $5 \times 3 + 4$ s'écrit  5\ 3 * 4 +
	\item $((1 + 2) \times 4) + 3$ s'écrit 1 2 + 4 * 3 +
\end{itemize}

\'Evaluer une expression post-fixée est facile. Pour cela il suffit de lire l'expression de gauche à
droite et d'appliquer chaque opérateur aux deux opérandes qui le précèdent. Si l'opérateur n'est
pas le dernier symbole on replace le résultat intermédiaire dans l'expression et on recommence avec
l'opérateur suivant.\\

On peut écrire un programme en Python qui utilise une pile s pour évaluer les expressions en NPI en suivant cet exemple
\begin{itemize}
	\item 	on considère une chaine de caractères : \mintinline{python}{c = '23 6 + 1 -'};
	\item 	on la transforme en une liste : \mintinline{python}{lst = c.split()};
	\item 	la valeur de \mintinline{python}{lst} est alors \mintinline{python}{['23', '6', '+', '1', '-']};
	\item 	on constate que \mintinline{python}{'23'} et \mintinline{python}{'6'} ne sont pas des symboles d'opérateurs donc sont convertibles en \mintinline{python}{int} et on empile ces deux entiers dans s;
 	\item 	le prochaine élément de \mintinline{python}{lst} est \mintinline{python}{'+'} donc on dépile les deux entiers 6 et 23, on les ajoute, et on empile le résultat 29;
 	\item	on continue : on empile 1;
 	\item	on arrive à \mintinline{python}{'-'}, on dépile les 2 entiers  1 et 29 et on les soustrait (dans le bon ordre) et on empile le résultat final 28.
\end{itemize}

\'Ecrire une fonction \mintinline{python}{mpi_compute} qui
\begin{itemize}
	\item 	en entrée prend un \mintinline{python}{str} qui est une expression NPI;
	\item 	en sortie renvoie un \mintinline{python}{float} qui est l'évaluation de cette expression (\mintinline{python}{float} car l'expression peut comporter des divisions).	
\end{itemize}
\end{exercice}
\end{document}