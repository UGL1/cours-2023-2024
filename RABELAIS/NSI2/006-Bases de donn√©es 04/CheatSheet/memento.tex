\documentclass[a4paper,10pt,article]{nsi}
\setminted{fontsize=\small}


\begin{document}
\titre{MÉMENTO SQL}
\classe{NSI2}
\maketitle


\section*{Création de BDD}

Si la base n'existe pas
\begin{sql}
    \begin{minted}{sql}
CREATE DATABASE ma_base;
\end{minted}
\end{sql}

Sinon si on veut la recréer
\begin{sql}
    \begin{minted}{sql}
DROP DATABASE ma_base;
CREATE DATABASE ma_base;
\end{minted}
\end{sql}

\section*{Création de tables}

Cas de base.
\begin{sql}
    \begin{minted}{sql}
DROP TABLE IF EXISTS MaTable;
CREATE TABLE MaTable(
    attribut_1 INT PRIMARY KEY
    attribut_2 TEXT,
    attribut_3 VARCHAR(50) -- TEXT limité à 50 caractères
);
\end{minted}
\end{sql}

Création avec id clé primaire automatique (voir comment insérer des tuples plus loin)$^*$
\begin{sql}
    \begin{minted}{sql}
DROP TABLE IF EXISTS MaTable;
CREATE TABLE MaTable(
    attribut_1 INT PRIMARY KEY AUTOINCREMENT,
    attribut_2 TEXT,
    attribut_3 VARCHAR(50) 
);
\end{minted}
\end{sql}

Création avec clé étrangère (référence à une clé primaire d'une autre table).
\begin{sql}
    \begin{minted}{sql}
DROP TABLE IF EXISTS MaTable;
CREATE TABLE MaTable(
    attribut_1 INT PRIMARY KEY,
    attribut_2 TEXT FOREIGN KEY REFERENCES autre_table (attribut_de_reference),
    attribut_3 VARCHAR(50)
); 
\end{minted}
\end{sql}

\section*{Insérer des tuples}

Dans une table, avec toutes les colonnes renseignées dans le bon ordre
\begin{sql}
    \begin{minted}{sql}
INSERT INTO MaTable VALUES
(val_1_attribut_1, val_1_attribut_2, val_1_attribut_3),
(val_2_attribut_1, val_2_attribut_2, val_2_attribut_3);
\end{minted}
\end{sql}

Dans une table, avec par exemple la clé primaire créé automatiquement$^*$
\begin{sql}
    \begin{minted}{sql}
INSERT INTO MaTable(attribut_2, attribut_3) VALUES
(val_1_attribut_2, val_1_attribut_3),
(val_2_attribut_2, val_2_attribut_3);
\end{minted}
\end{sql}

\section*{Requêtes simples}

Afficher toute une table
\begin{sql}
    \begin{minted}{sql}
SELECT * 
FROM MaTable;
\end{minted}
\end{sql}

Afficher certaines colonnes d'une table
\begin{sql}
    \begin{minted}{sql}
SELECT attribut_2, attribut_3 
FROM MaTable;
\end{minted}
\end{sql}

Afficher certaines colonnes d'une table en renommant les colonnes
\begin{sql}
    \begin{minted}{sql}
SELECT attribut_2 AS nouveau_nom, attribut_3 
FROM MaTable;
\end{minted}
\end{sql}

Afficher une table avec condition
Afficher certaines colonnes d'une table
\begin{sql}
    \begin{minted}{sql}
SELECT * 
FROM MaTable
    WHERE <condition(s)>;
\end{minted}
\end{sql}

Conditions : 
\begin{itemize}
    \item \mintinline{sql}{var_num BETWEEN val1 AND val2}
    \item \mintinline{sql}{var_txt LIKE "%motif%"} avec \texttt{\%} avant et/ou après le motif
    \item utilisation de \mintinline{sql}{<, >, <=, >=, =, AND, OR, NOT, IS NULL, IS NOT NULL}
\end{itemize}

Trier la table produite selon une colonne

\begin{sql}
    \begin{minted}{sql}
SELECT * 
FROM MaTable
    ORDER BY attribut_1 ASC; -- ou DESC
\end{minted}
\end{sql}

Limiter l'affichage de la table produite à 10 lignes
\begin{sql}
    \begin{minted}{sql}
SELECT * 
FROM MaTable
    LIMIT 10;
\end{minted}
\end{sql}

\begin{sql}
    \begin{minted}{sql}
SELECT * 
FROM MaTable
    ORDER BY attribut_1 ASC; -- ou DESC
\end{minted}
\end{sql}


\section*{Fonctions d'agrégation}

Compter les lignes d'une table suivant une condition

\begin{sql}
    \begin{minted}{sql}
SELECT COUNT(*) 
FROM MaTable
    WHERE <condition(s)>;
\end{minted}
\end{sql}

Faire la somme sur une colonne
\begin{sql}
    \begin{minted}{sql}
SELECT SUM(attribut_1) 
FROM MaTable
    WHERE <condition(s)>;
\end{minted}
\end{sql}

Existent aussi les fonctions \mintinline{sql}{MIN, MAX, AVG} (min, max et moyenne).

\section*{Mise à jour et suppressions}

Supprimer des valeurs d'une table suivant une condition

\begin{sql}
    \begin{minted}{sql}
DELETE FROM MaTable
WHERE <conditions>;
\end{minted}
\end{sql}

Mettre à jour les valeurs d'une table suivant une condition
\begin{sql}
    \begin{minted}{sql}
UPDATE MaTable
SET attribut_1 = valeur_1
WHERE condition;
\end{minted}
\end{sql}

\section*{Jointures}

Sélectionner dans plusieurs tables liées par une référence
\begin{sql}
    \begin{minted}{sql}
SELECT Table_1.attribut_1, Table_2.attribut_3
FROM Table_1
JOIN Table_2 ON Table_1.attribut_1 = Table_2.attribut_de_reference

-- attribut_de_reference est une clé étrangère de Table_2 qui fait référence à la clé primaire de Table_1
\end{minted}
\end{sql}

\mintinline{SQL}{ON DELETE CASCADE} et \mintinline{SQL}{ON UPDATE CASCADE} ne sont pas indispensables, au contraire : elles « affaiblissent » les contraintes logiques de référence.

\end{document}
