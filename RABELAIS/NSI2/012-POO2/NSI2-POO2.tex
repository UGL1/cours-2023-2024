\documentclass[a4paper,10pt,cours,firamath]{nsi}
\setcounter{chapter}{9}
\begin{document}

\chapter{POO - partie 2}
\introduction{Quel est l'objet de ce chapitre ?}
\begin{aretenir}
	\begin{itemize}
		\item 	Pour plus d'encapsulation et de modularité, on peut « cacher » les attributs des instances et, à la place, implémenter des \textit{getters} et des \textit{setters}.
		\item 	De nombreuses méthodes spéciales, les \textit{dunders} peuvent être redéfinies pour un code encore plus flexible.
	\end{itemize}
\end{aretenir}
\section{Les bonnes pratiques d'encapsulation}
Considérons la classe suivante

\inputminted{python}{scripts/classe1.py}

Elle modélise une personne avec un âge, un prénom (\textit{name}) et un nom de famille \textit{surname}. Tel qu'écrite on s'en servira ainsi :
\begin{pyc}
	\begin{minted}{python}
>>> p = person(30,'Louise','Dupont')
>>> print(p.surname) # pour afficher le nom de la dame
Dupont
>>> p.surname = 'Durant' # madame s'est mariée
\end{minted}
\end{pyc}
Cela fonctionne très bien mais va à l'encontre des règles d'encapsulation et de modularité. On va donc créer des attributs privés commençant par «\texttt{\_}» et pour chacun d'entre eux 
\begin{itemize}
	\item 	une méthode appelée \textit{accesseur} (getter en Anglais) qui permet d'accéder à la valeur de l'attribut;
	\item 	une méthode appelée \textit{mutateur} (setter en Anglais) pour changer la valeur de l'attribut;
\end{itemize}
\inputminted{python}{scripts/classe2.py}

C'est mieux du point de vue de l'encapsulation : les attributs de l'objet restent cachés mais on peut les voir et les modifier \textit{via} des méthodes.\\
C'est mieux du point de vue de la modularité : si on veut changer les attributs (ou autre chose, pour une raison ou une autre) dans la classe \texttt{Person}, on peut garder les \textit{getters} et les \textit{setters}.
\begin{remarque}[]
	Les \textit{getters} et les \textit{setters} font partie de l'interface d'une classe.
\end{remarque}

\begin{pyc}
	\begin{minted}{python}
>>> p = person(30,'Louise','Dupont')
>>> print(p.get_surname()) # pour afficher le nom de la dame
Dupont
>>> p.set_surname('Durant') # madame s'est mariée
\end{minted}
\end{pyc}

\section{Les autres méthodes dunder de Python}

À part \mintinline{python}{__init__}, il existe une multitude de fonctions \textit{dunder}.

\subsection{La méthode \texttt{\_\_str\_\_}}
Reprenons la classe \texttt{Person}, imaginons qu'on l'a enregistrée dans un module appelé \texttt{person.py}
\begin{pyc}
	\begin{minted}{python}
>>> print(p)
<person.Person object at 0x7fdd9d8866d0>
\end{minted}
\end{pyc}
Pas très parlant, n'est-ce -pas ? Il n'y a qu'à redéfinir la méthode \mintinline{python}{__str__} dans la classe \texttt{Person}
\begin{pyc}
	\begin{minted}{python}
	def __str__(self):
		return "Name : " + self._name + ", Surname : " + self._surname + ", Age : " + str(self._age)
\end{minted}
\end{pyc}
Ce qui nous donne ensuite 
\begin{pyc}
	\begin{minted}{python}
>>> print(p)
Name : Louise, Surname : Durant, Age : 30
\end{minted}
\end{pyc}
Et c'est bien mieux !


\subsection{La méthode \texttt{\_\_eq\_\_}}

Ce morceau de code nous navre :

\begin{pyc}
	\begin{minted}{python}
>>> p1 = Person(10,'Tom','Dupont')
>>> p2 = Person(10,'Tom','Dupont')
>>> print(p1==p2)
False
\end{minted}
\end{pyc}
C'est parce que deux instances différentes d'une même classe sont stockées à des endroits différents de la mémoire, comme on peut le voir ainsi :
\begin{pyc}
	\begin{minted}{python}
>>> print(id(p1))
    140411289495248
>>> print(id(p2))
    140411291729200
\end{minted}
\end{pyc}
Mais heureusement on peut redéfinir la méthode \mintinline{python}{__eq__} de la classe \texttt{Person} :
\begin{pyc}
	\begin{minted}{python}
	def __eq__(self, other):
		return self._age == other._age and self._name == other._name and self._surname == other._surname
\end{minted}
\end{pyc}
Et ainsi
\begin{pyc}
	\begin{minted}{python}
>>> print(p1==p2)
True
\end{minted}
\end{pyc}
Ouf, tout rentre dans l'ordre.
\begin{remarque}[]
	La méthode \mintinline{python}{__eq__} prend évidemment \texttt{self} en paramètre, et un deuxième appelé \texttt{other} qui est censé être la deuxième instance de la classe \texttt{Person} avec laquelle on veut comparer la première.
\end{remarque}
\subsection*{Les dunder de conteneurs}
Ils permettent d'utiliser \texttt{len}, les notations avec crochets, et d'itérer sur un objet :

\inputminted{python}{scripts/examplegetitem.py}

L'exemple précédent permet de simuler une mémoire à 10 cases, vides au départ mais que l'on peut remplir comme on veut. Ce qui est intéressant c'est que l'objet garde en mémoire le nombre de fois où une case a été changée. \texttt{len} donne le nombre de cases \textit{non vides} et quand on itère sur l'objet on n'itère que sur ses cases non-vides.

\begin{pyc}
	\begin{minted}{python}
>>> a = MyList()
>>> len(a)
0
>>> a[1]=2
>>> a[3]=4
>>> len(a)
2
>>> print(10 in a)
False
>>> for x in a:
>>> ...     print(x)
>>> ...
2
4
\end{minted}
\end{pyc}

\subsection{D'autres dunders pour d'autres opérateurs}
Voici un récapitulatif des dunders les plus utiles


\begin{center}
	\tabstyle[UGLiBlue]
	\begin{tabular}{|c|c|}
		
	\ccell Dunder                   & \ccell opérateur \\
		\hline
		\mintinline{python}{__lt__(self, other)}       & <                               \\\hline
		\mintinline{python}{__le__(self, other)}       & <=                              \\\hline
		\mintinline{python}{__ne__(self, other)}       & !=                              \\\hline
		\mintinline{python}{__gt__(self, other)}       & >                               \\\hline
		\mintinline{python}{__ge__(self, other)}       & >=                              \\\hline
		\mintinline{python}{__add__(self, other)}      & +                               \\\hline
		\mintinline{python}{__sub__(self, other)}      & –                               \\\hline
		\mintinline{python}{__mul__(self, other)}      & *                               \\\hline
		\mintinline{python}{__truediv__(self, other)}  & /                               \\\hline
		\mintinline{python}{__floordiv__(self, other)} & //                              \\\hline
		\mintinline{python}{__contains__(self, item)}  & \mintinline{python}{in}         \\\hline
	\end{tabular}
\end{center}
Par exemple, lorsqu'on implémente \mintinline{python}{__add__} pour une classe, alors on peut écrire \mintinline{python}{c = a + b}, pour \mintinline{python}{a} et \mintinline{python}{b} instances de cette classe.
\begin{definition}[ : surcharge]
	Lorsqu'on attribue un sens supplémentaire à un opérateur pour une nouvelle classe, on dit qu'on \textit{surcharge} cet opérateur. Les dunders précédents servent donc à surcharger.
\end{definition}

\section{Exercices}

\begin{exercice}[ : fractions]
	Reprendre le module \mintinline{python}{fractions_custom} des chapitres précédents et le reprogrammer « objet » :
	\begin{itemize}
		\item 	la classe \mintinline{python}{Fraction} implémente les fractions.
		\item 	on implémentera les dunders suivants :
		      \begin{itemize}
			      \item 	\mintinline{python}{__eq__} (deux fractions sont égales ssi les produits en croix sont égaux);
			      \item 	\mintinline{python}{__lt__} et \mintinline{python}{__le__};
			      \item 	\mintinline{python}{__add__}, \mintinline{python}{__sub__}, \mintinline{python}{__mul__} et  \mintinline{python}{__truediv__}.
		      \end{itemize}
		      On se concentrera sur la programmation objet en priorité, on peut laisser la programmation défensive de côté.
	\end{itemize}
	\textbf{On devra pouvoir écrire :}
	\begin{minted}{python}
>>> f1 = Fraction(2, 7)
>>> f2 = Fraction(3, 5)
>>> f1 == f2
False
>>> f1 < f2
True
>>> (f1+f2)/(f1*(f1-f2))
-217/22
\end{minted}
	
\end{exercice}

\begin{exercice}[]
	
	\begin{enumerate}
		\item 	Créer une classe \mintinline{python}{Rectangle} dont chaque instance possède (au moins) des attributs \mintinline{python}{top}, \mintinline{python}{left}, \mintinline{python}{width} et \mintinline{python}{height}.
		\item	Implémenter \mintinline{python}{__str__} pour afficher « proprement» un objet de la classe.
		\item   Implémenter \mintinline{python}{__contains__} pour pouvoir tester si un point (\mintinline{python}{tuple} composé de deux \mintinline{python}{float}) se situe dans un rectangle ou non.
		\item 	Implémenter \mintinline{python}{__add__} pour que la somme de 2 rectangles soit le plus petit rectangle contenant les 2 rectangles.
		\item 	Implémenter \mintinline{python}{__mul__} pour que la somme de 2 rectangles soit le plus grand rectangle contenu dans les 2 rectangles (on fera attention au cas ou les rectangles sont disjoints).
	\end{enumerate}
	\textbf{On devra pouvoir écrire :}
	\begin{minted}{python}
>>> r1 = Rectangle(10,20,100,200)
>>> (3,4) in r1
False
>>> (12,100) in r1
True
>>> r2 = Rectangle(30,40,50,300)
>>> r1 + r2
Rectangle : (10,20,100,320)
>>> r1 * r2
Rectangle (30,40,50,180)
\end{minted}
\end{exercice}


\end{document}
