\documentclass[10pt]{nsibeamer}

\title{Programmation Objet - 2\eme partie}
\subtitle{Chapitre 10}
\author{NSI2}

\begin{document}
\maketitle
\section{Les bonnes pratiques d'encapsulation}

\begin{frame}[fragile]{Définition de la classe}
	\inputminted{python}{scripts/classe1.py}
	Cette classe modélise une personne avec un âge, un prénom (\textit{name}) et un nom de famille \textit{surname}.
\end{frame}

\begin{frame}[fragile]{Utilisation de la classe}

\begin{minted}{python}
>>> p = person(30,'Louise','Dupont')
>>> print(p.surname) # pour afficher le nom de la dame
Dupont
>>> p.surname = 'Durant' # madame s'est mariée
\end{minted}
\end{frame}
\begin{frame}{Problème}
	Cela fonctionne très bien mais va à l'encontre des règles d'encapsulation et de modularité : on ne devrait pas pouvoir modifier \alert{directement} l'attribut \texttt{surname} d'une instance.
\end{frame}
\begin{frame}{Solution}
On va donc créer des attributs privés commençant par «\texttt{\_}» et pour chacun d'entre eux 
\begin{enumerate}[--]
	\item 	une méthode appelée \alert{accesseur} (getter en Anglais) qui permet d'accéder à la valeur de l'attribut;
	\item 	une méthode appelée \alert{mutateur} (setter en Anglais) pour changer la valeur de l'attribut.
\end{enumerate}
\end{frame}
\begin{frame}[fragile]{Code Python}
\inputminted[fontsize=\scriptsize]{python}{scripts/classe2.py}
\end{frame}
\begin{frame}{Pourquoi c'est mieux}
Du point de vue de l'\alert{encapsulation} : les attributs de l'objet restent cachés mais on peut les voir et les modifier \textit{via} des méthodes.\\

Du point de vue de la \alert{modularité} : si on veut changer les attributs (ou autre chose, pour une raison ou une autre) dans la classe \texttt{Person}, on peut garder les \textit{getters} et les \textit{setters}.


\begin{block}{Remarque}
		Les \textit{getters} et les \textit{setters} font partie de l'interface d'une classe.
\end{block}	
\end{frame}
\begin{frame}[fragile]{Utilisation}

\begin{minted}{python}
>>> p = person(30,'Louise','Dupont')
>>> print(p.get_surname()) # pour afficher le nom de la dame
Dupont
>>> p.set_surname('Durant') # madame s'est mariée
\end{minted}
\end{frame}
\section{Les autres méthodes dunder de Python}
\begin{frame}{La méthode \texttt{\_\_str\_\_}}
	On a déjà vu cette méthode, il y en a d'autres.
\end{frame}
\begin{frame}[fragile]{La méthode \texttt{\_\_equ\_\_}}
	
Ce morceau de code nous navre :
	
\begin{minted}{python}
>>> p1 = Person(10,'Tom','Dupont')
>>> p2 = Person(10,'Tom','Dupont')
>>> print(p1==p2)
False
\end{minted}
C'est parce que deux instances différentes d'une même classe sont stockées à des endroits différents de la mémoire, comme on peut le voir ainsi :
\begin{minted}{python}
>>> print(id(p1))
140411289495248
>>> print(id(p2))
140411291729200
\end{minted}
\end{frame}
\begin{frame}[fragile]{solution}
On peut redéfinir la méthode \mintinline{python}{__eq__} de la classe \texttt{Person} :
\begin{minted}{python}
def __eq__(self, other):
    return self._age == other._age 
           and self._name == other._name 
           and self._surname == other._surname
\end{minted}
Et ainsi
\begin{minted}{python}
>>> print(p1==p2)
True
\end{minted}
Ouf, tout rentre dans l'ordre !
\end{frame}
\begin{frame}[fragile]{Remarque}
	La méthode \mintinline{python}{__eq__} prend évidemment \texttt{self} en paramètre, et un deuxième appelé \texttt{other} qui est censé être la deuxième instance de la classe \texttt{Person} avec laquelle on veut comparer la première.

\end{frame}
\begin{frame}{Dunders de conteneurs}
\inputminted[fontsize=\scriptsize]{python}{scripts/examplegetitem.py}
\end{frame}
\begin{frame}{Explications}
L'exemple précédent permet de simuler une mémoire à 10 cases, vides au départ mais que l'on peut remplir comme on veut. Ce qui est intéressant c'est que l'objet garde en mémoire le nombre de fois où une case a été changée. \texttt{len} donne le nombre de cases \textit{non vides} et quand on itère sur l'objet on n'itère que sur ses cases non-vides.
\end{frame}
\begin{frame}[fragile]{Utilisation}
\begin{minted}{python}
>>> a = MyList()
>>> len(a)
0
>>> a[1]=2
>>> a[3]=4
>>> len(a)
2
>>> for x in a:
>>> ...     print(x)
>>> ...
2
4
\end{minted}
\end{frame}
\begin{frame}{D'autres dunders utiles}
\begin{center}
    \begin{tabular}{|c|c|}
        \hline\rowcolor{UGLiBlue!50!black}
        \color{white}\textbf{Dunder}&\color{white}\textbf{opérateur} \\
        \hline
        \mintinline{python}{__lt__(self, other)}  &< \\\hline
        \mintinline{python}{__le__(self, other)} & <= \\\hline
        
        \mintinline{python}{__ne__(self, other)} &!= \\\hline
        \mintinline{python}{__gt__(self, other)} &>\\\hline
        \mintinline{python}{__ge__(self, other)} &>=\\\hline
        \mintinline{python}{__add__(self, other)} &+\\\hline
        \mintinline{python}{__sub__(self, other)} & –  \\\hline
        \mintinline{python}{__mul__(self, other)} &* \\\hline
        \mintinline{python}{__truediv__(self, other)} & / \\\hline
        \mintinline{python}{__floordiv__(self, other)} &//\\\hline
    \end{tabular}
\end{center}
\end{frame}
\begin{frame}{Exemple et surcharge}
Lorsqu'on implémente \mintinline{python}{__add__} pour une classe, alors on peut écrire \mintinline{python}{c = a + b}, pour \mintinline{python}{a} et \mintinline{python}{b} instances de cette classe.
\begin{block}{Définition}
    Lorsqu'on attribue un sens supplémentaire à un opérateur pour une nouvelle classe, on dit qu'on \alert{surcharge} cet opérateur. Les dunders précédents servent donc à surcharger.
\end{block}
\end{frame}
\end{document}