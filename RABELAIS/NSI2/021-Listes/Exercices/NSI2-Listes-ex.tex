\documentclass[a4paper,12pt,french]{article}
\usepackage[margin=2cm]{geometry}
\usepackage[thinfonts]{uglix2}
\nouveaustyle

\begin{document}
\titre{Listes - Exercices}{NSI2}{2021}

\begin{exercice}
Ouvrir le fichier \texttt{LinkedList\_stump.py} (\textit{stump} veut dire moignon en anglais) et regarder en détail l'implémentation de la structure de liste chaînée vue en cours.\\
Pour l'utiliser inclure \mintinline{python}{from LinkedList_stump import LinkedList} dans votre code.
    \'Ecrire une fonction \mintinline{python}{liste_N} qui
    \begin{enumerate}[--]
        \item en entrée prend un \mintinline{python}{int} n;
        \item renvoie la liste chaînée composée des éléments 1, 2, ..., n.
    \end{enumerate}
\end{exercice}
\begin{exercice}[ : Terminer l'implémentation objet]
Le fichier \texttt{LinkedList\_stump.py}  est incomplet. Tu vas devoir le compléter. Une fois ceci fait tu pourras lui donner le nom \texttt{LinkedList.py}.
\begin{enumerate}
    \item À quoi sert la méthode \mintinline{python}{__getitem__} ?
    \item En s'inspirant de la méthode précédente, écrire la méthode \mintinline{python}{find}.
    \item Coder la méthode \mintinline{python}{extend} qui
          \begin{enumerate}[--]
              \item en entrée prend une deuxième liste chaînée;
              \item ajoute tous les éléments de cette liste à la première, dans l'ordre
          \end{enumerate}
            Par exemple, si \mintinline{python}{L} vaut \texttt{<1, 2, 4>} et si \mintinline{python}{L2} vaut \texttt{<6, 5>} alors \mintinline{python}{L.extend(L2)} vaut \texttt{<1, 2, 4, 6, 5>}.
     \item Coder la méthode \mintinline{python}{__eq__} qui
     \begin{enumerate}[--]
         \item en entrée prend une seconde liste chaînée;
         \item renvoie \mintinline{python}{True} si les deux listes ont les mêmes éléments aux mêmes places et \mintinline{python}{False} sinon.
     \end{enumerate}
\end{enumerate}
\end{exercice}

\begin{exercice}[ : Algorithmes récursifs]
\'Ecrire un script \texttt{recursive\_functions.py} qui importe \texttt{LinkedList.py} et
\begin{enumerate}
    \item \'Ecrire la fonction \mintinline{python}{recursive_length} qui
    \begin{enumerate}[--]
        \item en entrée prend une liste chainée;
        \item en sortie renvoie un \mintinline{python}{int} qui est sa longueur en procédant de manière récursive.
    \end{enumerate}
    \item \'Ecrire la fonction \mintinline{python}{recursive_find} qui
    \begin{enumerate}[--]
    \item en entrée prend une liste chainée et un élément;
    \item en sortie renvoie un \mintinline{python}{int} qui est la position de l'élément s'il figure dans la liste, et -1 sinon.\\
            Cette fonction procède récursivement.
\end{enumerate}
\textbf{Conseil :} Pour chacune de ces fonctions, écrire une sous-fonction récursive qui procède sur des données de type \mintinline{python}{Cell} et qui sera appelée par la fonction principale.
\end{enumerate}
\end{exercice}
\end{document}