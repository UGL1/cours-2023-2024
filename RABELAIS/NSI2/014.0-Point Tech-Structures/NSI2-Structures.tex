\documentclass[10pt]{beamer}
\usepackage{uglibeamer2}
\title{Point technique : \\ Structures de données}
\subtitle{Chapitre 11}
\author{T$^{\text{ale}}$ NSI}


\begin{document}

\maketitle

\begin{frame}{Qu'est-ce que c'est ?}\pause
	C'est une manière d'organiser les données.\pause
\end{frame}

\begin{frame}{Pourquoi faire ?}\pause
	Pour les utiliser \pause
	\begin{itemize}
		\item 	dans certaines situations elles sont plus appropriées car plus rapides ;\pause
		\item 	parfois elles se présentent naturellement comme la structure la plus simple à utiliser ;\pause
		\item 	dans le cadre d'autres algorithmes.
	\end{itemize}
\end{frame}

\begin{frame}{Les aspects étudiés}\pause
	\begin{itemize}
		\item 	Différentes implémentations de la structure (au moins une).\pause
		\item 	Opérations de bases sur la structure (ajout/suppression d'un élément, combinaison de structures...).\pause
		\item 	Complexité en temps ou en mémoire des opérations de base sur la structure ;\pause
		\item 	Quand choisir cette structure.
	\end{itemize}
\end{frame}

\begin{frame}{Les structures étudiées}\pause
	\begin{itemize}
	\item 	Piles\pause
	\item 	Files\pause
	\item 	Listes\pause
	\item 	Arbres (binaires et généraux)\pause
	\item 	Arborescences\pause
	\item 	Graphes	
\end{itemize}
\end{frame}
\begin{frame}
	\begin{center}
		\Huge À l'attaque !\\
		\includegraphics[width=5cm]{img/bytar}	
	\end{center}
\end{frame}
\end{document}