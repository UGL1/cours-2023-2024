\documentclass[a4paper,12pt,eval]{nsi}
\pagestyle{empty}
\titre{Interrogation 01}
\classe{NSI2}
\begin{document}
\maketitle


\question Explique \textit{brièvement} (la réponse doit rentrer dans le cadre) et sans donner d'exemple ce qu'est une fonction récursive.\bareme{2pts}\\

% une fonction récursive peut renvoyer un résultat dans un cas d'arrêt, sinon elle s'appelle elle-même une ou plusieurs fois avec un de ses paramètres qui devient plus petit. Une pile d'appels est utilisée
\carreauxseyes{16.8}{6.4}\\

\question \'Ecris ici une version \textsc{Python} de la fonction récursive \pythoninline{fibo} qui
\begin{itemize}
	\item 	en entrée prend un entier positif \pythoninline{n};
	\item 	en sortie renvoie la valeur de $F_n$ définie par $$F_n=\begin{cases}
		1 & \mbox{si } n=0\mbox{ ou }n=1\\
		F_{n-1}+F_{n-2} &\mbox{sinon}
	\end{cases}$$
    \bareme{2pts}
\end{itemize}

\carreauxseyes{16.8}{6.4}\\

\question Voici une première fonction récursive :
\inputminted{python}{scripts/mystere1.py}
Calcule \pythoninline{mystery1(0)},  \pythoninline{mystery1(1)} et \pythoninline{mystery1(2)}. \bareme{1pt}\\

\carreauxseyes{16.8}{11.2}\\

Explique ce que fait cette fonction.\bareme{2pts}\\

\carreauxseyes{16.8}{4}\\
\newpage
\question En voici une deuxième :
\inputminted{python}{scripts/mystere2.py}
Explique ce que renvoient \pythoninline{mystery2([1])} et \pythoninline{mystery2([7, 3, 5])}.\bareme{1pt}\\

\carreauxseyes{16.8}{9.6}\\

Explique ce que fait cette fonction.\bareme{2pts}\\

\carreauxseyes{16.8}{3.2}
\end{document}
