\documentclass[a4paper,12pt,french]{article}
\usepackage[margin=2cm]{geometry}
\usepackage[thinfonts]{uglix2}

\begin{document}
\titre{ABR - Exercices}{NSI2}{2022} 

\begin{exercice}
Donner tous les ABR formé des n\oe uds 1, 2 et 3.
\end{exercice}


\begin{exercice}
Quel parcours d'un ABR donne la liste de éléments dans l'ordre croissant ?
\end{exercice}

\begin{exercice}
Créer une classe \mintinline{python}{NodeBST} dans un fichier \mintinline{python}{node_bst.py}. BST est un sigle signifiant de \textit{binary search tree}.\\
Elle se compose de tout ce qu'il y a dans la classe \mintinline{python}{Node} avec en plus :
\begin{enumerate}[--]
	\item une méthode \mintinline{python}{add_value}, pour ajouter un élément comme vu dans le cours;
    \item la méthode spéciale \mintinline{python}{___contains__} qui permet de vérifier si un élément est présent ou non dans l'arbre à l'aide du mot-clé \mintinline{python}{in}
\end{enumerate}
\end{exercice}

\begin{exercice}
Où se trouve le minimum des n\oe uds ABR ? Où est le maximum ?\\
En déduire deux méthodes \mintinline{python}{smallest} et \mintinline{python}{greatest} pour la classe \mintinline{python}{NodeBST}.
\end{exercice}

\begin{exercice} [ hors programme]
Implementer la méthode \mintinline{python}{delete_value} qui enlève le n\oe ud contenant la valeur correspondante de l'ABR comme vu en cours.
\end{exercice}

\end{document}