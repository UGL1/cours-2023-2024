\documentclass[10pt, a4paper,article,landscape]{nsi}
\pagestyle{empty}
\geometry{margin=2cm}
\usepackage{tabularx}
\newcommand{\trait}{{\color{UGLiBlue}\hrule}}
\newcommand{\tit}[1]{{\large\color{UGLiBlue}\titlefont #1}\\}
\setminted{bgcolor=UGLiBlue!7}
\classe{SIO1, novembre 2023}
\titre{MÉMENTO PYTHON}
\begin{document}
\maketitle
\begin{multicols}{3}
    \tit{Principaux types}
    \mintinline{python}{int} : entier \\
    \mintinline{python}{float} : flottant \\
    \mintinline{python}{bool} : booléen \\
    \mintinline{python}{str} : chaîne de caractères \\
    \mintinline{python}{list} : liste\\
    \mintinline{python}{range} : plage d'entiers\\

    \tit{Affectation}
    \mintinline{python}{a = 2} : affectation simple \\
    \mintinline{python}{a, b = 2, 3} : affectations multiples \\
    \mintinline{python}{a += 1} : équivaut à \mintinline{python}{a = a + 1} \\
    \mintinline{python}{a *= 2} : équivaut à \mintinline{python}{a = a * 2}\\

    \tit{Transtypage}
    \mintinline{python}{int("12")} vaut \mintinline{python}{12}\\
    \mintinline{python}{str(12)} vaut \mintinline{python}{"12"}\\
    \mintinline{python}{float("12.5")} vaut \mintinline{python}{12.5}\\

    \tit{Opérations}
    quotient : \mintinline{python}{a = 11 // 4 # a vaut 2}\\
    reste : \mintinline{python}{a = 11 % 4 # a vaut 1}\\
    division : \mintinline{python}{a = 11 / 4 # a vaut 2.75}\\
    puissance : \mintinline{python}{a = 2 ** 10 # a vaut 1024}\\

    \tit{Entrées / sorties}
    \mintinline{python}{print(f"les valeurs sont {a} et {b}")}\\
    affiche la valeur de \mintinline{python}{a} et de \mintinline{python}{b}\\
    \mintinline{python}{a = input("Entrez une valeur")}\\stocke une entrée utilisateur de type \mintinline{python}{str} dans \mintinline{python}{a}\\

    \tit{Commun à str et list}
    \mintinline{python}{len(a)} : longueur de l'objet\\
    \mintinline{python}{a[i]} : élément d'indice \mintinline{python}{i} de l'objet\\
    \mintinline{python}{a[-1]} : dernier élément de l'objet\\

    \tit{Tests}

    \tabstyle[UGLiBlue]
    \begin{tabularx}{\columnwidth}{X|X}
        \ccell Opérateur            & \ccell Signification  \\
        \mintinline{python}{or}     & ou logique            \\
        \mintinline{python}{and}    & et logique            \\
        \mintinline{python}{not}    & négation logique      \\
        \mintinline{python}{<}      & strictement inférieur \\
        \mintinline{python}{<=}     & inférieur ou égal     \\
        \mintinline{python}{>}      & strictement supérieur \\
        \mintinline{python}{>=}     & supérieur ou égal     \\
        \mintinline{python}{==}     & égal                  \\
        \mintinline{python}{!=}     & différent             \\
        \mintinline{python}{in}     & appartient à          \\
        \mintinline{python}{not in} & n'appartient pas à    \\
    \end{tabularx}
    
    \columnbreak
    
    \tit{Test \mintinline{python}{if ... else ...} classique}
    \begin{minted}{python}
            if <condition>:
                # bloc conditionnel 1
            else: # facultatif
                # bloc conditionnel 2
    \end{minted}
    

    
    \tit{Test \mintinline{python}{if ... elif ... else ...}}
    \begin{minted}{python}
            if <condition 1>:
                # bloc conditionnel 1
            elif <condition 2>: 
                # bloc conditionnel 2
            # autant de elif que nécessaire
            else: # facultatif
                # bloc conditionnel final
    \end{minted}


    \tit{Boucle \mintinline{python}{while}}\\[-3.5em]
    \begin{minted}{python}
        while <condition>:
            # bloc conditionnel
        # en sortie de boucle la condition
        # n'est plus vérifiée
    \end{minted}

    \columnbreak

    \tit{Boucle \mintinline{python}{for ... in objet} avec \mintinline{python}{objet} de type \mintinline{python}{str} ou \mintinline{python}{list}}\\[-3em]
    \begin{minted}{python}
        for x in objet:
            # x prend successivement toutes 
            # les valeurs des éléments de 
            # l'objet
    \end{minted}
    \tit{Boucle \mintinline{python}{for ... in range()}}
    \begin{minted}{python}
        for i in range(...):
            # i prend successivement toutes 
            # le valeurs du range
    \end{minted}

    \tit{Utilisation de \mintinline{python}{range}}

    \mintinline{python}{range(<debut>,fin,<increment>)}\\
    par défaut \mintinline{python}{debut=0} et \mintinline{python}{increment=1}\\

    dans \mintinline{python}{range(<debut>,fin,<increment>)}:\\
    on part de la valeur de début, appelons la \mintinline{python}{val}\\
    tant que \mintinline{python}{val < fin}:\\
    $\qquad$ - ajouter \mintinline{python}{val} à la plage\\
    $\qquad$ - ajouter \mintinline{python}{increment} à \mintinline{python}{val}\\

    \columnbreak

    \tit{Utilisation de \mintinline{python}{list}}

    \mintinline{python}{a = [] # liste vide}\\
    \mintinline{python}{a.append(objet) # ajoute objet à a}\\
    \mintinline{python}{a.remove(objet) # retire objet à a}\\
    \mintinline{python}{a.insert(i,objet) # insère objet à}\\
    \mintinline{python}{# l'indice i}\\
    \mintinline{python}{del a[i] # retire l'élément d'indice i}\\



    
\end{multicols}


\end{document}