\chapter{Opérations en base 2 et 16}
\section{Additions}

\tabulardefault
On pose l'opération à la main : c'est la même chose qu'en base 10.
\subsection{En base 2}
La seule différence avec la base 10 c'est que deux 1 donnent $(2)_{10}$ donc $(10)_2$, donc un zéro et une retenue de 1.
Quand il y a deux 1 et une retenue de 1 en plus, cela donne $(3)_{10}$ donc $(11)_2$, donc un 1 et une retenue de 1.

\begin{exemple}[]
	\begin{center}
		\begin{tabular}{ccccccccc}
			  &   &   &   & $_1$ & $_1$ &   &   &   \\
			  & 1 & 1 & 0 & 1    & 0    & 1 & 0 & 1 \\
			+ &   &   &   & 1    & 1    & 1 & 0 & 0 \\
			\hline
			= & 1 & 1 & 1 & 1    & 0    & 0 & 0 & 1 \\
		\end{tabular} \\[2em]
	\end{center}
\end{exemple}

\subsection{En base 16}

C'est encore la même chose. Il faut bien se souvenir de la valeur de A, B, C, D, E et F.\\
Ajouter 8 et 3 ne provoque pas de retenue puisque $8+3=11$ et que $11$ est $B$ en base 16.\\
Dès que l'addition de 2 chiffres dépasse 15, il y aura une retenue : par exemple 9 et A donnent $(19)_{10}$, donc $(13)_{16}$. Ainsi on note 3 et une retenue de 1.


\begin{exemple}
	\begin{center}
		\begin{tabular}{ccccc}
			  &   & $_1$ &   &   \\
			  & 2 & 4    & 9 & 8 \\
			+ & 1 & 7    & A & 3 \\
			\hline
			= & 3 & C    & 3 & B \\
		\end{tabular}
	\end{center}
\end{exemple}

\section{Multiplications par 2 en binaire}

\begin{propriete}[]
	\begin{itemize}
		\item 	Multiplier un nombre écrit en binaire par 2 revient à décaler la virgule d'un cran vers la droite (ou ajouter un zéro à droite si le nombre est entier).
		\item 	Diviser un nombre écrit en binaire par 2 revient à décaler la virgule d'un cran vers la gauche.
	\end{itemize}
\end{propriete}

\begin{exemple}[s]
	\begin{itemize}
		\item 	$(1\ 0101)_2\times (2)_{10}=(10\ 1010)_2$
		\item 	$(11,01)_2\times (16)_{10}=(11\ 0100)_2$
		\item 	$(1\ 1101)_2\div (2)_{10}=(1110,1)_2$
		\item 	$(101)_2\div (32)_{10}=(0,0010\ 1)_2$
	\end{itemize}
\end{exemple}

\begin{remarque}
	Rappelons-nous que $(2)_{10}=(10)_2$, et que plus généralement $2^n$ s'écrit en binaire comme «  un 1 suivi de $n$ zéros » .
\end{remarque}



\begin{exercice}[]
	Ajouter $(1101\ 1011)_2$ et $(0011\ 0110)_2$ en posant l'opération.
\end{exercice}

\begin{exercice}[]
	On pose  $a=(1001)_2$, $b=(0010\ 1000)_2$ et $c=(0001\ 0111)_2$.\\

	Calculer $a +b+c$ en posant les opérations (on peut faire des étapes ou bien tout calculer en une fois).

\end{exercice}

\begin{exercice}[]
	On reprend les données de l'exercice précédent.	Calculer $a\times 4+b\div 8 +c$.
\end{exercice}

\begin{exercice}[]
	Calculer à la main $(AF3)_{16}+(8AD)_{16}$.
\end{exercice}

\begin{exercice}[]
	Calculer à la main $(123)_{16}+(456)_{16}+(789)_{16}$ en posant une seule opérations si possible.
\end{exercice}

