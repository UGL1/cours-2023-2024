\chapter{Matrices}
\section{Notion de matrice}
\begin{definition}[ : matrice]
	Une matrice $A$ peut être vue comme « un tableau de nombres ».\\
	Supposons qu'elle comporte n lignes et p colonnes (n et p sont des entiers plus grands que 1), on la note ainsi
	$$A = \begin{matrice}
			a_{11}      & a_{12}&\cdots & a_{1p} \\
			a_{21}  & a_{22}& \cdots & a_{2p} \\
			\vdots 	& \vdots & \ddots & \vdots \\
			a_{n1}  & a_{n2}    & \cdots & a_{np}
		\end{matrice}$$
	L'élément qui se situe à la i\eme ligne et à la j\eme colonne est noté $a_{ij}$. On l'appelle également \textit{coefficient}.\\
	
	\textbf{Attention} : les indices des lignes et des colonnes commencent à 1 (et non à zéro comme dans la plupart des langages informatiques).\\
	
	Pour résumer l'écriture précédente on écrit
	$$A=(a_{ij})_{\substack{1\leqslant i\leqslant n\\1\leqslant j\leqslant p}}$$
	
	On dit aussi que $A$ est une matrice $n\times p$. Si $n=p$ on dit que $A$ est une \textit{matrice carrée d'ordre $n$}.
\end{definition}

\begin{exemple}[s]
	\begin{itemize}
		\item 	$B =	\begin{matrice}
				      1      & 2 & 4 \\
				      -3  & 5 & 0
			      \end{matrice}$ est une matrice à 2 lignes et 3 colonnes. On a $b_{21}=-3$.
		\item 	$C =	\begin{matrice}
				      2,4      & 7 & 4 & -1 \\
				      -3  & 5 & 10,1 & 1 \\
				      0,01 & 3 & 12 & 100
			      \end{matrice}$ est une matrice à 3 lignes et 4 colonnes. On a $c_{33}=12$.
		\item 	$D =	\begin{matrice}
				      4      & 2\\
				      2  &8
			      \end{matrice}$ est une matrice carrée d'ordre 2.
		      
		      
	\end{itemize}
\end{exemple}

\begin{exercice}[]
	On considère 	$E =	\begin{matrice}
			-4      & 7,6 & 4 & -1 & 12 \\
			8 & -3  & 5,7 & 101 & 1 \\
			12 & 0,01 & 3 & 12 & 1
		\end{matrice}$.\\
	
	Donne les valeurs de $e_{12}$, $e_{21}$, $e_{35}$ et $e_{24}$.
\end{exercice}

Le script \textsc{Python} suivant permet de générer une matrice $n\times p$ avec des coefficients entiers aléatoires compris entre -100 et 100.

\begin{pyc}
	\begin{minted}{python}
		from random import randint

		n = int(input("Entrez le nombre de lignes : "))
		p = int(input("Entrez le nombre de colonnes : "))

		matrice = []  # une matrice est une liste de lignes

		for i in range(n):  # il y a n lignes
			ligne = []  # on construit une ligne vide
			for j in range(p):  # il y a p colonnes
				ligne.append(randint(-100, 100))  # on remplit la ligne aléatoirement
			matrice.append(ligne)  # on ajoute la ligne à la liste de lignes
	\end{minted}
\end{pyc}
\begin{exercice}[]
	\begin{enumerate}
		\item \'Ecris complètement la matrice suivante : $M=(m_{ij})_{\substack{1\leqslant i\leqslant 3\\1\leqslant j\leqslant 5}}$ où $m_{ij}=i$ si i=j et 0 sinon.
		\item 	\'Ecris complètement la matrice suivante : $M=(m_{ij})_{\substack{1\leqslant i\leqslant 4\\1\leqslant j\leqslant 5}}$ où $m_{ij}=0$ si i<j et 1 sinon.
		\item 	\'Ecris complètement la matrice suivante : $M=(m_{ij})_{\substack{1\leqslant i\leqslant 3\\1\leqslant j\leqslant 3}}$ où $m_{ij}=1$ si $i+j$ est pair et 0 sinon.
		\item \textsc{bonus} : écris des programmes \textsc{Python} qui génèrent ces matrices.
	\end{enumerate}
\end{exercice}

\begin{definition}[s : Matrices nulles et identités]
	\begin{itemize}
		\item 	Une matrice dont tous les coefficients sont nuls est dite \textit{nulle} (c'est « un tableau de zéros »);
		\item 	la matrice \textit{carrée d'ordre $n$} dont tous les éléments sont nuls sauf ceux de la \textit{diagonale} (c'est-à-dire ceux qui s'écrivent $a_{ii})$ qui valent 1 s'appelle \textit{la matrice identité d'ordre $n$} et se note $I_n$.
	\end{itemize}
\end{definition}

\begin{exemple}[]
	La matrice identité $I_3$ est
	$I_3= \begin{matrice}
			1&0&0\\
			0&1&0\\
			0&0&1
		\end{matrice}$.
\end{exemple}

\section{Opérations sur les matrices}

\begin{definition}[ : addition]
	Soient $A$ et $B$ deux matrices $n\times p$, on note $A+B$ la matrice $n\times p$ obtenu en ajoutant les coefficients correspondants de $A$ et de $B$ :
	
	$$\begin{matrice}
			a_{11}      & \cdots & a_{1p} \\
			
			\vdots 	& \ddots & \vdots \\
			a_{n1}      & \cdots & a_{np}
		\end{matrice} +\begin{matrice}
			b_{11}      & \cdots & b_{1p} \\
			
			\vdots 	& \ddots & \vdots \\
			b_{n1}      & \cdots & b_{np}
		\end{matrice}=
		\begin{matrice}
			a_{11} +b_{11}     & \cdots & a_{1p}+b_{1p} \\
			
			\vdots 	& \ddots & \vdots \\
			a_{n1}+n_{n1}      & \cdots & a_{np}+b_{np}
		\end{matrice}$$
\end{definition}

\begin{exemple}[]
	Prenons $A = \begin{matrice}
			3&1\\
			4&7\\
			-2&8
		\end{matrice}$ et $B = \begin{matrice}
			2&5\\
			1&-3\\
			-5&9
		\end{matrice}$, alors $A+B=\begin{matrice}
			5&6\\5&4\\-7&17
		\end{matrice}$.\end{exemple}
\begin{remarque}[]
	\textbf{Attention :} on ne peut ajouter deux matrices que si elles ont les mêmes dimensions (c'est-à-dire même nombre de lignes et même nombre de colonnes).
\end{remarque}
\begin{definition}[ : multiplication par un réel]
	Soient $A$ une matrice $n\times p$ et $k$ un nombre réel, on note $kA$ la matrice $n\times p$ obtenue en multipliant chaque coefficient de $A$ par $k$ :
	
	$$k\begin{matrice}
			a_{11}      & \cdots & a_{1p} \\
			\vdots 	& \ddots & \vdots \\
			a_{n1}      & \cdots & a_{np}
		\end{matrice}=\begin{matrice}
			k\times a_{11}      & \cdots & k\times a_{1p} \\
			\vdots 	& \ddots & \vdots \\
			k\times a_{n1}      & \cdots &k\times a_{np}
		\end{matrice}
	$$
\end{definition}

\begin{exemple}[]
	Prenons $A = \begin{matrice}
			8&3&1\\
			4&7&2\\
			-2&1&8
		\end{matrice}$ et $k=5$, alors on obtient que $5A = \begin{matrice}
			40&15&5\\
			20&35&10\\
			-10&5&40
		\end{matrice}$.\end{exemple}


La propriété suivante énonce quelques résultats utiles pour calculer.
\begin{propriete}[ : règles de calcul]
	Soient $A$, $B$ et $C$ trois matrices de mêmes dimensions et $k$ et $k'$ 2 réels.
	\begin{itemize}
		\item 	$A+B = B+A$
		\item 	$(A+B)+C = A+(B+C)$
		\item 	$k(A+B) = kA + kB$
		\item 	$(k+k')A = kA+k'A$
	\end{itemize}
\end{propriete}

\begin{exercice}[]
	On pose $A=\begin{matrice}
			1&2\\3&-4
		\end{matrice}$, $B=\begin{matrice}
			11&10\\-9&7
		\end{matrice}$ et $C=\begin{matrice}
			-3&-2\\5&-5
		\end{matrice}$.\\
	
	Montrer que $B-2A+3C$ est une matrice nulle.
\end{exercice}

\begin{definition}[ : multiplication de deux matrices]
	
	Soient $A$ une matrice $n\times p$ et $B$ une matrice $p\times q$ (le nombre de colonnes de la 1\ere est égal au nombre de lignes de la 2\eme) alors il est possible de définir la matrice $C=A\times B$, produit de $A$ par $B$.\\
	C'est une matrice $n\times q$ dont les coefficients sont ainsi :
	\begin{center}
		\begin{tabular}{cc}
			                                                                                                                                    & $\begin{matrice}
					                                                                                                                                       b_{11}   &\cdots   &\cellcolor{lightgray!50}\color{UGLiOrange}b_{1j}& \cdots & a_{1q} \\
					                                                                                                                                       \vdots 	& \ddots &\cellcolor{lightgray!50}\color{UGLiOrange}\cdots&\dots & \vdots \\
					                                                                                                                                       b_{k1} & \cdots & \cellcolor{lightgray!50}\color{UGLiOrange}b_{kj}&\cdots & b_{kq}\\
					                                                                                                                                       \vdots 	& \cdots &\cellcolor{lightgray!50}\color{UGLiOrange}\cdots&\ddots & \vdots \\
					                                                                                                                                       b_{p1}    &\cdots  &\cellcolor{lightgray!50}\color{UGLiOrange}b_{pj} & \cdots &b_{pq}
				                                                                                                                                       \end{matrice}$ \\
			
			$\begin{matrice}
					 a_{11}   &\cdots   &\cdots& \cdots & a_{1p} \\
					 \vdots 	& \ddots &\cdots&\dots & \vdots \\
					 \rowcolor{lightgray!50} \color{UGLiGreen}    a_{i1} & \color{UGLiGreen}\cdots &\color{UGLiGreen} a_{ik}&\color{UGLiGreen}\cdots & \color{UGLiGreen}a_{ip}\\
					 \rowcolor{white}\vdots 	& \cdots &\cdots&\ddots & \vdots \\
					 a_{n1}    &\cdots  & \cdots & \cdots &a_{np}
				 \end{matrice}$ & 
			$\begin{matrice}
					 c_{11}   &\cdots   &\cdots& \cdots & c_{1q} \\
					 \vdots 	& \ddots &\cdots&\dots & \vdots \\
					 \vdots & \cdots &\cellcolor{lightgray!50} \color{UGLiRed}c_{ij}&\cdots & \vdots\\
					 \vdots 	& \cdots &\cdots&\ddots & \vdots \\
					 c_{n1}    &\cdots  & \cdots & \cdots &c_{nq}
				 \end{matrice}$
		\end{tabular}
	\end{center}
	{\LARGE$$ {\color{UGLiRed}c_{ij}} = {\color{UGLiGreen}a_{i1}}\times {\color{UGLiOrange}b_{1j}}+ {\color{UGLiGreen}a_{i2}}\times {\color{UGLiOrange}b_{2j}}+\ldots+ {\color{UGLiGreen}a_{ip}}\times {\color{UGLiOrange}b_{pj}}$$}
\end{definition}

\begin{exemple}[]
	
	Prenons $A=\begin{matrice}
			1&3 & -2 \\5 &-3&4
		\end{matrice}$ et $B=\begin{matrice}
			-5 & 1 & 0 & 2\\2& 1 & -1 & -8\\3 &4 & 0 & 9
		\end{matrice}$ alors $A$ est une matrice $2\times 3$, $B$ est une matrice $3\times 4$ donc il est possible de définir la matrice $C=A\times B$, ce sera une matrice $2\times 4$.
	\begin{center}
		\begin{tabular}{cc}
			                                                                      & $\begin{matrice}
					                                                                         -5 & 1 & \color{UGLiOrange}0 & 2\\2& 1 & \color{UGLiOrange}-1 & -8\\3 &4 & \color{UGLiOrange}0 & 9
				                                                                         \end{matrice}$ \\
			$\begin{matrice}
					 1&3 & -2 \\\color{UGLiGreen}5 &\color{UGLiGreen}-3&\color{UGLiGreen}4
				 \end{matrice}$ & $\begin{matrice}
					                   -5&-4&-3&-40\\-19&18&\color{UGLiRed}3&70
				                   \end{matrice}$
		\end{tabular}
	\end{center}
	Par exemple, pour calculer $c_{33}$, on fait ${\color{UGLiGreen}5}\times {\color{UGLiOrange}0}+ {\color{UGLiGreen}(-3)}\times {\color{UGLiOrange}(-1)}+ {\color{UGLiGreen}4}\times {\color{UGLiOrange}0}={\color{red}3} $.
\end{exemple}

\begin{remarque}[]
	\textbf{Attention :}
	\begin{itemize}
		\item 	on ne peut multiplier $A$ par $B$ que si le nombre de colonnes de $A$ est égal au nombre de lignes de $B$;
		\item 	ce n'est pas parce qu'on peut calculer $A\times B$ qu'on peut calculer $B\times A$ : les matrices de l'exemple précédent ne permettent pas de calculer $B\times A$ car le nombre de colonnes de $B$ n'est pas égal au nombre de lignes de $A$:
		      \begin{center}
			      \begin{tabular}{cc}
				                                                                                                         & $\begin{matrice}
						                                                                                                            1&3 & -2 \\\color{UGLiGreen}5 &\color{UGLiGreen}-3&\color{UGLiGreen}4
					                                                                                                            \end{matrice}$ \\
				      $\begin{matrice}
						       -5 & 1 & \color{UGLiOrange}0 & 2\\2& 1 & \color{UGLiOrange}-1 & -8\\3 &4 & \color{UGLiOrange}0 & 9
					       \end{matrice}$ & Impossible
			      \end{tabular}
		      \end{center}
		\item pour pouvoir calculer $A\times B$ et $B\times A$ il faut que ces deux matrices soient carrées d'ordre $n$ et \textit{en général} on n'a pas $A\times B=B\times A$.
	\end{itemize}
\end{remarque}

\begin{exercice}[]
	\begin{itemize}
		\item 	On pose $A=\begin{matrice}
				      1&2\\3&0
			      \end{matrice}$ et $B=\begin{matrice}
				      5&2\\0&3
			      \end{matrice}$.\\
		      
		      Calculer $AB$ et $BA$.
		      
		\item 	Recommencer avec $A=\begin{matrice}
				      -4&6\\-3&5
			      \end{matrice}$ et $B=\begin{matrice}
				      8&-10\\5&-7
			      \end{matrice}$.
	\end{itemize}
\end{exercice}

\begin{propriete}[s de calcul]
	
	$A$, $B$ et $C$ sont des matrices.\\
	Lorsque les opérations sont possibles (bonnes dimensions des matrices) on a :
	\begin{itemize}
		\item 	$A(BC)=(AB)C$;
		\item 	$(A+B)C=AC+BC$;
		\item 	$A(B+C)=AB+AC$.\\
	\end{itemize}
	Soit $k$ un nombre réel alors on a également $A\times kB=kAB$.\\
	
	Si $A$ est \textit{carrée d'ordre} $n$ on a
	\begin{itemize}
		\item 	$AI_n=I_nA=A$ où $I_n$ est la matrice identité d'ordre $n$.
		\item 	$A\times 0=0\times A=0$ en notant 0 la matrice carrée d'ordre $n$ nulle.
	\end{itemize}
\end{propriete}

\section{Exemple concret d'utilisation}

Imaginons une école qui forme des ingénieurs en informatique, avec seulement 3 matières.
Trois élèves de première année ont obtenu les résultats suivants :

\begin{center}
	\textbf{Résultats pour le premier trimestre}\\[1em]
	\tabstyle[UGLiBlue]
	\begin{tabular}{|c|c|c|c|c|}
		\hline
		\bcell   & \ccell Maths & \ccell Physique & \ccell Info \\
		\hline
		Adam     & 12           & 8               & 16          \\
		\hline
		Bertrand & 18           & 14              & 12          \\
		\hline
		Charles  & 5            & 20              & 15          \\
		\hline
	\end{tabular}\\
	
	\textbf{Résultats pour le deuxième trimestre}\\[1em]
	
	\begin{tabular}{|c|c|c|c|c|}
		\hline
		\bcell   & \ccell Maths & \ccell Physique & \ccell Info \\
		\hline
		Adam     & 10           & 10              & 14          \\
		\hline
		Bertrand & 18           & 12              & 14          \\
		\hline
		Charles  & 7            & 14              & 17          \\
		\hline
	\end{tabular}
\end{center}

Ces deux tableaux peuvent s'écrire matriciellement
$S_1= \begin{matrice}
		12 & 8 & 16 \\
		18 & 14 & 12 \\
		5 & 20 & 15 \\
	\end{matrice}$
et $S_2= \begin{matrice}
		10 & 10 & 14 \\
		18 & 12 & 14 \\
		7 & 14 & 17 \\
	\end{matrice}$\\

Pour calculer les moyennes mensuelles des élèves « en une fois » on peut définir $M=0,5(S_1+S_2)$:

$$M = 0,5 \times \left[\begin{matrice}
			12 & 8 & 16 \\
			18 & 14 & 12 \\
			5 & 20 & 15 \\
		\end{matrice}+\begin{matrice}
			10 & 10 & 14 \\
			18 & 12 & 14 \\
			7 & 14 & 17 \\
		\end{matrice}\right]$$

$$M = 0,5\times \begin{matrice}
		22 & 18 & 30 \\
		36 & 16 & 26 \\
		12 & 34 & 32 \\
	\end{matrice}$$

$$M = \begin{matrice}
		11 & 9 & 15 \\
		18 & 8 & 13 \\
		6 & 17 & 16 \\
	\end{matrice}$$

Le coefficient des mathématiques est 1, celui de la physique est 2 et celui de l'informatique est 5.\\
Pour passer en deuxième année, il faut un total de points supérieur ou égal à 120.

Pour faire « d'un coup » le total des points, on peut considérer la matrice de coefficients $C=\begin{matrice}
		1\\2\\5
	\end{matrice}$.\\

Les points des élèves sont donnés par la matrice $$P=MC$$
$$P = \begin{matrice}
		11 & 9 & 15 \\
		18 & 8 & 13 \\
		6 & 17 & 16 \\
	\end{matrice}\begin{matrice}
		1\\2\\5
	\end{matrice}$$

$$P = \begin{matrice}
		104\\109\\120\\
	\end{matrice}$$

Ainsi seul Charles est admis à passer en 2\eme année.

\section{Matrices inversibles et systèmes}

\begin{definition}[ et propriété : Matrice inversible, inverse d'une matrice]
	Soit $A$ une matrice \textbf{carrée d'ordre $n$}. S'il existe une matrice $B$ d'ordre $n$ telle que
	$$AB = I_n\qquad\text{ou}\qquad BA=I_n$$
	
	alors automatiquement les deux égalités sont vérifiées, $B$ est nécessairement \textit{unique} et on dit alors que $B$ \textit{est l'inverse de $A$}. De manière symétrique $A$ est également l'inverse de $B$ si bien qu'on dit que $A$ et $B$ sont inverses l'une de l'autre.
	
	On note ceci $A=B^{-1}$ ou, ce qui revient au même, $B=A^{-1}$.
\end{definition}

\begin{exemple}[]
	$A=\begin{matrice}
			-1&2\\-2&3
		\end{matrice}$ et $B=\begin{matrice}
			3&-2\\2&-1
		\end{matrice}$ sont inverses l'une de l'autre :
	\begin{center}
		\begin{tabular}{cc}
			 & $\begin{matrice}
					    3&-2\\2&-1
				    \end{matrice}$ \\
			$\begin{matrice}
					 -1&2\\-2&3
				 \end{matrice}$
			 & $\begin{matrice}
					    1&0\\0&1
				    \end{matrice}$
		\end{tabular}
	\end{center}
\end{exemple}
\begin{exercice}[]
	Montrer que $A=\begin{matrice}
			3 & -2&1\\
			-2&2&-1\\
			1&-1&1
		\end{matrice}$ et
	$B=\begin{matrice}
			1& 1&0\\
			1&2&1\\
			0&1&2
		\end{matrice}$ sont inverses.
	
\end{exercice}
\begin{remarque}[]
	Il existe des matrices non inversibles, par exemple $\begin{matrice}
			1 & 2\\ 2 & 4
		\end{matrice}$.
\end{remarque}


\newcolumntype{C}{>{{}}c<{{}}} % for columns with binary operators
\renewcommand\vv{\multicolumn{1}{c}{\vdots}}
\begin{methode}[ : résoudre des systèmes avec des matrices]
	On considère un \textit{système de $n$ équations à $n$ inconnues} :
	
	$$\left\lbrace
		\setlength{\arraycolsep}{0pt}
		\begin{array}{c<{x_1} C c<{x_2} C c C c<{x_n} C l}
			a_{11} & + & a_{12} & + & \cdots & + & a_{1n} & = & b_1    \\
			a_{21} & + & a_{22} & + & \cdots & + & a_{2n} & = & b_2    \\
			\vv    &   & \vv    &   &        &   & \vv    &   & \vdots \\
			a_{n1} & + & a_{n2} & + & \cdots & + & a_{nn} & = & b_n    \\
		\end{array}\right.$$
	
	On connaît tous les nombres $a_{ij}$ et tous les $b_i$, et on veut trouver les valeurs des inconnues $x_i$.\\
	
	Ce système peut se réécrire de manière matricielle :
	
	$$\begin{matrice}
			a_{11}      & a_{12}&\cdots & a_{1n} \\
			a_{21}  & a_{22}& \cdots & a_{2n} \\
			\vdots 	& \vdots & \ddots & \vdots \\
			a_{n1}  & a_{n2}    & \cdots & a_{nn}
		\end{matrice}
		\begin{matrice}
			x_{1}       \\
			x_2\\
			\vdots \\
			x_n
		\end{matrice}
		=\begin{matrice}
			b_{1}       \\
			b_2\\
			\vdots \\
			b_n
		\end{matrice}$$
	
	Ou encore :
	$$AX=B$$
	
	Si la matrice $A$ est inversible (en pratique ce sera toujours le cas parce qu'on nous donnera sa matrice inverse ou bien parce qu'on l'aura déterminée à l'aide de la calculatrice) alors, on peut reprendre l'égalité précédente et écrire : $A^{-1}AX=A^{-1}B$, ce qui donne $I_nX=A^{-1}B$. En définitive on a $$X = A^{-1}B$$
	
	Ainsi pour trouver les valeurs des inconnues $x_i$, on effectue simplement le produit matriciel $A^{-1}B$ : chacune de ses lignes nous donne la valeur du $x_i$ correspondant.
\end{methode}

\begin{remarque}[]
	Pour savoir comment utiliser la calculatrice, regarder ici :
	\begin{itemize}
		\item 	modèles \textsc{CASIO} \texttt{https://youtu.be/yjvQx13Vhlk}
		\item 	modèles \textsc{Texas Instrument} \texttt{https://youtu.be/rxDxBnIwaGo}
	\end{itemize}
\end{remarque}

\begin{exemple}[]
	On considère le système suivant : \systeme{2x+5y+2z=1,
		5x-3y-2z=2,
		-x+2y+z=-3}\\
	
	Il peut se réécrire de manière matricielle :
	
	$$\begin{matrice}
			2 & 5 & 2 \\
			5 & -3 & -2\\
			-1 & 2 & 1
		\end{matrice}
		\begin{matrice}
			x \\ y \\ z
		\end{matrice}=
		\begin{matrice}
			1 \\ 2 \\ -3
		\end{matrice}$$
	
	Appelons $A$ la matrice carrée du membre de gauche. On détermine que $A$ est inversible avec la calculatrice et que son inverse est
	$$A^{-1}=\begin{matrice}
			1 & -1 & -4 \\
			-3 & 4 & 14\\
			7 & -9 & -31
		\end{matrice}$$
	On a donc
	
	$$\begin{matrice}
			x \\ y \\ z
		\end{matrice}=\begin{matrice}
			1 & -1 & -4 \\
			-3 & 4 & 14\\
			7 & -9 & -31
		\end{matrice}\begin{matrice}
			1 \\ 2 \\ -3
		\end{matrice}$$
	C'est à dire, en effectuant le produit dans le membre de droite
	
	$$\begin{matrice}
			x \\ y \\ z
		\end{matrice}=\begin{matrice}
			11\\-37\\82
		\end{matrice}$$
	
	On a donc résolu le système : \systeme{x=11,y=-37,z=82}
\end{exemple}
\begin{exercice}[]
	\begin{enumerate}
		\item 	Effectue le produit suivant : $\begin{matrice}
				      1 & 2 \\ -3 & 3\end{matrice}\begin{matrice}x\\y\end{matrice}$.
		\item 	\'A l'aide de la calculatrice détermine l'inverse de la matrice $\begin{matrice}
				      1 & 2 \\ -3 & 3\end{matrice}$.
		\item Résous le système suivant : \systeme{x+2y = 15,-3x+3y = -6}
	\end{enumerate}
\end{exercice}


\section{Exercices}

\begin{exercice}

	Dans un parc d’une ville, deux marchands ambulants vendent des beignets,
	des crêpes et des gaufres. On a noté les ventes de chacun pour samedi et dimanche derniers.
	
	\begin{center}
		Marchand 1\\[1em]
		\tabstyled
		\begin{tabular}{|c|c|c|c|}
			\hline
			\bcell   & \ccell beignets & \ccell crêpes & \ccell gaufres \\
			\hline
			samedi   & 20              & 36            & 12             \\
			\hline
			dimanche & 26              & 40            & 18             \\
			\hline
		\end{tabular}\ \\[2em]
		
		Marchand 2\\[1em]
		
		\begin{tabular}{|c|c|c|c|}
			\hline
			\bcell   & \ccell beignets & \ccell crêpes & \ccell gaufres \\
			\hline
			samedi   & 30              & 40            & 22             \\
			\hline
			dimanche & 30              & 48            & 38             \\
			\hline
		\end{tabular}
	\end{center}
	
	
	On peut retenir l’information donnée par un tableau en conservant uniquement les
	nombres disposés de la même façon. On représente le 1er tableau par la matrice A :
	$$A=\begin{matrice}
			20&36&12\\
			26&40&18
		\end{matrice}$$
	
	\begin{enumerate}
		\item 	Donner la matrice B représentant le deuxième tableau.
		\item 	Que valent $a_{12}$, $a_{11}$, $a_{23}$ et $b_{11}$ ?
		\item 	Calculer $A+B$ et donner la signification de la matrice.
		\item 	Calculer $A-B$ et donner la signification de la matrice.
		\item  	Samedi et dimanche prochains, weekend de fête, on prévoit que les ventes vont augmenter de 50\%.\\
		      Par quel nombre k faut-il multiplier chacune des ventes du 1\er marchand ? \'Ecrire la matrice  kA.\\
		      Donner la matrice kB correspondant aux ventes du 2\eme marchand.
		\item 	Un beignet est vendu 2 euros, une crêpe 1 euro et une gaufre 1,50 euro.\\
		      On note V la matrice des prix de vente
		      $$V=\begin{matrice}
				      2\\
				      1\\
				      1,5
			      \end{matrice}$$
		      Quelle opération matricielle donne le montant des ventes par jour pour le 1\er marchand ? Pour le 2\eme ?
		\item 	 Les deux marchands travaillent pour le compte du même patron, qui leur demande	de calculer les coûts d’achats et les revenus pour chaque jour. Le coût d’achat d’un
		      beignet est 0,40 euro, d’une crêpe 0,25 euro, d’une gaufre 0,30 euro. On note T la matrice donnant prix d’achat et prix de vente par catégorie
		      $$T=\begin{matrice}
				      0,4 & 2\\
				      0,25 & 1\\
				      0,3 & 1,5
			      \end{matrice}$$
		      Quelle opération matricielle permet le calcul des coûts d’achat et revenus par jour
		      pour le 1\er marchand ?\\
		      Calculer, de même, les coûts d’achats et les revenus par jour pour le 2\eme marchand
		      puis, globalement, pour le patron.
	\end{enumerate}
\end{exercice}


\begin{exercice}[]
	$A=\begin{matrice}
			1&0&-2\\
			2&3&1
		\end{matrice}$,
	$B=\begin{matrice}
			1&0&1\\
			2&1&3\\
			3&-1&-2
		\end{matrice}$ et
	$C=\begin{matrice}
			3&0&-2&0\\
			-2&1&1&1\\
			1&-1&0&3
		\end{matrice}$.\\
	
	\begin{enumerate}
		\item 	Calculer $A\times B$, puis $(A\times B)\times C$.
		\item 	Calculer $B\times C$, puis $A\times(B\times C)$.
		\item   Pouvait-on prévoir ce résultat ?\\
	\end{enumerate}
\end{exercice}

\begin{exercice}
	$A=\begin{matrice}
			1&3\\
			2&6
		\end{matrice}$,
	$B=\begin{matrice}
			-3&-6\\
			1&2
		\end{matrice}$,
	$C=\begin{matrice}
			1&1&2\\
			2&2&4\\
			3&3&6
		\end{matrice}$ et
	$D=\begin{matrice}
			1&2&-6\\
			1&2&-6\\
			1&-2&6
		\end{matrice}$.
	
	\begin{enumerate}
		\item 	Calculer le produit $A\times B$.
		\item 	Calculer le produit $C\times D$.
		\item  Que peut-on en conclure ?
	\end{enumerate}
	
\end{exercice}


\begin{exercice}[ ]
	
	$A=\begin{matrice}
			1&1\\
			0&1
		\end{matrice}$.\\
	
	Calculer $A^5$.
\end{exercice}

\begin{exercice}[ : Calculs à la main]
	
	On considère les matrice $A=\begin{matrice}
			-5&2\\
			-3&1
		\end{matrice}$ et $B=\begin{matrice}
			1&-2\\
			3&-5
		\end{matrice}$.
	
	\begin{enumerate}
		\item 	Montrer \textit{à la main} que A et B sont inverses.
		\item 	On considère le système (S) suivant :
		      $$\begin{cases}
				      -5x+2y & =7 \\
				      -3x+y  & =8
			      \end{cases}$$
		      
		      Montrer que ce système peut se réécrire matriciellement $$AX=Y$$ et préciser $X$ et $Y$
		\item 	En déduire \textit{à la main} les solutions du système (S).\\
	\end{enumerate}
\end{exercice}

\begin{exercice}[ : À la calculatrice]
	
	On considère les matrice $A=\begin{matrice}
			8&11&3\\
			4&7&2\\
			1&3&1\\
		\end{matrice}$ et $B=\begin{matrice}
			1&-2&1\\
			-2&5&-4\\
			5&-13&12
		\end{matrice}$.
	
	\begin{enumerate}
		\item 	Comment avec la calculatrice vérifie-t-on que A et B sont inverses ?
		\item 	On considère le système (S) suivant :
		      $$\begin{cases}
				      8x+11y+3z & =1 \\
				      4x+7y+2z  & =2 \\
				      x+3y+z    & =3
			      \end{cases}$$
		      
		      Montrer que ce système peut se réécrire matriciellement $$AX=Y$$ et préciser $X$ et $Y$
		\item 	En déduire \textit{à la main} les solutions du système (S).\\
	\end{enumerate}
	
\end{exercice}

\begin{exercice}
	À la papeterie:
	\begin{itemize}
		\item 	3 stylos, 2 cahiers et 4 gommes coûtent 6,30€;
		\item 	5 stylos, 7 cahiers et 1 gomme coûtent 15€;
		\item 	10 stylos, 1 cahier et 6 gommes coûtent 6€.
	\end{itemize}
	À l'aide de la calculatrice et en expliquant la démarche, déterminer le prix de chaque article.
\end{exercice}

\begin{exercice}[]
	
	Une société produit trois types de fibres optiques à partir de silice, forme naturelle du dioxyde de silicium (SiO$_2$) qui entre dans la composition de nombreux minéraux. Elle produit:
	
	\begin{itemize}
		\item $x$ pièces du type A, dont le débit supporté vaut 1 gigabit par seconde;
		\item $y$ pièces du type B, dont le débit supporté vaut 10 gigabits par seconde;
		\item $z$ pièces du type C, dont le débit supporté vaut 100 gigabits par seconde.
	\end{itemize}
	
	
	Pour une pièce, la masse de silice utilisée et le temps de production de chacun de ces types de fibres sont récapitulés dans le tableau suivant.
	
	\begin{center}
		\tabstyled
		\begin{tabular}{c|c|c|c}
			\hline
			\ccell Type de fibre                        & A & B & C \\
			\hline
			\ccell Masse de silice en kg (par pièce)    & 3 & 4 & 7 \\
			\hline
			\ccell Temps de production en h (par pièce) & 2 & 3 & 5 \\
			\hline
		\end{tabular}
	\end{center}
	
	La société modélise cette fabrication afin d'envisager différents scénarios sur une période donnée. Pour cette période, on note $N$ le nombre total de pièces produites, $S$ la masse totale en kg de silice utilisée et $H$ le temps total de production exprimé en heure.
	
	\begin{enumerate}
		\item  Justifier le fait que $x$, $y$, $z$ vérifient le système
		      $\left\lbrace
			      \begin{array}{l !{=} l}
				      x+y+z    & N \\
				      3x+4y+7z & S \\
				      2x+3y+5z & H
			      \end{array}
			      \right .$.
		      
		\item On considère les matrices colonnes
		      $X= \begin{matrice} x \\ y \\ z \end{matrice}$ et
		      $Y= \begin{matrice} N \\ S \\ H \end{matrice}$.
		      Déterminer la matrice carrée $M$ qui traduit le système ci-dessus par l'équation matricielle $M\times X = Y$.
		      
		\item Calculer $Y$ lorsque
		      $X= \begin{matrice} 20 \\ 10 \\ 30 \end{matrice}$.
		      Interpréter les résultats obtenus dans le contexte de l'exercice.
		      
		\item On considère la matrice carrée
		      $P=
			      \begin{matrice}
				      1 & 2 & -3 \\
				      1 & -3 & 4 \\
				      -1 & 1 & -1 \\
			      \end{matrice}$.
		      
		      \begin{enumalph}
			      \item Calculer le produit matriciel $P\times M$.
			      \item Montrer que si $M\times X = Y$, alors $X = P\times Y$.
			      \item Pour une période donnée, l'entreprise dispose de 94~kg de silice et de 67 heures de production. Elle souhaite fabriquer 21 pièces de fibres.
			      
			      Combien de pièces de chaque type peut-elle fabriquer?
		      \end{enumalph}
	\end{enumerate}
\end{exercice}