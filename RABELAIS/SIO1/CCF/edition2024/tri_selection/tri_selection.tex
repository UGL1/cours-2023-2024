\documentclass[a4paper,12pt,article,firamath]{nsi}

\documentclass[a4paper,12pt,cours,firamath]{nsi}
		
\setminted{fontsize=\small}
\begin{document}
{\large\bfseries \scshape Nom Prénom : \makebox[6cm]{\dotfill}\hfill Heure de passage : \makebox[3cm]{\dotfill}\hfill\\
\vspace{2em}
\hrule
\vspace{2mm}
\begin{center}\titlefont\Huge\color{UGLiBlue} BTS SIO\\
	Sous-épreuve E22 \\ 
    Algorithmique appliquée\\
	Contrôle en Cours de Formation\end{center}
\vspace{2mm}
\hrule}
\vspace{2em}
\begin{encadrecolore}{Déroulement de l'épreuve }{UGLiOrange}
	Cette épreuve de Contrôle en cours de Formation (CCF) se déroule en trois étapes :
\begin{itemize}
	\item \textbf{\'Etape 1 : \'Ecrit (30 minutes)}\par
	Vous devez traiter la partie A du sujet. Pour cette partie, l'ordinateur est interdit mais la calculatrice est autorisée.\\
    
    \textbf{Vous inscrirez vos réponses dans le document réponse à la fin du sujet.}\\
    
    Les algorithmes à écrire peuvent être rédigés en \textbf{langage naturel} ou en \textsc{Python}	mais ni en \textsc{C\#} ni en \textsc{VB.Net}.\\
    
    \textbf{À la fin de l'étape 1, votre document réponse doit être remis à la personne surveillant l'épreuve.} Vous garderez le sujet.
    \item \textbf{\'Etape 2 : sur machine (30 minutes)}\par
	Vous devez traiter la partie B du sujet à l'aide d'un ordinateur. Le langage utilisé est celui travaillé dans l'année, à savoir \textsc{Python}.
	Vous sauvegarderez votre travail sur la clé USB fournie.\par 
	La durée totale pour effectuer les deux premières étapes est exactement d'une heure. \par
	\item \textbf{\'Etape 3 : oral (20 minutes au maximum)}\\
	Cette partie se déroule en deux temps. Tout d'abord, vous disposez de 10 minutes pour présenter votre travail de l'étape 2 puis, au cours des 10 minutes suivantes, un entretien permet de préciser votre démarche.
\end{itemize}	

\textbf{À la fin de l'épreuve le sujet devra être rendu à l'examinateur.}
\end{encadrecolore}
\newpage


\titre{Tri par sélection}
\classe{CCF Algo SIO}
\maketitle
On dispose d'une liste d'entiers \mintinline{python}{lst} que l'on aimerait trier dans l'ordre croissant.\\
Nous allons utiliser l'algorithme de tri par sélection qui consiste à
\begin{itemize}
    \item	parcourir la liste et trouver l'indice du plus petit élément ;
    \item	échanger ce plus petit élément avec celui d'indice 0 ;
    \item   maintenant que l'élément d'indice 0 est bien placé, on parcourt la liste à partir de l'élément d'indice 1 pour trouver son plus petit élément ;
    \item   on l'échange avec l'élément d'indice 1 ;
    \item   on répète cela « jusqu'au bout de la liste ».
\end{itemize}
Voici un exemple avec \mintinline{python}{lst = [3, 2, 7, 4]} :
\begin{itemize}
    \item	on parcourt la liste en entier on trouve que le plus petit élément est 2, à l'indice 1 ;
    \item	on l'échange avec celui d'indice 0 on obtient la liste \mintinline{python}{[2, 3, 7, 4]} ;
    \item   on parcourt la liste à partir de l'indice 1, on trouve que le plus petit élément est 3, à l'indice 1 ;
    \item   on l'échange avec lui-même (même si c'est inutile) ;
    \item   on parcourt la liste à partir de l'indice 2, on trouve que le plus petit élément est 4, à l'indice 3 ;
    \item   on l'échange avec l'élément d'indice 2, on obtient \mintinline{python}{[2, 3, 4, 7]} ;
    \item   la liste est de longueur 4 donc comme on a rangé les 3 plus petits au début, le dernier élément est obligatoirement le plus grand donc on s'arrête.
\end{itemize}

\begin{questionnum}
    Applique cet algorithme étape par étape en faisant comme à l'exemple précédent avec \mintinline{python}{lst = [4, 2, 8, 0, 3]} 
\end{questionnum}

On a commencé à écrire l'algorithme de la fonction \mintinline{python}{indice_min} qui
\begin{itemize}
    \item	en entrée prend une liste \mintinline{python}{lst} et un entier \mintinline{python}{i}  
    \item	parcourt la liste \mintinline{python}{lst} à partir de l'élément d'indice \mintinline{python}{i}  inclus et renvoie l'indice du plus petit élément rencontré lors du parcours ;
\end{itemize} 
Par exemple \mintinline{python}{indice_min([0, 2, 4, 1, 8],2)} 
\begin{itemize}
    \item	parcourt la liste \mintinline{python}{[0, 2, 4, 1, 8]} à partir de l'indice 2 (donc ignore 0 et 2) ;
    \item trouve que le plus petit élément est 1, à l'indice 3 de la liste, donc renvoie 3.
\end{itemize}
\begin{questionnum}
    Complète sur ta copie l'algorithme de la fonction.
    \begin{minted}{pseudocode}
        fonction indice_minimum(lst, i)
            variables
                j, j_min : entiers
            jmin ← i
            pour j allant de i à longueur(lst) repeter
                si lst[j] < ... alors
                    ...
                fin si
            fin pour
            renvoyer ...
    \end{minted}
\end{questionnum}

On a commencé à écrire la fonction \mintinline{python}{tri_insertion} qui
\begin{itemize}
    \item	en entrée prend la liste \mintinline{python}{lst} à trier ; 
    \item	ne renvoie rien mais trie \mintinline{python}{lst} en appliquant le principe de tri par selection exposé plus haut.
\end{itemize}

\begin{questionnum}
    Complète sur ta copie l'algorithme de la fonction.
    \begin{minted}{pseudocode}
        fonction tri_selection(lst)
            variables
                i, i_min, tmp : entiers
            pour i allant de 0 à ... repeter
                i_min ← ...
                temp ← lst[i]
                lst[i] ← ...
                lst[i_min] ← temp
            fin pour
    \end{minted}
\end{questionnum}

\section*{Étape 2}

\begin{questionnum}
    Ouvrir le fichier \mintinline{python}{tri_selection.py} et coder les fonctions manquantes. 
\end{questionnum}

\begin{questionnum}
    En s'inspirant de ce qui a été fait, créer une fonction \mintinline{python}{tri_selection_decroissant} qui trie les listes dans l'ordre décroissant. 
\end{questionnum}
\end{document}
