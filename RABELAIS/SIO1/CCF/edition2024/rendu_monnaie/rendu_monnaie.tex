\documentclass[a4paper,12pt,article,firamath]{nsi}

\documentclass[a4paper,12pt,exos,firamath]{nsi}
		
\setminted{fontsize=\small}
\begin{document}
{\large\bfseries \scshape Nom Prénom : \makebox[6cm]{\dotfill}\hfill Heure de passage : \makebox[3cm]{\dotfill}\hfill\\
\vspace{2em}
\hrule
\vspace{2mm}
\begin{center}\titlefont\Huge\color{UGLiBlue} BTS SIO\\
	Sous-épreuve E22 \\ 
    Algorithmique appliquée\\
	Contrôle en Cours de Formation\end{center}
\vspace{2mm}
\hrule}
\vspace{2em}

\begin{encadrecolore}{Déroulement de l'épreuve }{UGLiGreen}
	Cette épreuve de Contrôle en cours de Formation (CCF) se déroule en trois étapes :
\begin{itemize}
	\item \textbf{\'Etape 1 : \'Ecrit (30 minutes)}\par
	Vous devez traiter la partie A du sujet. Pour cette partie, l'ordinateur est interdit mais la calculatrice est autorisée.\\
    
    \textbf{Vous inscrirez vos réponses dans le document réponse à la fin du sujet.}\\
    
    Les algorithmes à écrire peuvent être rédigés en \textbf{langage naturel} ou en \textsc{Python}	mais ni en \textsc{C\#} ni en \textsc{VB.Net}.\\
    
    \textbf{À la fin de l'étape 1, votre document réponse doit être remis à la personne surveillant l'épreuve.} Vous garderez le sujet.
    \item \textbf{\'Etape 2 : sur machine (30 minutes)}\par
	Vous devez traiter la partie B du sujet à l'aide d'un ordinateur. Le langage utilisé est celui travaillé dans l'année, à savoir \textsc{Python}.
	Vous sauvegarderez votre travail sur la clé USB fournie.\par 
	La durée totale pour effectuer les deux premières étapes est exactement d'une heure. \par
	\item \textbf{\'Etape 3 : oral (20 minutes au maximum)}\\
	Cette partie se déroule en deux temps. Tout d'abord, vous disposez de 10 minutes pour présenter votre travail de l'étape 2 puis, au cours des 10 minutes suivantes, un entretien permet de préciser votre démarche.
\end{itemize}	

\textbf{À la fin de l'épreuve le sujet devra être rendu à l'examinateur.}
\end{encadrecolore}
\newpage

\titre{Rendu de monnaie}
\classe{CCF Algo SIO}
\maketitle


\section*{Étape 1}

On dispose d'une liste de pièces et billets de monnaie rangée dans l'ordre décroissant. Pour faire simple on appellera « pièce » une pièce ou un billet sans faire la distinction.\\
Voici la variable \mintinline{python}{piece} ainsi que sa valeur :\\

\mintinline{python}{pieces = [100, 50, 20, 10, 5, 2, 1]}\\

Dans cette situation, pour rendre une certaine somme en le moins de pièces possible, c'est simple : il faut commencer par rendre la plus grosse pièce possible et recommencer tant qu'il reste de l'argent à rendre.\\

Par exemple $13 = 10 + 2 + 1$ et cette égalité indique qu'on rend 13€ en 3 pièces.
\begin{questionnum}
    Tu dois rendre 26€. Écris l'égalité correspondant au rendu de pièces.\\
    Combien de pièces faut-il pour rendre cette somme ?
\end{questionnum}

\begin{questionnum}
    Même question avec 267€.
\end{questionnum}

On a commencé à écrire une fonction \mintinline{python}{trouve_piece} qui
\begin{itemize}
    \item	en entrée prend un entier \mintinline{python}{somme} qui est la somme d'argent à rendre ;
    \item	renvoie le plus grand élément de la liste \mintinline{python}{piece} que l'on peut rendre
\end{itemize}
Par exemple \mintinline{python}{trouve_piece(49)} vaut 20 car 20 est la plus grosse pièce qu'on peut rendre quand on veut rendre 49€.

\begin{questionnum}
    Sur ta copie, complète l'algorithme de la fonction \mintinline{python}{trouve_piece} :
    \begin{minted}{pseudocode}
        fonction trouve_piece(somme)
            variables
                resultat, i, n : entiers
                trouve : booleen

            resultat ← 0
            trouve ← faux
            i ← 0
            n ← longueur(...)
            tant que i < n et trouve = faux repeter
                si piece[i] <= ... alors
                    resultat ← ...
                    trouve ← vrai
                fin si
                i ← ...
            fin tant que
            renvoyer resultat
    \end{minted}
\end{questionnum}

On a ensuite écrit la fonction \mintinline{python}{nb_pieces_rendues} qui
\begin{itemize}
    \item	en entrée prend un entier \mintinline{python}{somme} qui est la somme d'argent à rendre ;
    \item	renvoie le nombre de pièces nécessaires pour rendre la somme à l'aide des pièces de la liste \mintinline{python}{piece}.
\end{itemize}
Par exemple \mintinline{python}{nb_pieces_rendues(13)}  vaut 3 puisqu'on rend 13€ en 3 pièces.

\begin{questionnum}
    Sur ta copie, complète l'algorithme de la fonction \mintinline{python}{nb_pieces_rendues} :
    \begin{minted}{pseudocode}
        fonction trouve_piece(somme)
            variables
                resultat: entier
            
                resultat ← ...
            tant que somme > 0 repeter
                somme ← ...
                resultat ← resultat + 1
            fin tant que
            renvoyer resultat
    \end{minted}
\end{questionnum}

On suppose maintenant que \mintinline{python}{piece = [4,3,1]} 

\begin{questionnum}
    \begin{enumalph}
        \item	Que renvoie alors \mintinline{python}{nb_pieces_rendues(6)} ?
        \item	Peut-on rendre la valeur 6 avec moins de pièces ?
        \item   L'algorithme utilisé est-il optimal dans ce cas ?
    \end{enumalph}
\end{questionnum}

\section*{Étape 2}
\begin{questionnum}
    Ouvrir le fichier \mintinline{python}{rendu_monnaie.py} et compléter les fonctions.    
\end{questionnum}

\begin{questionnum}
    Écrire la fonction \mintinline{python}{liste_pieces_rendues qui}
    \begin{itemize}
        \item	en entrée prend un entier \mintinline{python}{somme} qui est la somme d'argent à rendre ;
        \item	renvoie la liste des pièces nécessaires pour rendre la somme à l'aide des pièces de la liste \mintinline{python}{piece}.
    \end{itemize}
    Par exemple \mintinline{python}{liste_pieces_rendues(13)}  vaut \mintinline{python}{[10, 2, 1]}.\\
    Autre exemple : \mintinline{python}{liste_pieces_rendues(43)}  vaut \mintinline{python}{[20, 20, 2, 1]}.
    
\end{questionnum}


\end{document}