\documentclass[a4paper,12pt,article,firamath]{nsi}

\documentclass[a4paper,12pt,cours,firamath]{nsi}
		
\setminted{fontsize=\small}
\begin{document}
{\large\bfseries \scshape Nom Prénom : \makebox[6cm]{\dotfill}\hfill Heure de passage : \makebox[3cm]{\dotfill}\hfill\\
\vspace{2em}
\hrule
\vspace{2mm}
\begin{center}\titlefont\Huge\color{UGLiBlue} BTS SIO\\
	Sous-épreuve E22 \\ 
    Algorithmique appliquée\\
	Contrôle en Cours de Formation\end{center}
\vspace{2mm}
\hrule}
\vspace{2em}
\begin{encadrecolore}{Déroulement de l'épreuve }{UGLiOrange}
	Cette épreuve de Contrôle en cours de Formation (CCF) se déroule en trois étapes :
\begin{itemize}
	\item \textbf{\'Etape 1 : \'Ecrit (30 minutes)}\par
	Vous devez traiter la partie A du sujet. Pour cette partie, l'ordinateur est interdit mais la calculatrice est autorisée.\\
    
    \textbf{Vous inscrirez vos réponses dans le document réponse à la fin du sujet.}\\
    
    Les algorithmes à écrire peuvent être rédigés en \textbf{langage naturel} ou en \textsc{Python}	mais ni en \textsc{C\#} ni en \textsc{VB.Net}.\\
    
    \textbf{À la fin de l'étape 1, votre document réponse doit être remis à la personne surveillant l'épreuve.} Vous garderez le sujet.
    \item \textbf{\'Etape 2 : sur machine (30 minutes)}\par
	Vous devez traiter la partie B du sujet à l'aide d'un ordinateur. Le langage utilisé est celui travaillé dans l'année, à savoir \textsc{Python}.
	Vous sauvegarderez votre travail sur la clé USB fournie.\par 
	La durée totale pour effectuer les deux premières étapes est exactement d'une heure. \par
	\item \textbf{\'Etape 3 : oral (20 minutes au maximum)}\\
	Cette partie se déroule en deux temps. Tout d'abord, vous disposez de 10 minutes pour présenter votre travail de l'étape 2 puis, au cours des 10 minutes suivantes, un entretien permet de préciser votre démarche.
\end{itemize}	

\textbf{À la fin de l'épreuve le sujet devra être rendu à l'examinateur.}
\end{encadrecolore}
\newpage

\titre{Suites de Syracuse}
\classe{CCF Algo SIO}
\maketitle

\section*{\'Etape 1}

On considère un entier $n\geqslant 1$ et on lui applique le processus suivant :

\begin{encadrecolore}{Processus}{UGLiGreen}
    \begin{itemize}
        \item si $n$ est pair, on le divise par 2;
        \item sinon, on le multiplie par 3 et on ajoute 1.
    \end{itemize}
\end{encadrecolore}
Par exemple, si on choisit 5, puisqu'il est impair, on le multiplie par 3 et on ajoute 1, ce qui fait 16.\\

Le but de cette épreuve est d'étudier des propriétés de ce processus lorsqu'on le répète plusieurs fois à la suite.


\begin{questionnum}
    \begin{enumalph}
        \item Répéter ce processus huit fois en prenant 5 comme valeur de départ en indiquant les résultats successifs.\\
        Comment cela va-t-il se poursuivre ?
        \item Répéter 4 fois ce processus en prenant 13 et expliquer comment on peut continuer un peu « sans se fatiguer ».
    \end{enumalph}
\end{questionnum}
Le résultat suivant n'est pas (encore) démontré :
\begin{encadrecolore}{Conjecture}{UGLiRed}
    Quel que soit l'entier positif $n$ que l'on prend au départ, en répétant ce processus on atteint \textit{immanquablement} la valeur 1, après quoi le cycle $(4,\,2,\,1)$ se répète indéfiniment.
\end{encadrecolore}
On aimerait écrire une fonction \pythoninline{temps_de_vol} qui
\begin{itemize}
    \item en entrée prend un entier \texttt{n}
    \item renvoie le nombre de répétitions du processus nécessaire pour obtenir la valeur 1.
\end{itemize}


\begin{questionnum}
    \begin{enumalph}
        \item Que devra renvoyer \pythoninline{temps_de_vol(5)} ?
        \item Que devra renvoyer \pythoninline{temps_de_vol(13)} ?
        \item Compléter sur votre copie la ligne de définition de la fonction avec les indications de type nécessaires :
    \end{enumalph}
\end{questionnum}

\begin{pyc}
    \begin{minted}{python}
def temps_de_vol(.........) -> ......... :        
    \end{minted}
\end{pyc}


\begin{questionnum}
    Compléter sur votre copie la fonction suivante.
\end{questionnum}

\begin{pyc}
    \begin{minted}{python}
resultat = 0
    while .........:
        if .........:
            n = .........
        else:
            n = .........
        resultat = .........
     return .........        
    \end{minted}
\end{pyc}


On appelle \textit{hauteur de vol} de $n$ le plus grand nombre qu'on obtient quand on répète le processus jusqu'à obtenir 1.

\begin{questionnum}
    \begin{enumalph}
        \item Quelle est la hauteur de vol de 5 ? Quelle est celle de 13 ?
        \item \'Ecrire en \textsc{Python} la fonction \pythoninline{hauteur_de_vol}
    \end{enumalph}
\end{questionnum}



\section*{\'Etape 2}

Ouvrir le fichier \pythoninline{syracuse.py}.\\
Celui-ci contient une implémentation de la fonction \pythoninline{temps_de_vol} de l'étape 1.

\begin{questionnum}
    Implémenter la fonction \pythoninline{hauteur_de_vol} de l'étape 1 
\end{questionnum}


On dit qu'un nombre vole bien lorsque son temps de vol est plus grand ou égal à lui-même. Par exemple, 7 vole bien car son vol est
$$(22,\, 11,\,34,\, 17,\, 52,\, 26,\, 13,\, 40,\, 20,\, 10,\, 5,\, 16,\, 8,\, 4,\, 2,\, 1)$$
et sa longueur (16), est plus grande que 7.


\begin{questionnum}
    Coder la fonction \pythoninline{vole_bien} qui
\begin{itemize}
    \item en entrée prend un entier $n$ positif;
    \item renvoie \pythoninline{True} si $n$ vole bien et \pythoninline{False} sinon.
\end{itemize}
\end{questionnum}

\end{document}
