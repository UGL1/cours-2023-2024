\documentclass[a4paper,12pt,article,firamath]{nsi}

\documentclass[a4paper,12pt,cours,firamath]{nsi}
		
\setminted{fontsize=\small}
\begin{document}
{\large\bfseries \scshape Nom Prénom : \makebox[6cm]{\dotfill}\hfill Heure de passage : \makebox[3cm]{\dotfill}\hfill\\
\vspace{2em}
\hrule
\vspace{2mm}
\begin{center}\titlefont\Huge\color{UGLiBlue} BTS SIO\\
	Sous-épreuve E22 \\ 
    Algorithmique appliquée\\
	Contrôle en Cours de Formation\end{center}
\vspace{2mm}
\hrule}
\vspace{2em}
\begin{encadrecolore}{Déroulement de l'épreuve }{UGLiOrange}
	Cette épreuve de Contrôle en cours de Formation (CCF) se déroule en trois étapes :
\begin{itemize}
	\item \textbf{\'Etape 1 : \'Ecrit (30 minutes)}\par
	Vous devez traiter la partie A du sujet. Pour cette partie, l'ordinateur est interdit mais la calculatrice est autorisée.\\
    
    \textbf{Vous inscrirez vos réponses dans le document réponse à la fin du sujet.}\\
    
    Les algorithmes à écrire peuvent être rédigés en \textbf{langage naturel} ou en \textsc{Python}	mais ni en \textsc{C\#} ni en \textsc{VB.Net}.\\
    
    \textbf{À la fin de l'étape 1, votre document réponse doit être remis à la personne surveillant l'épreuve.} Vous garderez le sujet.
    \item \textbf{\'Etape 2 : sur machine (30 minutes)}\par
	Vous devez traiter la partie B du sujet à l'aide d'un ordinateur. Le langage utilisé est celui travaillé dans l'année, à savoir \textsc{Python}.
	Vous sauvegarderez votre travail sur la clé USB fournie.\par 
	La durée totale pour effectuer les deux premières étapes est exactement d'une heure. \par
	\item \textbf{\'Etape 3 : oral (20 minutes au maximum)}\\
	Cette partie se déroule en deux temps. Tout d'abord, vous disposez de 10 minutes pour présenter votre travail de l'étape 2 puis, au cours des 10 minutes suivantes, un entretien permet de préciser votre démarche.
\end{itemize}	

\textbf{À la fin de l'épreuve le sujet devra être rendu à l'examinateur.}
\end{encadrecolore}
\newpage


\titre{Nombre de diviseurs d'un entier}
\classe{CCF Algo SIO}
\maketitle

\section*{\'Etape 1}

On s'intéresse au nombre de diviseurs qu'un entier peut avoir : on veut fabriquer une fonction \pythoninline{nb_div} qui
\begin{itemize}
	\item en entrée prend un \pythoninline{int} non nul \pythoninline{n};
    \item renvoie  le nombre de diviseurs de \pythoninline{n}.
\end{itemize}

\begin{questionnum}
Que devra renvoyer \pythoninline{nb_div(12)} ? \pythoninline{nb_div(49)} ?
\end{questionnum}

On a implémenté la fonction en \textsc{Python} :

\begin{pyc}
    \begin{minted}{python}
def nb_div( n : int) -> int:
    r = ......
    for i in range(......):
        if n % i == ......:
            r = ......
    return r        
    \end{minted}
\end{pyc}
\begin{questionnum}
    Compléter sur votre copie les pointillés pour que la fonction remplisse son rôle.    
\end{questionnum}

On souhaite construire une fonction \pythoninline{nb_div_tous_entiers} qui :
\begin{itemize}
	\item  en entrée prend un \pythoninline{int} non nul \pythoninline{n};
    \item  renvoie la liste de tous les diviseurs de 1, 2, 3 jusqu'à \pythoninline{n}.
\end{itemize}
Par exemple \pythoninline{nb_div_tous_entiers(6)} renverra \\
\pythoninline{[1, 2, 2, 3, 2, 4]} car 
\begin{itemize}
	\item 1 n'a qu'un diviseur;
    \item 2 en a 2;
    \item 3 en a 2;
    \item \textit{et c\ae tera}.
\end{itemize}

\begin{questionnum}
    \'Ecrire la fonction \pythoninline{nb_div_tous_entiers}.    
\end{questionnum}


On dispose de la fonction \pythoninline{mystere} suivante :

\begin{pyc}
    \begin{minted}{python}
def mystere(n : int) -> int:
    m = 0
    r = 0
    for i in range(1, n+1):
        if nb_div(i) > m:
            m = nb_div(i)
            r = i
    return r
        
    \end{minted}
\end{pyc}

\begin{questionnum}
    Que fait la fonction \pythoninline{mystere} ?\\
    Que renvoie \pythoninline{mystere(10)} ?        
\end{questionnum}

\section*{\'Etape 2}

Les fonctions de l'étape précédente ont été implémentées en \textsc{Python} dans le fichier \texttt{nombre\_diviseurs.py}.

\begin{questionnum}
    \'Ecrire la fonction \pythoninline{nb_div_connus} qui
    \begin{itemize}
        \item en entrée prend un entier \pythoninline{n} non nul;
        \item renvoie le plus petit entier qui a \pythoninline{n} diviseurs.
    \end{itemize}
    Par exemple \pythoninline{nb_div_connus(4)} devra renvoyer 6 car 6 est le plus petit entier qui a 4 diviseurs.        
\end{questionnum}


\begin{questionnum}
    \'Ecrire la fonction \pythoninline{liste_div_connus} qui
    \begin{itemize}
        \item en entrée prend un entier 3 entiers \pythoninline{a}, \pythoninline{b} et \pythoninline{n} avec \pythoninline{0 < a < b}  et \pythoninline{n} non nul;
        \item renvoie la liste des entiers compris entre \pythoninline{a} et \pythoninline{b} \textbf{inclus} qui ont exactement \pythoninline{n} diviseurs
    \end{itemize}        
\end{questionnum}
\end{document}
