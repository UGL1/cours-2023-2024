\chapter{Variables et affectations}

\section{Le symbole =}

En mathématiques, le symbole $=$ a plusieurs significations :
\begin{itemize}
	\item dans $2+2=4$, on peut comprendre $=$ comme un opérateur d'évaluation : $2+2$, cela « donne » $4$ ;
	\item dans $\mathcal{P}=2\times(\ell+L)$, on peut considérer que $=$ sert à définir ce qu'est le périmètre d'un rectangle de dimensions $\ell$ et $L$ ;
	\item dans $3x + 2 = 4x +5$, le $=$ sert à convenir que les 2 membres ont la même valeur et on cherche s'il existe un ou des nombres $x$ qui satisfont l'égalité (appelée équation) ;
	\item \textit{et cætera}.
\end{itemize}

En \textsc{Python}, le symbole \mintinline{python}{=}  n'a qu'un seul sens : il sert à l'\textit{affectation}.

\section{L'affectation}
Il s'agit de « stocker » une valeur dans un endroit de la mémoire auquel \textsc{Python} donne un nom\footnote{En réalité c'est plus compliqué mais cela ne nous intéresse pas pour le moment.}. Voici un exemple d'affectation :
\begin{center}\Large
	\mintinline{python}{a = 2}
\end{center}\
\begin{itemize}
	\item   \mintinline{python}{2} est créee en tant que  \textit{valeur} de type \mintinline{python}{int} ;
	\item   la \textit{variable} \mintinline{python}{a} est créée ;\
	\item   \mintinline{python}{a} est « attachée » à la valeur \mintinline{python}{2} ;
	\item   par extension \mintinline{python}{a} est également de type \mintinline{python}{int}.
\end{itemize}
Au cours d'un programme la valeur associée à une variable peut changer\ldots D'où le nom de \textit{variable}.\\


\begin{definition}[ : affectation]
	Lors d'une affectation
	\begin{itemize}
		\item d'abord \textsc{Python} évalue ce qu'il y a à droite du symbole \mintinline{python}{=} ;
		\item si cette valeur n'existe pas déjà en mémoire, elle est créée ;
		\item ensuite il affecte cette valeur à la variable qui figure à gauche du symbole \mintinline{python}{=} ;
		\item si la variable n'existe pas déjà, elle est créée automatiquement ;
		\item le type de la variable, c'est le type de la valeur qu'on lui affecte.
	\end{itemize}
\end{definition}
Que fait le programme suivant ?

\begin{pyc}
	\begin{minted}{python}
x = 0
x = x + 1
print(x)    
\end{minted}
\end{pyc}

\begin{itemize}
	\item il crée une variable \mintinline{python}{x} de type \mintinline{python}{int} valant \mintinline{python}{0} ;
	\item il évalue \mintinline{python}{x + 1}, trouve \mintinline{python}{1} et affecte cette valeur à \mintinline{python}{x} ;
	\item évalue \mintinline{python}{x}, trouve 1 et donc affiche \mintinline{python}{1}.
\end{itemize}

\begin{aretenir}
	En mathématiques, $x = x + 1$ est une équation sans solution.\\

	En \textsc{Python}, l'instruction \mintinline{python}{x = x + 1} sert à augmenter la valeur de \mintinline{python}{x} de 1 (on dit aussi \textit{incrémenter}).
\end{aretenir}

\subsection{Affectations multiples}
\textsc{Python} permet d'affecter plusieurs valeurs à plusieurs variables en même temps.

\begin{pyc}\begin{minted}{python}
>>> a, b = 10, 2
>>> a
10

>>> b
2

>>> a, b = b, a # permet d'échanger a et b
>>> a
2

>>> b
10

\end{minted}
\end{pyc}

\subsection{Notation condensée}
On est souvent amené à écrire des instructions telles que  \mintinline{python}{a = a + 1} ou \mintinline{python}{b = b / 2}. Cela peut être lourd quand les variables ne s'appellent
pas \mintinline{python}{a} ou \mintinline{python}{b} mais \mintinline{python}{rayon_sphere} ou \mintinline{python}{largeur_niveau}. On peut utiliser les notation suivantes :
\begin{pyc}\begin{minted}{python}
>>> rayon_sphere = 3.4
>>> rayon_sphere /= 2 # rayon_sphere = rayon_sphere / 2
>>> rayon_sphere
1.7

>>> largeur_niveau = 19
>>> largeur_niveau += 1 # largeur_niveau = largeur_niveau + 1
>>> largeur_niveau
20
\end{minted}
\end{pyc}

On dispose également de \mintinline{python}{*=}, \mintinline{python}{//=}, \mintinline{python}|%=|, \mintinline{python}{-=} et \mintinline{python}{**=}.

\section{Le cas des variables de type str ou list}

\subsection{Les str}

Les valeurs de type \mintinline{python}{str} sont composées de caractères \emph{alphanumériques}. On peut accéder à chacun d'eux de la manière suivante :
\begin{pyc}\begin{minted}{python}
>>> chaine = 'Bonjour !'
>>> chaine[0]
'B'
>>> chaine[5]
'u'
\end{minted}
\end{pyc}

Voici comment \textsc{Python} représente  la chaîne précédente :

\begin{center}
	\alternaterowcolors
	\tabstyle[UGLiBlue]
	\begin{tabular}{|c|c|c|c|c|c|c|c|c|c|}
		\hline
		\ccell  i    &  0 &   1 &   2 &   3 &   4 &   5 &   6 &   7 &   8 \\
		\hline
		\ccell \mintinline{python}{chaine[i]} & B      & o      & n      & j      & o      & u      & r      &        & !      \\
		\hline
	\end{tabular}
\end{center}

On a parfois besoin de connaître la longueur (\emph{length} en anglais) d'une chaîne de caractères :

\begin{pyc}\begin{minted}{python}
>>> chaine = 'onzelettres'
>>> len(chaine)
11
\end{minted}
\end{pyc}

On peut aussi accéder facilement au dernier (ou à l'avant dernier) caractère d'une variable de type \mintinline{python}{str} :

\begin{pyc}
	\begin{minted}{python}
>>> a = "M'enfin ?!"
>>> a[-1]
'!'

>>> a[-2]
'?'
\end{minted}
\end{pyc}

\subsection{Les list}

Cela se passe un peu comme pour les \mintinline{python}{str}.

\begin{pyc}
	\begin{minted}{python}
		>>> lst = [3, 4, 8]
		>>> lst[1] # élément d'indice 1 de lst
		4

		>>> lst[-1] # dernier élément de lst
		8

		>>> len(lst) # longueur de la liste
		3
	\end{minted}
\end{pyc}

Lorsqu'on essaie d'accéder à un élément dont l'indice est supérieur ou égal à la longueur de la liste, on obtient une erreur. Dans l'exemple précédent si on évalue \mintinline{python}{lst[3]} on obtient :
\color{UGLiRed}
\begin{verbatim}
IndexError: list index out of range	
\end{verbatim}
\color{black}

Un chapitre est consacré à l'étude détaillée des \mintinline{python}{list}.

