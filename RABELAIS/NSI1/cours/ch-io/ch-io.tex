\chapter{Entrées-sorties et transtypage}

\section{Entrées-sorties}

Lorsqu'on écrit un programme en \textsc{Python}, on a très souvent besoin
\begin{itemize}
    \item qu'il affiche des informations à l'écran ;
    \item qu'il demande des informations à l'utilisateur\cdot trice.
\end{itemize}

\subsection{la fonction print}

C'est elle qui sert à afficher à l'écran :

\begin{pyc}
    \begin{minted}{python}
        print("Bonjour") # affiche bonjour
        print(4) # affiche 4
        print(2 + 4) # évalue 2 + 4 et affiche 6
        a = 3
        print(a) évalue a et affiche 3
    \end{minted}
\end{pyc}

On peut s'en servir de diverses manières :
\begin{pyc}
    \begin{minted}{python}
        age = 16
        print("Vous avez", age, "ans.")
    \end{minted}
\end{pyc}
Dans ce premier cas, on a donné 3 paramètres à \mintinline{python}{print}:
\begin{itemize}
    \item le \mintinline{python}{str "Vous avez"} ;
    \item la variable \mintinline{python}{age} ;
    \item  le \mintinline{python}{str "ans."}.
\end{itemize}
On peut aussi utiliser ce qu'on appelle une « f-string » de la manière suivante :
\begin{pyc}
    \begin{minted}{python}
        age = 16
        print(f"Vous avez {age} ans.") # noter le f au début
    \end{minted}
\end{pyc}
Dans ce cas, \textsc{Python} construit le \mintinline{python}{str} à afficher en y reportant la valeur de \mintinline{python}{age}.\\
C'est cette méthode que nous utiliserons par la suite.\\

Enfin, \mintinline{python}{print} revient à la ligne par défaut, mais on peut changer cela :
\begin{pyc}
    \begin{minted}{python}
        print("Bonjour",end="") # aucun retour à la ligne
        print("à tous",end="!") # aucun retour et ! à la fin
    \end{minted}
\end{pyc}
Ce programme affiche \texttt{Bonjour à tous!} sans revenir à la ligne.

\subsection{La fonction input}
C'est elle qui est chargée de récupérer des informations que l'utilisateur\cdot trice entre au clavier.
\begin{pyc}
    \begin{minted}{python}
        nom = input("Entrez votre nom : ")
    \end{minted}
\end{pyc}
Le programme précédent
\begin{itemize}
    \item affiche \mintinline{python}{Entrez votre nom : } ;
    \item attend que l'utilisateur\cdot trice tape quelque chose, puis valide avec la touche d'entrée ;
    \item stocke le message tapé dans la variable \mintinline{python}{nom}.
\end{itemize}
\begin{encadrecolore}{Important}{UGLiRed}
    La valeur renvoyée par la fonction \mintinline{python}{input} est \textit{toujours} de type \mintinline{python}{str}. Cela va donc nous amener à faire du \textit{transtypage}.
\end{encadrecolore}

\section{Transtypage}

\begin{definition}[ : transtypage]
    C'est l'action de changer le type d'une valeur ou d'une variable en un autre type \textit{compatible}.
\end{definition}

Voici un exemple :
\begin{pyc}
    \begin{minted}{python}
        a = 2 # a est un int
        b = float(a) # la valeur 2 est convertie en float et stockée dans b
        print(b) # affiche 2.0
    \end{minted}
\end{pyc}

En voici un autre dont nous nous servirons beaucoup
\begin{pyc}
    \begin{minted}{python}
        a = '12' # a est un str
        b = int(a) # '12' est converti en l'entier 12
        print(b + 1) # évalue b + 1 à 13 et affiche 13
    \end{minted}
\end{pyc}

\begin{encadrecolore}{Attention}{UGLiRed}
    Le transtypage n'est pas une action anodine et peut produire des erreurs lorsque la valeur à transtyper est incompatible avec le nouveau type qu'on veut lui donner.
\end{encadrecolore}

\begin{pyc}
    \begin{minted}{python}
        a = "Salut"
        b = int(a) # impossible à réaliser
    \end{minted}
    {\color{UGLiRed}\texttt{ValueError: invalid literal for int() with base 10: 'Salut'}}
\end{pyc}
Par ce message d'erreur, \textsc{Python} nous indique que la valeur \mintinline{python}{'Salut'} ne peut pas être envisagée comme l'écriture en base 10 d'un entier.

\section{Bilan}
Dès l'écriture de programmes simples, le recours aux fonctions \mintinline{python}{print}, \mintinline{python}{input} et au transtypage sont inévitables, comme le montre le programme suivant :

\begin{pyc}
    \begin{minted}{python}
        age = input("Entrez votre âge : ") # on obtient un str
        age = int(age) # on convertit age en int
        nouveau = age + 10 # qui nous permet de calculer
        print(f"Dans 10 ans, vous aurez {nouveau} ans.")
    \end{minted}
\end{pyc}

Il existe d'autres actions de transtypage que \mintinline{python}{str} $\rightarrow$ \mintinline{python}{int} :
\begin{itemize}
    \item \mintinline{python}{int} $\rightarrow$ \mintinline{python}{str} ;
    \item \mintinline{python}{float} $\rightarrow$ \mintinline{python}{int}, qui induit souvent une perte d'information car la partie décimale de la valeur disparaît ;
    \item \mintinline{python}{str} $\rightarrow$ \mintinline{python}{list} qui transforme par exemple \mintinline{python}{"abc"} en \mintinline{python}{["a", "b", "c"]}.  
\end{itemize}
