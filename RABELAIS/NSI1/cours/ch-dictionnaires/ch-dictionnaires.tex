\chapter{Dictionnaires}
\section{Un nouveau type}

\picright{0.3}{ch-dictionnaires/img/dico}{
    On demande à des jeunes quel est leur sport préféré, les résultats sont présentés sur la figure ci-contre.\\
    Un sport peut être cité par \textit{plusieurs} jeunes, en revanche chaque jeune ne peut citer qu'\textit{un seul} sport.\\
    On pourrait utiliser une ou plusieurs listes pour représenter ces données mais il y a mieux : le \textit{dictionnaire}.
    La variable \mintinline{python}{sport} est de type \mintinline{python}{dict} :
}
\begin{pyc}
    \begin{minted}{python}
        sport = {'A': 'Tennis', 'B': 'Basket',
                 'C': 'Judo', ' D': 'Boxe',
                 'E': 'Foot', 'F': 'Gym',
                 'G': 'Foot'}
        \end{minted}
\end{pyc}


\begin{definition}[ : dictionnaire]
    Un dictionnaire est un ensemble d'\textit{éléments}.\\
    les éléments sont des couples de la forme \textit{clé : valeur}.\\

    La syntaxe est :\\ \mintinline{python}{variable = { cle1 : valeur1, cle2 : valeur2, ...}}\\

    Les valeurs peuvent être de n'importe quel type. Les clés peuvent être
    \begin{itemize}
        \item 	des \mintinline{python}{bool}, des \mintinline{python}{int}, des \mintinline{python}{float};
        \item 	des \mintinline{python}{str}...
        \item 	mais pas des \mintinline{python}{list} !
    \end{itemize}
\end{definition}


On peut tout de même utiliser des \mintinline{python}{tuples} en guise de clés :  les \mintinline{python}{tuples} ressemblent aux \mintinline{python}{list} mais sont \textit{non mutables}.\\

\mintinline{python}{a = (1, 2, 3)} est un exemple de \mintinline{python}{tuple}.

\section{Opérations sur les dictionnaires}

\subsection{Accéder à une valeur par sa clé}
Pour connaître le sport préféré de \mintinline{python}{'A'}, c'est simple :
\begin{pyc}
    \begin{minted}{python}
>>> sport['A']
'Tennis'        
    \end{minted}
\end{pyc}


\subsection{Créer de nouveaux couples clé: valeur}

Contrairement aux listes, il n'y a pas de méthode \texttt{append}.\\
Pour intégrer l'information « le sport préféré de H est le Rugby » on écrira simplement :\\
\begin{pyc}
    \begin{minted}{python}
        >>> sport['H']='Rugby'
    \end{minted}
\end{pyc}


\subsection{Créer un dictionnaire vide et le peupler}

On peut partir d'un dictionnaire vide et remplir ses valeurs au fur et à mesure :

\begin{pyc}
    \begin{minted}{python}
        >>> d = dict()
        >>> d['bonjour'] = 'hello'
        >>> d['crayon'] = 'pencil'
        >>> d['se prélasser'] = 'to bask'
    \end{minted}
\end{pyc}
\subsection{Supprimer un élément du dictionnaire}

\mintinline{python}{del d['crayon']} supprime l'élément \mintinline{python}{'crayon': 'pencil'}.

\subsection*{fusionner 2 dictionnaires}
\begin{minted}{python}
>>> d1 = {"anglais": "bread", 
          "français": "pain", 
          "slovaque": "chlieb"}
>>> d2 = {"allemand": "brot", "italien": "pane"}
>>> d1.update(d2) # fusionne d2 dans d1
>>> d1
{"anglais": "bread", "français": "pain", "slovaque": "chlieb", "allemand": "brot", "italien": "pane"}
\end{minted}

\subsection{Parcourir l'ensemble des clés d'un dictionnaire}
\begin{pyc}
    \begin{minted}{python}
    for cle in d1.keys():
        print(cle)
    \end{minted}
\end{pyc}
Ce script affiche
\begin{minted}{python}
anglais
français
slovaque
allemand
italien
\end{minted}

\subsection{Parcourir l'ensemble des valeurs d'un dictionnaire}
\begin{pyc}
    \begin{minted}{python}
for valeur in d1.values():
    print(valeur)
\end{minted}
\end{pyc}
Ce script affiche
\begin{minted}{python}
bread
pain
chlieb
brot
pane
\end{minted}

\subsection{Précisions}
\mintinline{python}{d1.keys()} et \mintinline{python}{d1.values()} ressemblent à des listes mais n'en sont pas !\footnote{Ce sont des \textit{itérateurs}, sctructures destinées à être parcourues. }\\

Pour avoir par exemple la liste des clés de \mintinline{python}{d1} on écrira :\\

\mintinline{python}{list(d1.keys())}\\

ou bien en \textit{compréhension} (ce qui revient au même mais peut s'avérer utile)

\mintinline{python}{[k for k in d1.keys()]}

\subsection{Erreurs de clé}
\begin{minted}{python}
print(d1['suédois'])
\end{minted}

Ce script produit une erreur :

{\color{red}\texttt{KeyError : 'suédois'}}

\begin{exemple}[ : Utilisation d'un dictionnaire]
    On veut créer un tableau de $10\times 10$ cases avec la valeur 0 dedans.\\

    On peut bien sûr créer cela avec une liste de listes (en compréhension) mais on peut également utiliser un dictionnaire :\\

    \mintinline{python}{{(x, y) : 0 for x in range(1, 11) for y in range(1, 11)}}\\

    \textbf{Avantages :}
    \begin{itemize}
        \item 	plus simple à manipuler : on écrit \mintinline{python}{d[x, y]} au lieu de \mintinline{python}{d[x][y]} ;
        \item 	on n'est pas obligé de faire commencer les indices à zéro.
    \end{itemize}

    \textbf{Inconvénients :}
    \begin{itemize}
        \item prend plus de place en mémoire (on s'en fiche un peu) ;
        \item plus flexible entraîne plus de possibilité d'erreurs !
    \end{itemize}
\end{exemple}


\section{Utilisation des dictionnaires}


Typiquement, pour stocker des \textit{données structurées} :    \\
\begin{minted}{python}
reseau = {'nom'        : 'local',
          'ip'         : '192.168.1.0',
          'masque'     : '255.255.255.0',
          'passerelle' : '192.168.1.254'}
\end{minted}

On utilise fréquemment des listes de dictionnaires, ou bien des dictionnaires de listes.
