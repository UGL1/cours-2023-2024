\documentclass[12pt]{nsibeamer}

\title{Correction}
\subtitle{Le Roi et ses nains}
\author{NSI2}

\begin{document}
	\maketitle

\begin{frame}{Résultat préliminaire}
    \pause
    Chaque puissance de 2 est plus grande que la somme de toutes celles qui précèdent.\pause\\

    Plus précisément :\pause

    Soit $n\in\N$,\pause
    $$2^{n+1} > 2^n + 2^{n-1}+\ldots + 2^1+2^0$$
\end{frame}

\begin{frame}{Preuve}
    $2^{n+1} + 2^{n}+\ldots + 2^1+2^0$ s'écrit $ \left(\underbrace{1\ldots 1}_{n+1\textrm{ chiffres}}\right)_2$ en binaire.\\\pause
    On a montré que cela vaut $2^{n+1}-1$ dans un exercice précédent.\\
    Donc c'est bien plus petit que $2^n$.
\end{frame}

\begin{frame}[standout]
    \Large
    Ce résultat nous sert pour résoudre l'exercice.
\end{frame}
\begin{frame}{Solution}
    \pause
    Le roi dispose des nains, numérotés de 0 à 7.\\\pause
    Il demande au nain numéro $n$ d'apporter $2^n$ pièces.\\\pause
    Il pèse l'ensemble et regarde combien il manque de grammes par rapport à la somme.\\\pause

    Soit $M$ ce nombre. On l'écrit en base 2.\pause

    $$M = (b_7b_6b_5b_4b_3b_2b_1b_0)_2$$\pause

    Les voleurs sont les nains dont le bit est à 1 !
\end{frame}

\begin{frame}{Preuve}
    Si le nain 7 est voleur il manque au moins 128 grammes et $b_7$ est à 1.\\\pause
    Réciproquement s'il manque au moins 128 grammes c'est que le bit $b_7$ est à 1 car même si tous les autres nains sont voleurs, il ne peuvent au total voler que $2^6+\ldots+2^0=127$ grammes. Donc le nain 7 est un voleur.\\\pause 
    
    Si c'est le cas on met le nain 7 au cachot et $M$ devient $M-128$.\\\pause
\end{frame}
\begin{frame}{Preuve}
    Puis on recommence avec le nain 6 et 64 grammes.\\\pause

    Et ainsi de suite. 
\end{frame}
\end{document}