\chapter{Valeurs et types}
\label{ch:valeurs}
\section{Python, machine à évaluer}
\textsc{Python} est en premier lieu une machine à calculer, ou plutôt une machine à \textit{évaluer} : lorsque \textsc{Python} rencontre une \textit{expression}, c'est-à-dire une écriture qui produit une valeur, il commence par déterminer cette valeur.\\
Pour s'en rendre compte, il suffit d'écrire des expressions dans une \textit{console}.
\begin{pyc}
  \begin{minted}{python}
        >>> 5 - 3
        2
        
        >>> 11 / (1 + 2)
        3.6666666666666665

        >>> 30 / 15
        2.0
    \end{minted}
\end{pyc}
On se rend compte que les valeurs rencontrées ne sont pas présentées de la même manière :\\ \mintinline{python}{5 - 3} a la valeur \mintinline{python}{2} alors que \mintinline{python}{30 / 15} a la valeur \mintinline{python}{2.0}.\\
Pour y voir plus clair, on peut appeler la fonction \mintinline{python}{type} qui
\begin{itemize}
  \item en entrée prend une expression ;
  \item renvoie le \textit{type} de l'expression.
\end{itemize}
\begin{pyc}
  \begin{minted}{python}
    >>> type(2)
    <class int>

    >>> type(2.0)
    <class float>
  \end{minted}
\end{pyc}

Il y a donc au moins deux types de valeurs, le type \mintinline{python}{int} et le type \mintinline{python}{float}.\\
En fait il existe une multitude de types prédéfinis selon la nature de la valeur à représenter et nous allons les passer en revue.


\section{Le type int}
\label{sec:int}
Il sert à représenter les \textit{entiers relatifs} (\textit{integer} signifie  « entier » en Anglais).\\
Le type \mintinline{python}{int} dispose des opérations \mintinline{python}{+} (addition), \mintinline{python}{-} (soustraction) et \mintinline{python}{*} (multiplication).

\begin{pyc}\begin{minted}{python}
>>> 3 + 2
5

>>> 2 * 3
6

>>> 3 - 2 * 2
-1

>>> 10_000 # on utilise des _ pour séparer les chiffres
\end{minted}
\end{pyc}

Les parenthèses sont utilisées comme en mathématiques, pour indiquer une \textit{priorité opératoire}. Cependant les crochets et les accolades sont réservés à un autre usage.

\begin{pyc}
  \begin{minted}{python}
    >>> (3 + 4) * 5
    35
  \end{minted}
\end{pyc}

On dispose également de \textit{deux opérations très pratiques} : soient \mintinline{python}{a} et \mintinline{python}{b} deux \mintinline{python}{int}, et \mintinline{python}{b} non nul, alors on
\begin{itemize}
  \item \mintinline{python}{a // b} est le \textit{quotient} de la \textit{division euclidienne} de \mintinline{python}{a} par \mintinline{python}{b} ;
  \item \mintinline{python}|a % b| est le \textit{reste}.
\end{itemize}


\begin{pyc}\begin{minted}{python}
>>> 64 // 10 # 64, c'est 6 * 10 + 4
6

>>> 64 % 10
4

>>> 22 // 7 # 22, c'est 3 * 7 + 1
3

>>> 22 % 7
1
\end{minted}
\end{pyc}


On dispose de l'opération d'\textit{exponentiation} (opération puissance), notée \mintinline{python}{**}.\\
\textbf{Attention :} Cette opération peut produire un résultat \textit{non-entier}, de type \mintinline{python}{float} (voir partie suivante).

\begin{pyc}\begin{minted}{python}
>>> 2 ** 3
8

>>> 10 ** 4
10000

>>> 2 ** (-1)
0.5
\end{minted}
\end{pyc}

Pour finir, la \textit{division décimale} peut être effectuée sur des entiers, mais elle renvoie un résultat de type \mintinline{python}{float}.

\section{Le type float}
Il sert à représenter les \textit{nombres à virgule flottante} (\textit{to float} : flotter en Anglais). Ce sont (en gros) des nombres
décimaux.\ \textsc{Python} comprend et utilise la notation scientifique : \mintinline{python}{2.35e6} vaut $2,35\times 10^6$, c'est-à-dire 2 350 000.

\begin{pyc}\begin{minted}{python}
>>>  2 / 7
0.2857142857142857 # c'est une valeur approchée

>>> 1 / 100_000
1e-05

>>> 1.2e-4
0.00012
\end{minted}
\end{pyc}

On peut pratiquer sur les \mintinline{python}{float} toutes les opérations vues avec les \mintinline{python}{int}. Pour des fonctions plus compliquées telles le cosinus ou
l'exponentielle, on fait appel au module\footnote{Un module est un ensemble de \textit{fonctions} et/ou de \textit{constantes} que l'on peut importer.}
\mintinline{python}{math} :

\begin{pyc}\begin{minted}{python}
>>> from math import *
>>> pi
3.141592653589793

>>> cos(pi / 3)
0.5000000000000001

>>> exp(2)
7.38905609893065

>>> log(2)
0.6931471805599453

>>> exp(log(2))
2.0
\end{minted}
\end{pyc}

\mintinline{python}{exp} est la \textit{fonction exponentielle} et \mintinline{python}{log} la \textit{fonction logarithme népérien}\footnote{Voir le programme de mathématiques
  de terminale scientifique.} notée $\ln$ en France.



\section{Le type str}

Il sert à représenter les \textit{chaînes de caractères} (\mintinline{python}{str} est l'abréviation de \textit{string}, qui veut dire chaîne en anglais).
Lorsqu'on écrit une valeur de type \mintinline{python}{str}, on peut utiliser les symboles \mintinline{python}{'}, \mintinline{python}{"} ou même \mintinline{python}{'''} (suivant que la chaîne contient des apostrophes, ou des guillemets).

\begin{pyc}\begin{minted}{python}
>>> 'Bonjour.'
'Bonjour'

>>> 'J'aime Python.'
SyntaxError

>>> "J'aime Python."
"J'aime Python."

>>> "Je n'aime pas qu'on m'appelle "geek"."
SyntaxError

>>> """Je n'aime pas qu'on m'appelle "geek"."""
'Je n\'aime pas qu\'on m\'appelle "geek".'
\end{minted}
\end{pyc}
La dernière évaluation produit une valeur correcte. \textsc{Python} utilise simplement \mintinline{python}{\'} pour écrire les apostrophes qui sont à l'intérieur de la valeur.\\


Le symbole \mintinline{python}{+} sert à \textit{concaténer 2 chaînes}, c'est-à-dire à les mettre bout à bout.

\begin{pyc}\begin{minted}{python}
>>> 'Yes' + 'No'
'YesNo'

>>> 'No' + 'Yes'
'NoYes'
\end{minted}
\end{pyc}

On peut même multiplier un \mintinline{python}{str} par un \mintinline{python}{int} :
\begin{pyc}
  \begin{minted}{python}
    >>> 3 * 'Aïe ! '
    'Aïe ! Aïe ! Aïe ! '
  \end{minted}
\end{pyc}
\textsc{Python} évalue \mintinline{python}{3*'Aïe !'} comme \mintinline{python}{'Aïe ! ' + 'Aïe ! ' + 'Aïe ! '}.


Voici deux types très utiles que nous étudierons en détail plus tard.

\section{Le type list}

Une valeur de type \mintinline{python}{list} est une... liste ordonnée de valeurs. Celles-ci peuvent être du même type ou non.

\begin{pyc}
  \begin{minted}{python}
    >>> [] # liste vide
    []

    >>> [1, 4, 5] # liste comportant 3 int
    [1, 4, 5]

    >>> [2.0, -4, 'Bonjour', 'Coucou']
    [2.0, -4, 'Bonjour', 'Coucou']
  \end{minted}
\end{pyc}

L'intérêt de ce type est de rassembler plusieurs valeurs au sein d'une seule, qui pourra ensuite être \textit{parcourue}.

\section{Le type dict}

Celui-ci sert à établir des \textit{associations} du type \textit{clé : valeur}.

\begin{pyc}
  \begin{minted}{python}
      >>> {} # dictionnaire vide
      {}

      >>> {'France': 'Paris', 'Canada': 'Ottawa', 'Pays-Bas': 'La Haye'}
      {'France': 'Paris', 'Canada': 'Ottawa', 'Pays-Bas': 'La Haye'}
    \end{minted}
\end{pyc}

Tout comme le type \mintinline{python}{list}, ce type sert à \textit{structurer les données}. L'exemple précédent fait correspondre des capitales à des pays.



\section{Le type bool}
Il sert à représenter les \textit{valeurs booléennes}, valant \mintinline{python}{True} (vrai) ou
\mintinline{python}{False} (faux).

\begin{pyc}\begin{minted}{python}
>>> False
False

>>> True
True
\end{minted}
\end{pyc}

Cela peut paraître un peu pauvre, c'est trompeur : les \textit{expressions logiques} sont des écritures dont la valeur est un booléen. Lorsque \textsc{Python} les rencontre, il les évalue pour trouver soit \mintinline{python}{True} soit \mintinline{python}{False}.

\begin{pyc}
  \begin{minted}{python}
    >>> 3 >= 2 # évalue si 3 est supérieur ou égal à 2
    True
    
    >>> 3 + 5 == 2 # évalue si 3 + 5 vaut 2
    False

    >>> 1 in [3, 4, 1, 5] # évalue si 1 est un élément de la liste
    True
  \end{minted}
\end{pyc}

\begin{encadrecolore}{Attention}{UGLiRed}
  Pour \textit{tester} si deux valeurs sont égales, on utilise \mintinline{python}{==}  (et pas \mintinline{python}{=} ).

\end{encadrecolore}


Ce type dispose d'\textit{opérations logiques} : \mintinline{python}{or} (ou), \mintinline{python}{and} (et) et \mintinline{python}{not} (non).

\begin{pyc}\begin{minted}{python}
>>> True and False # vrai que si les 2 sont vrais
False

>>> True or False # faux que si les 2 sont faux
True

>>> not (3 < 1) # contraire
True

>>> (2 < 1) or (3 >= 0)
True
\end{minted}
\end{pyc}